\section{Lecture 11}

We saw last time in some computations that we need a better understanding of the unit group $\Ocal_{K}^{\times}$ where $K$ is a number field.
To obtain this, we will use a multiplicative version of Minkowski theory.

Consider the map
\[ j : K^{\times} \to K_{\Cbb}^{\times} = \prod_{\tau} \Cbb^{\times} \]
induced by the (additive) map $j : K \to K_{\Cbb}$ from before.
Taking products yields a map
\[ N : K_{\Cbb}^{\times} \to \Cbb^{\times} \]
which is the usual norm $N_{K|\Qbb}$ when composed with $j$ above.

Since we would still like to obtain a lattice at the end of the day, we need to convert from a multiplicative back to an additive version, and we ddo this by taking logs.
Namely, consider the map $\ell : z \mapsto \log|z|$.
This induces a surjective map
\[ \ell : K_{\Cbb}^{\times} \to \prod_{\tau} \Rbb. \]
Note that $\ell(N(z_{\tau})_{\tau}) = \Trace(\ell(z_{\tau})_{\tau})$, where $\Trace : \Rbb^{n} \to \Rbb$ is the summation map.

We have a natural action of complex conjugation on this whole situation.
On $K$, $K_{\Cbb}$ it acts as before, and on $\prod_{\tau} \Rbb$ it acts by interchanging the $\tau$ coordinate with the $\bar\tau$ coordinate.
The maps above are all compatible with this action, so we may take invariants to obtain maps
\[ j : K^{\times} \to K_{\Rbb}^{\times}, \ \ell : K_{\Rbb}^{\times} \to (\prod_{\tau} \Rbb)^{\Gal(\Cbb|\Rbb)} \]
and
\[ N : K_{\Rbb}^{\times} \to \Rbb^{\times}, \ \Trace : (\prod_{\tau} \Rbb)^{\Gal} \to \Rbb. \]

We can describe $(\prod_{\tau} \Rbb)^{\Gal}$ explicitly.
It is
\[ \prod_{i = 1}^{r} \Rbb \times \prod_{i = 1}^{s} \Delta \]
where $\Delta \subset \Rbb \times \Rbb$ is the diagonal.
Identify $\Delta$ with $\Rbb$ via $(x,x) \mapsto 2x$ (to preserve compatibility with the trace) to obtain that $(\prod_{\tau} \Rbb)^{\Gal} = \Rbb^{r + s}$.

Next, consider the following:
\begin{enumerate}
  \item $S_{K} := \ker(z \mapsto |N(z)| : K_{\Rbb}^{\times} \to \Rbb^{\times})$.
  \item $H_{K} := \ker(z \mapsto \Trace(z) : (\prod_{\tau} \Rbb)^{\Gal} \to \Rbb)$.
\end{enumerate}
Note that $\ell$ restricts to a map $S_{K} \to H_{K}$ and that $j$ \todo{restricts to a map}{Why?} $j : \Ocal_{K}^{\times} \to S_{K}$.
Write $\Lambda_{K}$ for the image of $\Ocal_{K}^{\times}$ in $H$ and $\lambda : \Ocal_{K}^{\times} \to H_{K}$ the corresponding map (with image $\Lambda_{K}$).

\begin{proposition}
  The kernel of $\lambda$ is $\mu(K)$, the group of roots of unity in $K$.
\end{proposition}
\begin{proof}
  It's clear that the kernel contains the roots of unity.
  Conversely, an element $x$ of the kernel satisfies $|\tau x| = 1$ for all complex embeddings $\tau$, and by homework 1, it follows that $x$ is a root of unity.
\end{proof}

\begin{proposition}
  The group $\mu(K)$ of roots of unity is finite.
\end{proposition}
\begin{proof}
  Note that $\mu(K) \subset \Ocal_{K}$.
  Embed this set into $K_{\Rbb}$ via $j$ and recall that the image of $\Ocal_{K}$ is a lattice.
  The image of $\mu_{K}$ is contained in the set of elements $(z_{\tau})_{\tau}$ of $K_{\Rbb}^{\times}$ such that $|z_{\tau}| = 1$ for all $\tau$, and this set is bounded (in fact, it's compact), so its intersection with any lattice (such as $j(\Ocal_{K})$) is finite.
\end{proof}

\begin{theorem}
  The group $\Lambda_{K}$ is a complete lattice in $H$.
  In particular, $\Lambda_{K}$ is isomorphic to $\Zbb^{r + s - 1}$.
\end{theorem}
\begin{proof}
  Note that $H_{K}$, being a hyperplane in a $r+s$-dimension real vector space, has dimension $r + s - 1$, so we only need to prove that $\Lambda_{K}$ is a lattice in this vector space.
  For this, we show that $\Lambda_{K}$ is both discrete and cocompact.

  For discreteness, let $c > 0$ be some real number, and consider the set
  \[ U_{c} := \{(x_{\tau}) \in \prod_{\tau} \Rbb \ | \ |x_{\tau}| \le c\}. \]
  The preimage of $U_{c}$ with respect to $\ell$ in $K_{\Cbb}^{\times}$ is the set
  \[ \{(z_{\tau})_{\tau} \in K_{\Cbb}^{\times} \ | \ e^{-c} \le |z_{\tau}| \le e^{c} \}. \]
  Since $j \Ocal_{K}$ is a lattice in $K_{\Cbb}$, this set (being bounded) contains only finitly many elements of $j \Ocal_{K}$ hence also of $j \Ocal_{K}^{\times}$.
  It thus follows that $\Lambda_{K}$ is discrete.

  To see that this lattice is a complete lattice, we must find a bounded set $M$ in $H_{K}$ such that $M + \Lambda_{K} = H_{K}$.
  Consider the surjective map
  \[ \ell : S_{K} \to H_{K}. \]
  The goal is to construct a bounded set $M'$ of $S_{K}$ such that $S = M' \cdot j(\Ocal_{K}^{\times})$.
  Since elements of $S_{K}$ have norm $1$, the elements of $M'$ will also be bounded away from $0$ in $K_{\Rbb}$ and thus $\ell(M') = M$ will be bounded and will satisfy the necessary conditions.

  Choose real numbers $c_{\tau} > 0$ satisfying $c_{\tau} = c_{\bar \tau}$ and put
  \[ C := \prod_{\tau} c_{\tau}. \]
  Choose $c_{\tau}$ large enough so that $C > (2/\pi)^{s} \sqrt{|d_{K}|}$, and consider the set
  \[ X := \{ (z_{\tau}) \in K_{\Rbb} \ | \ |z_{\tau}| < c_{\tau} \}. \]
  If $y = (y_{\tau})_{\tau} \in S$, then
  \[ X \cdot y = \{ (z_{\tau})_{\tau} \in K_{\Rbb} \ | \ |z_{\tau}| < c'_{\tau}\}, \]
  where $c'_{\tau} = c_{\tau}|y_{\tau}|$, while also $c'_{\tau} = c'_{\bar\tau}$ and
  \[ C' := \prod_{\tau} c'_{\tau} = C \]
  since the product of the $|y_{\tau}|$ is $1$.
  By Theorem~\ref{theorem:exists_norm_mult}, we can find a nonzero $a \in \Ocal_{K}$, such that $ja \in Xy$.

  Next, let's note that there are only finitely many elements of $\Ocal_{K}$ with a given norm, up-to multiplication by $\Ocal_{K}^{\times}$.
  Indeed, if $a$ is some positive integer, then \todo{there are at most}{Add proof. See Lemma 7.2 of Neukirch.} $[\Ocal_{K} : a \cdot \Ocal_{K}]$ elements of norm $\pm a$ up-to multiplication by units.
  Thus, we may choose finitely many nonzero elements $\alpha_{1},\ldots,\alpha_{N} \in \Ocal_{K}$ such that for every $a \in \Ocal_{K}$ with $0 < |N(a)| \le C$, there exists some $i$ such that $a \in \alpha_{i} \cdot \Ocal_{K}^{\times}$.

  The set $M' := S \cap \bigcup_{i} X \cdot (j \alpha_{i})^{-1}$ will satisfy the required properties.
  Indeed, it is bounded since $X$ is bounded and this is a finite union of multiplicative translates of $X$.
  Furthermore, we have
  \[ S = M' \cdot j(\Ocal_{K}^{\times}).\]
  Indeed, if $s \in S$ is given, then we may find some nonzero $a \in \Ocal_{K}$ such that $j a \in X \cdot s^{-1}$, say $ja = x \cdot s^{-1}$.
  Computing norms, we find
  \[ |N(a)| = |N(xs^{-1})| = |N(x)| \le C, \]
  so that, $a \in \alpha_{i} \cdot \Ocal_{K}^{\times}$ for some $i$.
  It follows that
  \[ s = x ja^{-1} \in x j(\alpha_{i}^{-1} \cdot \Ocal_{K}^{\times}). \]
  As both $s$ and $j(\Ocal_{K}^{\times})$ are contained in $S$, we see that indeed $s \in M' \cdot j(\Ocal_{K}^{\times})$, as required.
\end{proof}

We can now completely understand the structure of $\Ocal_{K}^{\times}$.
As an abelian group, it is simply
\[ \Ocal_{K}^{\times} \cong \mu(K) \times \Zbb^{r + s - 1}, \]
while $\mu(K)$ is a finite subgroup of $K^{\times}$, hence $\mu_{K}$ \todo{is cyclic.}{Exercise.}

\begin{proof}
  $\mu_{K}$ is a finite abelian group, let $a \in \mu{K}$ an element of maximal order.
  Now let $b \in \mu{K}$, we claim that the order of $a$ divides that of $b$.
  Assume not, then there is an prime $p$ such that for some natural number $d$, $p^d \ | \ ord(b)$, but such power of $p$ does not divide $ord(a)$.
  Take $d$ to be minimal such this occurs, so that $p^{d-1} \ | \ ord(a)$
  Now let $m = p^{d-1}$ and $n = ord(b)/p^d$, consider $h = n \cdot a + m \cdot b$.
  Then as $gcd(ord(n\cdot a), ord(m \cdot b)) = 1$ and our group is abelian, we have this new element has order $ord(n\cdot a) \cdot ord(m \cdot b) = p*ord(a)$.
  
  This contradicts the maximality of the order of $a$. Thus the order of all elements divides the order of $a$.

  We now claim that $a$ generates our group $\mu{K}$. Note that the polynomial $x^{ord(a)} - 1$ has a zero for every element of $\mu{K}$. 
  But a polynomial has at most its degree solutions. Thus $|\mu{K}| = ord(a)$, so $a$ generates $\mu{K}$.

  Thus, $\mu{K}$ is indeed cyclic, infact an finite subgroup of the group of units of any field is.
\end{proof}

We can also extract an important invariant of the number field $K$ from this discussion, called the \emph{regulator} of $K$.
Indeed, as $H$ is a codimension $1$ subspace of $\Rbb^{r+s}$, it inherits an inner product (where $\Rbb^{r+s}$ is endowed with the usual dot product).

We saw above that $\Lambda_{K}$ is a complete lattice in $H_{K}$, hence it has a well-defined covolume.
To compute this covolume, we choose $\epsilon_{1},\ldots,\epsilon_{k} \in \Ocal_{K}^{\times}$, $k = r+s-1$, which are a basis for $\Ocal_{K}^{\times}/\mu(K)$.
Such $\epsilon_{i}$ are called a system of \emph{fundamental units of $K$}.
The images $\lambda(\epsilon_{i})$ form a basis for the complete lattice $\Lambda_{K}$.
On the other hand, the vector
\[ \lambda_{0} = \frac{1}{\sqrt{r+s}}(1,\ldots,1) \in \Rbb^{r+s} \]
is orthogonal to $H_{K}$, and has length $1$.
We can thus compute the covolume of $\Lambda_{K}$ in $H_{K}$ by computing the covolume of the (complete) lattice spanned by $\lambda_{0},\lambda(\epsilon_{1}),\ldots,\lambda(\epsilon_{k})$ in $\Rbb^{r+s}$, which amounts to a determinant calculation.
More precisely, we take the absolute value of the determinant of the $(k+1)\times(k+1)$ matrix
\[ M := [\lambda_{0}|\lambda(\epsilon_{1})|\cdots|\lambda(\epsilon_{k})]. \]
If we add all the rows of this matrix to the $i$-th row, and use the fact that $\lambda(\epsilon_{j}) \in H_{K}$, we will see that the $i$-th row must contain all zeros, except for the leftmost entry which then \todo{becomes}{Add calculation.} $\sqrt{r+s}$.
We thus see that the absolute value of the determinant of the matrix in question, hence the covolume of $\Ocal_{K}^{\times}$ in $H_{K}$, can be calculated as $\sqrt{r+s} \cdot R$ where $R$ is the absolute value of a full rank minor of the $(k+1) \times k$ matrix
\[ M' := [\lambda(\epsilon_{1})|\cdots|\lambda(\epsilon_{k})]. \]
This number $R = R_{K}$ is called the \emph{regulator} of $K$.

Let's now go back to the exmaples from the last lecture.
The first one we considered and which we left unfinished was the computation of the class group $Cl_{K}$ for $K = \Qbb(\sqrt{82})$.
In this case, we have $n = 2$, $s = 0$, $r = 1$ so the rank of the unit group $\Ocal_{K}^{\times}$ is $1$.
We deduced last time that if $\pfrak_{2}$ is a prime lying above $2$, and $\pfrak_{3}$ is above $3$, then $\pfrak_{3}^{2} = \pfrak_{2}$ in the class group, while $\pfrak_{3}$ has order dividing $4$.
We also saw that $Cl_{K}$ is generated by $\pfrak_{3}$.
So to determine whether the class number is $2$ or $4$, we just have to check whether $\pfrak_{2}$ is principal (we will see that it's not, so the class number is $4$).

Indeed $\pfrak_{2}$ has norm $2$ as we saw last time, so assuming that $\pfrak_{2} = (x + y\sqrt{82})$ for some integers $x,y$, we would have $x^{2} - 82y^{2} = \pm 2$.
We will see that this has no integer solutions, as follows.
Suppose this is the case and put $x + y \sqrt{82} := \alpha$.
Then we have $(\alpha^{2}) = (2)$ since $(2) = \pfrak_{2}^{2}$ as we saw last time.
It follows that $\alpha^{2} = 2u$ for some unit $u \in \Ocal_{K}^{\times}$.
Let $\epsilon$ be a fundamental unit of $K$.

Let's consider the norm of $u$.
We have
\[ \Norm(u) \cdot \Norm(2) = 4 \Norm(u) = \Norm(\alpha)^{2}. \]
Since $\Norm(\alpha)$ is an integer, it follows that $\Norm(u) = 1$.
On the other hand, we have $\mu(K) = \{\pm 1\}$ since $K$ is totally real.
Thus $u = \pm \epsilon^{k}$ for some $k$.

Now note that $e := 9 + \sqrt{82}$ is a unit.
Thus
\[ e = \pm \epsilon^{j} \]
for some $j$.
Since $e$ has norm $-1$, we have $-1 = N(\epsilon)^{j}$ while $N(\epsilon) = \pm 1$, so that $N(\epsilon) = -1$.
As $u$ has norm $1$, it follows that $k$ must be even.
(In fact, you can show that $e$ \emph{is} a fundamental unit, but we don't need this.)
We have thus shown that $u = \pm \epsilon^{2k}$ for some $k$.
Going back, we have
\[ \alpha^{2} = 2u = \pm 2\epsilon^{2k}. \]
Thus either $2$ or $-2$ must be a square in $K$.
Since $82 = 2 \cdot 41$, we can easily see that this is not the case, for otherwise $K|\Qbb$ would have had to have degree $4$.

%%% Local Variables:
%%% mode: latex
%%% TeX-master: "main"
%%% End:
