\section{Lecture 21}

\begin{proposition}
  A local field is locally compact and its valuation ring is compact.
\end{proposition}
\begin{proof}
  It sufffices to show that the valuation ring is compact.
  We know that this valuation ring $\Ocal$ has the form
  \[ \Ocal \cong \varprojlim_{n} \Ocal/\mfrak^{n} \]
  while multiplication by $\pi^{n}$ ($\pi$ a uniformizer) induces an isomorphism between $\Ocal/\mfrak$ and $\mfrak^{n}/\mfrak^{n+1}$.
  This shows that $\Ocal/\mfrak^{n}$ is finite hence compact.
  By general topology, the same thus holds for the inverse limit and thus also for $\Ocal$.
\end{proof}

\begin{theorem}
  The local fields are the finite extensions of $\Qbb_{p}$ or of $\Fbb_{p}((t))$.
\end{theorem}
\begin{proof}[Proof Sketch]
  It is easy to see from what we have already done that such finite extensions are local, so we only focus on the forward direction.
  We will have to use Ostrowski's theorem, which we recall (without proof) during lecture.

  Let $K$ be a local field with absolute value $|-|$ and valuation $v$.
  Assume first that $K$ has characteristic $0$.
  The restriction of $v$ to $\Qbb$ is thus $v_{p}$ for some prime $p$ (Ostrowski), which must then be the residue characteristic of $K$.
  Taking the closure of $\Qbb$ in $K$ must therefore give the completion of $\Qbb$ with respect to the $p$-adic valuation, i.e.~$\Qbb_{p} \subset K$.
  To see that this extension is finite, one uses the local compactness of $K$.
  If $K$ has positive characteristic then we proceed similarly with $\Fbb_{p}(t)$ instead of $\Qbb$, where $t$ is a uniformizer of $K$.
\end{proof}

Let $K$ be a local field.
We consider the following invariants of $K$:
\begin{enumerate}
  \item The valuation ring $\Ocal = \Ocal_{K}$ with maximal ideal $\mfrak$.
  \item The absolute value $|-|$ on $K$.
  \item The residue field $\kappa := \Ocal/\mfrak$, which is \emph{finite}.
  \item The \emph{size} of the residue field $\kappa$, denoted by $q$.
  \item $U^{(n)}$ as before.
\end{enumerate}

Let us start by studying the multiplicative structure of a local field using these invariants.
\begin{theorem}
  Let $K$ be a local field.
  Suppose that $K$ has characteristic $0$.
  Then there is a (topological) isomorphism
  \[ K^{\times} \cong \Zbb \times \Zbb/(q-1) \times \Zbb/p^{a} \times \Zbb_{p}^{d} \]
  where $p^{a}$ is the (finite) size of $\mu_{p^{\infty}}(K)$ and $d = [K:\Qbb_{p}]$.
  If $K$ has positive characteristic $p$ then
  \[ K^{\times} \cong \Zbb \times \Zbb/(q-1) \times \Zbb_{p}^{\Nbb}. \]
\end{theorem}

We will need to prove some intermediate results and discuss some additional constructions before we can actually prove this.
Let's start.

\begin{proposition}
  Let $K$ be a local field and choose a uniformizer $\pi$ of $K$.
  The canonical map
  \[ \pi^{\Zbb} \times \mu_{q-1} \times U^{(1)} \to K^{\times}. \]
  is a (topological) isomorphism.
\end{proposition}
\begin{proof}
  \todo{Use Hensel's lemma plus general nonsense.}{Add proof.}
\end{proof}

At this point, our task is to describe the structure of $U^{(1)}$.
We will do this by considering analogues of the exponential and logarithm in the $p$-adic case.
In the equal characteristic case, we can use the discussion from previous lectures to see that $K \cong \Fbb_{q}((t))$, and do things ``explicitly'' (I won't prove this, see the argument in Neukirch, which follows an argument of Iwasawa).

\begin{proposition}
  Suppose $K = \Fbb_{q}((t))$.
  Then $U^{(1)} \cong \Zbb_{p}^{\Nbb}$.
\end{proposition}

From now on, let's focus on the $p$-adic case and assume $K$ is $p$-adic.
\begin{lemma}
  The series
  \[ \log(1+x) := x - x^{2}/2 + x^{3}/3 - \cdots \]
  converges in $K$ for $|x| < 1$.
\end{lemma}
\begin{proof}
  Let's compute the valuation of the terms of the series.
  \[ v(x^{n}/n) = n v(x) - v(n). \]
  But $v(x) > 0$.
  Say $\pi$ is a uniformizer and $v(\pi) = c$ so that $v(x) = m \cdot c$ for some positive integer $m$.
  Renormalize and choose $c$ so that $v(p) = 1$.
  Thus $p^{v(n)} \le n$, so that $v(n) \le \log(n)/\log(p)$.
  Put $r := p^{v(x)} > 1$.
  Thus we have
  \[ v(x^{n}/n) = n v(x) - v(n) \geq n \log_{p}(r) - \log_{p}(n) = \log_{p}(r^{n}/n). \]
  But $r^{n}/n$ goes to $\infty$ as $n \to \infty$ since $r > 1$ and so the same holds after taking logs.
  It follows that $x^{n}/n$ is a nullsequence so the series converges.
\end{proof}

This defines a function $\log : U^{(1)} \to K$ which is easily seen to satisfy $\log(x y) = \log(x) + \log(y)$ by the formal properties of the series.
\todo{We can extend this to all of $K^{\times}$ by setting $\log(p) = 0$.}{Explain how.}
In fact, $\log$ is uniquely determined by these properties and is a continuous homomorphism.

What about the exponential?
Consider the usual series definition of the exponential:
\[ \exp(x) = 1 + x + x^{2}/2 + x^{3}/3! + \cdots \]
For this to make sense, the series should converge, so we need to understand the region on which it does indeed converge.
The situation is more subtle in this context, since the denominators have absolute value which gets smaller and smaller in the $p$-adic context (why?).
We thus need $x$ to be sufficiently close to zero.
Let's unravel precisely when this happens.

\begin{proposition}
  Let $e$ denote the \emph{absolute} ramification degree of $K$.
  Then for all $n > e/(p-1)$, the series
  \[ \exp(x) = \sum_{i} x^{i}/i! \]
  converges for $x \in \mfrak^{n}$.
  Its image lands in $U^{(n)}$, and $\exp$ resp.~$\log$ induce topological isomorphisms $\mfrak^{n} \cong U^{(n)}$.
\end{proposition}

The proof essentially boils down to understanding the $p$-adic valuation of a natural number of the form $n!$, unsurprisingly.

\begin{lemma}
  Let $n$ be a natural number and write
  \[ n = \sum_{i = 0}^{r} a_{i} p^{i}, \ \ 0 \le a_{i} < p \]
  for its $p$-adic expansion.
  Then
  \[ v_{p}(n!) = \frac{1}{p-1} \sum_{i = 0}^{r} a_{i}(p^{i}-1). \]
\end{lemma}
\begin{proof}
  See Neukirch Ch.~II Lemma 5.6.
\end{proof}

\begin{proof}[Proof of proposition]
  Consider the valuation $v$ as extending the $p$-adic valuation.
  In particular, if $\pi$ is a uniformizer then $v(\pi) = 1/e$ (not $1$).
  Let's estimate $v_{p}(a)$ for a natural number $a > 1$.
  Write $a = p^{b}a_{0}$ with $p$ not dividing $a_{0}$ and $b$ positive (otherwise the valuation is zero).
  Then
  \[ \frac{v_{p}(a)}{a-1} = \frac{b}{p^{b}a_{0}-1} \le \frac{b}{p^{b}-1} \le \frac{1}{p-1}. \]
  For $v(z) > 1/(p-1)$ (this is the inequality in the statement for the normalized valuation), we have
  \[ v(z^{a}/a) - v(z) = (a-1)v(z) - v(a) > (a-1) (1/(p-1)-v(a)/(a-1)) \geq 0. \]
  Thus $v(\log(1+z)) = v(z)$.
  Thus $\log$ maps $U^{(n)}$ into $\mfrak^{n}$.

  Now consider the exponential series $\sum_{a} x^{a}/a!$, and compute the valuations of the terms.
  If $a = \sum_{i} a_{i} p^{i}$ is the $p$-adic expansion, then the lemma above shows
  \[ v(a!) = (1/(p-1)) \sum_{i} a_{i}(p^{i}-1) = (1/(p-1)) (a-(a_{0} + a_{1} + \cdots + a_{r})). \]
  Let $s_{a} = a_{0} + \cdots + a_{r}$ so that

  \[ v(x^{a}/a!) = a v(x) - (a-s_{a})/(p-1) = a(v(x)-(1/(p-1))) + s_{a}/(p-1). \]
  For $x$ satisfying the bounds in question, this implies convergence for the exponential series as the terms above would go to $\infty$.

  Furthermore, if $x$ is nonzero and $a > 1$, then
  \[ v(x^{a}/a!) - v(x) = (a-1)v(x) - (a-1)/(p-1) + (s_{a}-1)/(p-1) \geq (s_{a}-1)/(p-1) \geq 0. \]
  It follows that $\exp$ indeed maps $\mfrak^{n}$ into $U^{(n)}$.
  The formal properties of $\exp$ and $\log$ show that they are inverses to eachother, and provide isomorphisms as stated.
\end{proof}

Now let's go back to the task at hand regarding the multiplicative structure of $K$.
For $n$ sufficiently large, we have $U^{(n)} \cong \mfrak^{n} \cong \pi^{n} \cdot \Ocal \cong \Ocal$.
As $\Zbb_{p}$ is a DVR and $\Ocal$ is its integral closure in $K$, it follows that $\Ocal \cong \Zbb_{p}$.
Also, the index $[U^{(1)}:U^{(n)}]$ is finite of the form $q^{n}$.

The torsion subgroup of $U^{(1)}$ is the group $\mu_{p^{\infty}}(K)$.
Indeed, $U^{(1)}$ is a $\Zbb_{p}$-module with respect to the operation $(1+x) \mapsto (1+x)^{z}$ for $z \in \Zbb_{p}$ and the above considerations show that it is finitely-generated as such.
Its torsion is thus a finite $p$-subgroup of $K^{\times}$ and since the other torsion parts of $K^{\times}$ have order prime to $p$, the claim follows.
It follows therefore that, indeed $U^{(1)} = \mu_{p^{\infty}}(K) \times \Zbb_{p}^{d}$.

\begin{example}
  The square classes of $\Qbb_{2}$ are generated by $2$, $-1$ and $5$.
  The square classes of $\Qbb_{p}$ for odd $p$ are...
\end{example}

%%% Local Variables:
%%% mode: latex
%%% TeX-master: "main"
%%% End:
