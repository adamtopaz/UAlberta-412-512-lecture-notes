\section{Lecture 3}

\begin{definition}
  A \emph{number field} is a finite field extension of $\Qbb$, and a \emph{global function field} is a finite extension of $\Fbb_{p}(t)$ for some prime $p$.
  A \emph{global field} is either a number field or a global function field.
\end{definition}

\begin{definition}
  Let $K$ be a number field.
  The \emph{ring of integers of $K$}, denoted $\Ocal_{K}$, is the integral closure of $\Zbb$ in $\Ocal_{K}$.
\end{definition}

Any element of $\Ocal_{K}$ can be written as a product of irreducible elements.
To see this, we will need to use the norm, which we now discuss in the appropriate generality for this course.
The general context involves a finite field extension $L|K$, and a normal domain $A$ whose fraction field is $K$.
We write $B$ for the integral closure of $A$ in $L$.

Let $x \in L$ be given, and write $M_{x}$ for the $K$-linear map $L \to L$ given by $t \mapsto x \cdot t$.
\begin{definition}
  The trace of $x$, denoted $\Trace_{L|K}(x)$ is the trace of $M_{x}$, and the norm of $x$, denoted $\Norm_{L|K}(x)$ is the determinant of $M_{x}$.
  These are both maps $L \to K$.
\end{definition}

\begin{lemma}
  The map $\Trace_{L|K}$ is additive and the map $\Norm_{L|K}$ is multiplicative.
\end{lemma}
\begin{proof}
  \todo{Clear!}{Add proof.}
\end{proof}

The trace and norm can be described using Galois theory.
Fix an algebraic closure $\bar K$ of $K$.
Recall that separable degree $[L:K]_{s}$ is the degree of the maximal separable subextension of $L|K$, say $L_{s}|K$, and the inseparable degree $[L:K]_{i}$ is the degree of $L|L_{s}$.
Thus $[L:K] = [L:K]_{i} \cdot [L:K]_{s}$.
The separable degree $[L:K]_{s}$ is the number of $K$-embeddings of $L$ (or of $L_{s}$) into $\bar K$.
If $L|K$ is separable, then $L_{s} = L$, $[L:K]_{i} = 1$ hence $[L:K] = [L:K]_{s}$ agrees with the number of $K$-embeddings of $L$ into $\bar K$.

\begin{proposition}\label{proposition:norm_trace_formula_embeddings}
  Suppose that $L|K$ is separable.
  Then the characteristic polynomial of $M_{x}$ is the product $\prod_{\sigma} (X - \sigma x)$, as $\sigma \in \Hom_{K}(L,\bar K)$ varies.
  Thus $\Trace_{L|K}(x) = \sum_{\sigma} \sigma x$ and $\Norm_{L|K}(x) = \prod_{\sigma} \sigma x$, as $\sigma \in \Hom_{K}(L,\bar K)$ varies.
\end{proposition}
\begin{proof}
  \todo{Split up $L|K$ into $L|K(x)|K$, etc.}{Add the proof discussed in class.}
\end{proof}

\begin{lemma}
  The norm and trace are transitive in towers.
  Namely, suppose that $M|L|K$ is a tower of finite extensions.
  Then $\Norm_{M|K} = \Norm_{L|K} \circ \Norm_{M|L}$ and $\Trace_{M|K} = \Trace_{L|K} \circ \Trace_{M|L}$.
\end{lemma}
\begin{proof}
  \todo{Reduce to the separable case, which follows from Proposition~\ref{proposition:norm_trace_formula_embeddings}.}{Add details.}
\end{proof}

\begin{lemma}
  Suppose that $L|K$ is separable and that $b \in B$ is given.
  Then $\Trace_{L|K}(b) \in A$ and $\Norm_{L|K}(b) \in A$.
  Furthermore, $\Norm_{L|K}(b) \in A^{\times}$ if and only if $b \in B^{\times}$.
\end{lemma}
\begin{proof}
  \todo{Use the formula for the trace and norm involving embeddings (and the multiplicativity of the norm for one direction of the second part).}{Add details.}
\end{proof}

We can now prove the claim about irreducible factorization mentioned above.
\begin{lemma}
  Let $K$ be a number field and let $a \in \Ocal_{K}$ be given.
  Then $a$ can be factored as a product of irreducibles.
\end{lemma}
\begin{proof}
  If $a$ is irreducible or a unit, then we are done.
  Otherwise, we can write $a = x \cdot y$ with $x,y$ nonunits.
  Thus both $x$ and $y$ have norm (in $\Qbb$) which are nonunits of $\Zbb$, hence
  \[ 1 < |\Norm_{K|\Qbb}(t)| < |\Norm_{K|\Qbb}(a)| \]
  for both $t = x,y$.
  Proceed by induction.
\end{proof}

However, the \emph{uniqueness of such factorizations} fails in general, so $\Ocal_{K}$ is not, in general, a UFD.
\begin{example}
  Let $K$ be the quadratic number field $\Qbb(\sqrt{-5})$.
  In this case, we have $\Ocal_{K} = \Zbb[\sqrt{-5}]$.
  Express $21$ as $3 \cdot 7$ and as $(1 + 2 \cdot \sqrt{-5}) \cdot (1 - 2 \cdot \sqrt{-5})$.
  Note that both are factorizations in $\Ocal_{K}$ into products of irreducibles, which are not equivalent (in the sense of factorizations).
\end{example}

This ``issue'' of nonuniqueness of factorizations will be soon resolved by replacing elements of $\Ocal_{K}$ by ideals of $\Ocal_{K}$.

%%% Local Variables:
%%% mode: latex
%%% TeX-master: "main"
%%% End:
