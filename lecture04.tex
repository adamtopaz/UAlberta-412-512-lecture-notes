\section{Lecture 4}

The ``issue'' of nonuniqueness of factorizations, discussed in the last lecture, will be soon resolved by replacing elements of $\Ocal_{K}$ by ideals of $\Ocal_{K}$.
We will first try to pinpoint exactly the ring-theoretic properties of rings such as $\Ocal_{K}$ that makes the theory of ideals well-behaved in this respect.
The main definition here is the following.
\begin{definition}
  A Noetherian, normal domain in which every nonzero prime ideal is maximal is called a \emph{Dedekind domain}.
\end{definition}
Our goal for this lecture is to show that $\Ocal_{K}$, where $K$ is a number field, is a Dedekind domain.

We continue with the context from the last lecture: $L|K$ is a finite extension, $A$ is a normal domain with fraction field $K$ and $B$ is the normalization of $A$ in $L$.

Recall that a ring $A$ is called \emph{Noetherian} if it satisfies one of the following (equivalent) conditions:
\begin{enumerate}
  \item The ascending chain condition is satisfied for ideals.
  \item Every ideal is finitely-generated.
  \item Every nonempty set of ideals has a maximal element.
  \item Every submodule of a finitely-generated $A$-module is finitely-generated.
  \item Submodules of every finitely-generated $A$-module satisfy the ascending chain condition.
\end{enumerate}

We will need to revisit the trace.
Let $x_{1},\ldots,x_{n}$ be a basis for $L|K$ and assume that $L|K$ is separable.
Define the \emph{discriminant} of $(x_{1},\ldots,x_{n})$ as follows:
\[ d(x_{1},\ldots,x_{n}) = \det((\sigma_{i}x_{j})_{i,j})^{2}, \]
where $\sigma_{i}$ varies over the $K$-embeddings of $L$ into $\bar K$.
Using the formula for the trace in terms of the $\sigma_{i}$, we have
\[ d(x_{1},\ldots,x_{n}) = \det(\Trace_{L|K}(x_{i}x_{j})_{i,j}). \]

In the special case where the basis is $1,\theta,\ldots,\theta^{n-1}$ with $\theta$ a primitive element of $L|K$, then one has
\[ d(1,\theta,\ldots,\theta^{n-1}) = \prod_{i<j} (\sigma_{i} \theta - \sigma_{j}\theta)^{2}. \]
To see this, use \todo{Vandermonde matrices.}{Add details.}

\begin{lemma}
  If $L|K$ is separable, then the discriminant $d(x_{1},\ldots,x_{n})$ is nonzero, and the map
  \[ (x,y) \mapsto \Trace_{L|K}(x \cdot y) \]
  is a nondegenerate $K$-bilinear form on $L$.
\end{lemma}
\begin{proof}
  Choose a primitive element $\theta$ and use the description of the discriminant in terms of conjugates to see that the determinant is nonzero.
\end{proof}

\begin{lemma}
  Suppose $L|K$ is separable and that $x_{1},\ldots,x_{n}$ is a basis of $L|K$ which is contained in $B$.
  Let $d = d(x_{1},\ldots,x_{n})$.
  Then $d \cdot B \subset A x_{1} + \cdots + A x_{n}$.
\end{lemma}
\begin{proof}
  Let $b \in B$ be given and write $b = a_{1} x_{1} + \cdots + a_{n} x_{n}$, $a_{j} \in K$.
  The $a_{j}$ are solutions of the system
  \[ \Trace_{L|K}(x_{i} b) = \sum_{j} \Trace_{L|K}(x_{i}x_{j}) a_{j}. \]
  Since $\Trace_{L|K}(x_{i} b) \in A$, the $a_{j}$ are all elements of $(1/d) \cdot A$.
  The claim follows.
\end{proof}

\begin{proposition}
  Suppose that $A$ is a PID and $M$ is a nonzero $B$-submodule of $L$.
  Then $M$ is free over $A$ of rank $[L:K]$.
\end{proposition}
\begin{proof}
  Let $x_{1},\ldots,x_{n}$ be a basis of $L|K$ contained in $B$.
  Let $m_{1},\ldots,m_{k} \in M$ be generators over $B$.
  We can find $a \in A$ such that $a \cdot m_{i} \in B$, hence
  \[ a \cdot d \cdot M \subset d \cdot B \subset A x_{1} + \cdots + A x_{n} =: N. \]
  Since generators of $B$ over $A$ also generate $L$ over $K$, we see that $B$ has rank $[L:K]$, and the first inclusion also shows that $M$ has rank $[L:K]$.
  The structure theorem for modules over PIDs gives the freeness assertion.
\end{proof}

This shows that $B$ is Noetherian if $A$ is a PID.
In fact, a \todo{similar assertion holds only under the assumption that $A$ is Noetherian.}{Add proof from Lang.} But we will not need it.
Now let's turn to studying prime ideals.
It will be useful to use \emph{localizations} in this context, which we now review.

Let $A$ be a domain, and $S$ a multiplicative submonoid of $A$ which does not contain $0$.
The \emph{localization} $S^{-1}A$ is the subset of $K$ of elements of the form $a/s$ for $a \in A$ and $s \in S$.
If $M$ is a submodule contained in some field extension $L|K$, we define $S^{-1}M$ in a similar way as elements of the form $m/s$, $m \in M$ and $s \in S$.
If $\pfrak$ is a prime ideal of $A$ and $S = A \smin \pfrak$, then $A_{\pfrak} := S^{-1}A$.

We say that $A$ is \emph{local} provided that it has a unique maximal ideal.
The ring $A_{\pfrak}$ is such a ring.
If $\Afrak$ is an ideal of $S^{-1}A$, \todo{then $\Afrak = S^{-1}(\Afrak \cap A)$.}{Prove this.}

\begin{proposition}
  The localizatsion $S^{-1}A$ is normal.
  The algebra $S^{-1}B$ is integral closure of $S^{-1}A$.
\end{proposition}
\begin{proof}
  If $x \in K$ is integral over $S^{-1}A$, say
  \[ x^{n} + \frac{a_{1}}{s_{1}} x^{n-1} + \cdots + \frac{a_{n}}{s_{n}} = 0, \]
  then we can multiply by elements of $S$ to obtain an integral equation of $s \cdot x$ for some $s \in S$, with coefficients in $A$.
  This shows that $s \cdot x \in A$ hence $x \in S^{-1} A$.
  \todo{The rest is similar and left to the reader.}{Add proof.}
\end{proof}

Suppose that $\Pfrak$ is a prime ideal of $B$ annd $\pfrak$ is a prime of $A$.
Note that $\Pfrak \cap A$ is a prime ideal in $A$.
We say that $\Pfrak$ \emph{lies over} $\pfrak$, and write $\Pfrak | \pfrak$, provided that $\Pfrak \cap A = \pfrak$.
In this case, we have an inclusion
\[ A/\pfrak \to B/\Pfrak. \]

\begin{proposition}\label{proposition:exists_prime_above}
  Let $\pfrak$ be a prime of $A$.
  Then $\pfrak \cdot B \neq B$ and there exists a prime ideal $\Pfrak$ of $B$ lying above $\pfrak$.
\end{proposition}
\begin{proof}
  Recall that $B_{\pfrak}$ is integral over $A_{\pfrak}$ and that $A_{\pfrak}$ is local with maximal ideal $\pfrak_{\pfrak}$.
  This reduces the question to the case where $A$ is local and $\pfrak$ is the unique maximal ideal.

  If $\pfrak B = B$, then $1$ can be expressed as a finite linear combination of the form
  \[ 1 = a_{1} b_{1} + \cdots + a_{n} b_{n}, \]
  with $a_{i} \in \pfrak$ and $b_{i} \in B$.
  Write $B_{0} = A[b_{1},\ldots,b_{n}]$ hence $\pfrak B_{0} = B_{0}$ and $B_{0}$ is finite over $A$.
  By \todo{\emph{Nakayama's Lemma}}{Add statement and proof.} we have $B_{0} = 0$, which is impossible.

  Now, choose some maximal ideal $\Mfrak$ of $B_{\pfrak}$ containing $\pfrak \cdot B_{\pfrak}$.
  Since $\pfrak$ is maximal, we have $\pfrak = \Mfrak \cap A$.
  Take $\Pfrak = \Mfrak \cap B$ to find $\Pfrak$ such that $\Pfrak \cap A = \pfrak$.
\end{proof}

%%% Local Variables:
%%% mode: latex
%%% TeX-master: "main"
%%% End:
