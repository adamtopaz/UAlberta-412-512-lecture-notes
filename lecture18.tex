\section{Lecture 18}

Recall from the last lecture that any rank one valuation $v$ on a field $K$ induces an equivalence class of absolute values on $K$.

\begin{theorem}
  Suppose that $|-|$ is a nonarchmedean absolute value on $K$.
  Then $|-|$ arises from a rank $1$ valuation on $K$.
\end{theorem}
\begin{proof}
  We must check that the ultrametric inequality
  \[ |x+y| \le max(|x|,|y|) \]
  holds true.
  To see this, take powers $|x+y|^{n}$, and use the fact that $|n| \le 1$ for any $n$ of the form $1 + \cdots + 1$.
\end{proof}

Suppose that $(K,|-|)$ is a valued field.
We can take the completion of $K$, defined in terms of Cauchy sequences modulo nullsequences.
The result $\hat K$ is again a field, and the absolute value $|-|$ extends to $\hat K$ by taking limits.

\begin{theorem}[Ostrowski]
  If $K$ is complete with respect to an archemedean absolute value, then $K \cong \Cbb$ or $\Rbb$ (compatibly with the usual absolute value, up-to equivalence).
\end{theorem}
\begin{proof}
  See Theorem 4.2 in Ch.~II of Neukirch.
\end{proof}

Let us now focus on the nonarchemean case.
As we saw above, the absolute value on $K$ arises from a valuation $v$.
We can thus extend $v$ to $\hat v$ on $\hat K$ by taking limits.
Suppose that $a = \lim_{n} a_{n}$ is such a limit in $\hat K$, with $a_{n} \in K$.
Let's note that the sequence $v(a_{n})$ is eventually constant!
\begin{exercise}
  This holds for arbitrary valuations $v$ on a field $K$ (not just rank $1$):
  If $v(a) \neq v(b)$ then $v(a+b) = \min(v(a),v(b))$.
\end{exercise}

Now note that for some sufficiently large $n$, we have $\hat v(a-a_{n}) > \hat v(a)$ and thus
\[ v(a_{n}) = \hat v(a_{n}-a+a) = \hat v(a). \]
From this it also follows that the value groups of $v$ and of $\hat v$ are the same!
For example, if $v$ is a rank $1$ discrete valuation (so that $\Gamma_{v} = \Zbb$) then the same is true for $\hat v$.

\begin{exercise}
  Suppose that $K$ is a field with a nonarchemean absolute value.
  A sequence $a_{n}$ is Cauchy if and only if $a_{n+1}-a_{n}$ is a nullsequence.
  A series
  \[ \sum_{n \geq 0} a_{n} \]
  converges if and only if $a_{n}$ is a nullsequence.
\end{exercise}

\begin{theorem}
  Let $\Ocal$ be the valuation ring of $K$ and $\hat \Ocal$ the valuation ring of $\hat K$.
  The valuation ring $\hat\Ocal$ is the completion of $\Ocal$.
  Then $\Ocal$ annd $\hat\Ocal$ have (canonically) isomorphic residue fields (cf.~HW3).
  If furthermore $v$ is discrete, and $\mfrak$ resp.~$\hat\mfrak$ denote the maximal ideal of $\Ocal$ resp.~$\hat\Ocal$, then
  \[ \Ocal/\mfrak^{n} \cong \hat\Ocal/\hat\mfrak^{n} \]
  for any $n$.
  Similarly,
  \[ \Ocal^{\times}/U^{n} \cong \hat\Ocal^{\times}/\hat U^{n} \]
\end{theorem}
\begin{proof}
  If $a = \lim_{n} a_{n}$ is some element of $\hat\Ocal$, then $\lim_{n} v(a_{n}) \geq 0$.
  Replacing $a_{n}$ with a subsequence, we may assume that all the $a_{n} \in \Ocal$, and the first assertion follows from this.

  Now if $a$ has positive valuation, then its image in the residue field is trivial, so there is nothing to show.
  Otherwise, we may as well assume that all the $a_{n}$ have valuation $0$, and are thus units.
  But the sequence is also Cauchy, and thus eventually we must have that $a_{n+1}-a_{n}$ is contained in the maximal ideal.
  Assume WLOG that this is always the case, and from this it follows that $a_{0}$ represents the element $a$ in $\hat\Ocal/\hat\mfrak$.

  The assertion for powers of $\mfrak$ in the discrete case is proved similarly.
\end{proof}

\begin{theorem}
  Suppose that $K$ is a complete discretely valued field with ring of integers $\Ocal$ and maximal ideal $\mfrak$.
  Then
  \[ \Ocal \cong \varprojlim_{n} \Ocal/\mfrak^{n} \]
  and
  \[ \Ocal^{\times} \cong \varprojlim_{n} \Ocal^{\times}/U^{(n)}. \]
\end{theorem}
\begin{proof}
  Argue similarly to the theorem above.
\end{proof}

%%% Local Variables:
%%% mode: latex
%%% TeX-master: "main"
%%% End:
