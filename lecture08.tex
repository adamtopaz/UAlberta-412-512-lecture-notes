\section{Lecture 8}

Our primary goal for this lecture is to discuss the proof of the following theorem, and its applciations to finiteness of the class group.
Throughout today's lecture, $K$ will denote a number field.
Let $\tau_{1},\ldots,\tau_{n}$ denote the collection of complex embeddings of $K$.
As before, $\rho_{1},\ldots,\rho_{r}$ will be the real embeddings and $\sigma_{1},\bar\sigma_{1},\ldots,\sigma_{s},\bar\sigma_{s}$ the (conjugate pairs) of complex embeddings, so that $n = r + 2s$.

\begin{definition}
  Let $\afrak$ be an (integral) ideal of $\Ocal_{K}$.
  Then the \emph{absolute norm} of $\afrak$, denoted $\Norm(\afrak)$, is the index (as abelian groups) $[\Ocal_{K}:\afrak]$.
\end{definition}

\begin{theorem}\label{theorem:exists_norm_mult}
  Let $\afrak$ be a nonzero ideal of $\Ocal_{K}$.
  For each complex embedding $\tau$, let $c_{\tau}$ be a positive real, and assume that $c_{\tau} = c_{\bar\tau}$ for all $\tau$.
  Assume also that
  \[ \prod_{\tau} c_{\tau} > \left(\frac{2}{\pi}\right)^{s} \cdot \sqrt{|d_{K}|} \cdot \Norm(\afrak). \]
  Then there exists a nonzero element $a \in \afrak$ such that $|\tau a|<c_{\tau}$ for all complex embeddings $\tau$.
\end{theorem}
\begin{proof}
  Consider the set
  \[ S := \{(z_{\tau}) \in K_{\Rbb} \ | \ |z_{\tau}| < c_{\tau} \}. \]
  Here the $z_{\tau}$ are the coordinates of $K_{\Rbb}$ via the embedding
  \[ K_{\Rbb} \subset K_{\Cbb} = \prod_{\tau} \Cbb. \]
  This set is symmetric about the origin, meaning that $-x \in S$ for all $x \in S$, and \emph{convex}.

  Let's compute its volume.
  Identify
  \[ K_{\Rbb} \cong \prod_{\tau} \Rbb \]
  as in the last lecture, \todo{so that $z_{\rho}$ maps to itself while $z_{\sigma}$ maps to its real and imaginary parts.}{Add details.}

  We can compute the usual volume in $\prod_{\tau}\Rbb$ fairly easily.
  Namely, its image in $\prod_{\tau} \Rbb$ is the set whose real coordinates satisfy $|x_{\rho}| < c_{\rho}$ and whose complex coordinates satisfy $x_{\sigma}^{2} + x_{\bar\sigma}^{2} < c_{\sigma}^{2}$.
  The usual volume is thus
  \[ \prod_{i = 1}^{r} (2 c_{\rho_{i}}) \cdot \prod_{i = 1}^{s} (\pi \cdot c_{\sigma}^{2}) = 2^{r} \pi^{s} \prod_{\tau} c_{\tau}. \]
  Taking into account the factor of $2^{s}$ discussed in the last lecture, the volume of $S$ in $K_{\Rbb}$ is precisely
  \[ 2^{r+s} \pi^{s} \prod_{\tau} c_{\tau}. \]
  By assumption along with Proposition~\ref{proposition:covolume_ideal_computation}, we have
  \[ vol(S) = 2^{r+s} \pi^{s} \prod_{\tau} c_{\tau} > 2^{r+s} \pi^{s} \left( \frac{2}{\pi} \right)^{s} \sqrt{|d_{K}|} \cdot N(\afrak) = 2^{n} covol(\afrak), \]
  where the covolume of $\afrak$ is computed via the embedding $K \to K_{\Rbb}$ (the image of $\afrak$ with respect to this map is a complete lattice).
  Minkowski's lattice point theorem (see below) gives the result.
\end{proof}

\begin{theorem}[Minkowski's lattice point theorem]
  Let $\Lambda$ be complete lattice in a finite-dimensional real inner product space $V$.
  Let $S$ be a subset of $V$ which is symmetric about the origin and which is convex.
  Assume that $vol(S) > 2^{n} covol(\Lambda)$.
  Then $S$ contains a nonzero element of $\Lambda$.
\end{theorem}
\begin{proof}
  Consider sets of the form $S + \lambda$ for $\lambda \in \Lambda$.
  Assume that these sets are pairwise disjoint, and let $M$ be a fundamental mesh of $\Lambda$.
  Then the intersections $M \cap (S + \lambda)$ are \todo{also pairwise disjoint}{why?} and thus the volume of $M$ (which is the covolume of $\Lambda$) would satisfy
  \[ covol(\Lambda) \geq \sum_{\lambda} vol(M \cap (S + \lambda)). \]
  But if we translate $M \cap (S + \lambda)$ by $-\lambda$, we obtain $(M - \lambda) \cap S$, which has equal volume, while $M - \lambda$ cover all of $V$ and thus $(M - \lambda) \cap S$ cover all of $S$.
  Thus, we have
  \[ covol(\Lambda) \geq \sum_{\gamma} vol((M - \lambda) \cap S) = vol(S). \]

  Now if we replace $S$ with $(1/2) \cdot S$, the same argument goes through, except that $vol(S)$ gets scaled by $1/2^{n}$, where $n$ is the dimension of $V$, hence
  \[ covol(\Lambda) \geq (1/2^{n}) \cdot vol(S). \]
  But this contradicts the assumption of the theorem.

  It follows that there exist distinct $\lambda_{1},\lambda_{2} \in \Lambda$ such that $(1/2) \cdot S + \lambda_{1}$ and $(1/2) \cdot S) + \lambda_{2}$ contain a common element, say $t$.
  Writing
  \[ t = (1/2) s_{1} + \lambda_{1} = (1/2) s_{2} + \lambda_{2} \]
  and solving for $\lambda_{1} - \lambda_{2} \in \Lambda$ gives $(1/2) \cdot (s_{2} - s_{1})$ which is an element of $S$ since $S$ is symmetric about the origin and convex.
  This element is nonzero since $\lambda_{1} \neq \lambda_{2}$.
\end{proof}

\begin{theorem}[A variant]
  In the context of the previous theorem, assume furthermore that $S$ is compact.
\end{theorem}
\begin{proof}
  Replace $S$ with $(1 + \epsilon) \cdot S$ for $\epsilon > 0$ to see that this has nonzero points of $\Lambda$.
  By compactness, there can only be finitely many such points.
  Also, $S = \bigcap_{\epsilon} (1 + \epsilon) \cdot S$.
  If this intersection contains only the origin, then (using the discreteness of $\Lambda$) shows that the same must hold for a single $\epsilon$.
\end{proof}

Let's now discuss some variants and applications of the above results.
\begin{theorem}
  Let $\afrak$ be an ideal of $\Ocal_{K}$.
  Then there exists a nonzero element $a \in \afrak$ such that
  \[ |N_{K|\Qbb}(a)| \leq \frac{n!}{n^{n}}\left(\frac{4}{\pi}\right)^{s} \sqrt{|d_{K}|} \cdot N(\afrak). \]
\end{theorem}
\begin{proof}
  Discussed in the next lecture.
\end{proof}

Let us now discuss some general facts about norms of ideals.
Recall that we have
\[ \Norm(\afrak) := [\Ocal_{K}:\afrak]. \]
This is \emph{multiplicative}.
\begin{lemma}
  Let $\afrak$ and $\bfrak$ be two integral ideals.
  Then $\Norm(\afrak\cdot\bfrak) = \Norm(\afrak) \cdot \Norm(\bfrak)$.
  Furthermore, if $a \in \Ocal_{K}$ is a nonzero element, then $\Norm(a) = |\Norm_{K|\Qbb}(a)|$.
\end{lemma}
\begin{proof}
  Discussed in the next lecture.
\end{proof}

\begin{corollary}
  Every ideal class of $K$ has an integral representative $\afrak$ satisfying
  \[ \Norm(\afrak) \leq \frac{n!}{n^{n}}\left(\frac{4}{\pi}\right)^{2} \sqrt{|d_{K}|}. \]
\end{corollary}
\begin{proof}
  Let $\bfrak$ be a fractional ideal.
  Choose some $x \in K^{\times}$ such that $x \cdot \bfrak^{-1}$ is integral, say $\afrak = x \cdot \bfrak^{-1}$.
  There exists a nonzero element $a \in \afrak$ satisfying
  \[ |\Norm_{K|\Qbb}(a)| \leq \frac{n!}{n^{n}}\left(\frac{4}{\pi}\right)^{2} \sqrt{|d_{K}|} \cdot \Norm(\afrak). \]
  We have $(a) \subset \afrak$, so we may write $(a) = \afrak \cdot \afrak'$ with \todo{$\afrak'$ integral}{Why?}.
  Thus $\afrak$ and $\afrak'^{-1}$ are equal in the class group, while $\afrak^{-1}$ and $\bfrak$ are equal in the class group.

  On the other hand, we have
  \[ \Norm(\afrak) \cdot \Norm(\afrak') = |\Norm_{K|\Qbb}(a)| \leq \frac{n!}{n^{n}}\left(\frac{4}{\pi}\right)^{2} \sqrt{|d_{K}|} \cdot \Norm(\afrak). \]
  Thus $\afrak'$ is an integral representative of $\bfrak$ satisfying the assertion of the theorem.
\end{proof}

\begin{corollary}
  The class group of $K$ is finite.
\end{corollary}
\begin{proof}
  It suffices to show that there are only finitely many integral ideals whose absolute norm is less than the Minkowski bound.
  In fact, if $M$ is any fixed positive real number, there are only finite many integral ideals $\afrak$ whose absolute norm is bounded above by $M$, as can be easily seen from the multiplicativity of the absolute norm.
\end{proof}

%%% Local Variables:
%%% mode: latex
%%% TeX-master: "main"
%%% End:
