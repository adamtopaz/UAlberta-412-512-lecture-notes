\section{Lecture 24}

In this lecture we will discuss the main statements from global class field theory for number fields.
First, let's talk about the relationship between number fields and their completion from the point of view of ramification theory.

Let $K$ be a number field, and let $|-|$ be an absolute value on $K$.
Ostrowski's theorem tells us that $|-|$ is either Archemedean arising from some complex embedding, or it arises from some nonzero prime ideal of the ring of integers of $K$.
In any case, we know that for any finite extension $L$ of $K$, there are finitely many extensions of $|-|$ to $L$.

\begin{proposition}
  Let $|-|_{i}$, $i = 1,\ldots,g$ denote the extensions of $|-|$ to $L$.
  Let $\hat K$ denote the completion of $K$ and $\hat L_{i}$ the completion of $L$ with respect to $|-|_{i}$.
  Then $L \otimes_{K} \hat K = \prod_{i} L_{i}$.
\end{proposition}

Now let's focus on the nonarchemedean case.
Let $L|K$ be a finite extension of number fields, and $|-|$ compatible nonarchemedean absolute values on $L$ and $K$.
Since these arise from primes, say $\Pfrak|\pfrak$, we can talk about the decompositoin group $Z_{\Pfrak|\pfrak}$.
This is also the stabilizer of the action of $\Gal(L|K)$ on the absolute value chosen for $L$ (why?).
It follows that $Z_{\Pfrak|\pfrak}$ acts on $L$ continuously with respect to the chosen absolute value, and thus we have a natural map
\[ Z_{\Pfrak|\pfrak} \to \Gal(\hat L|\hat K). \]
\begin{proposition}
  This map is an isomorphism.
\end{proposition}
\begin{proof}[Proof sketch]
  This map is clearly injective since $L$ is dense in $\hat L$.
  For $\sigma \in \Gal(L|K)$, the absolute value $|\sigma(-)|$ has decomposition group $\sigma Z_{\Pfrak|\pfrak} \sigma^{-1}$, which has the same size as $Z_{\Pfrak|\pfrak}$, so that its size is bounded above by the degree of $\hat L_{\sigma}|\hat K$, for all such $\sigma$.
  thus we have
  \[ \# \Gal(L|K) = [G:Z] \cdot \# Z \leq \sum_{\sigma \in \Gal(L|K)/Z} [\hat L_{\sigma}:\hat K] \leq [L:K] \]
  So equality must hold everywhere, and the claim follows.
\end{proof}

\begin{proposition}
  This isomorphism is compatible with inertia groups.
\end{proposition}
\begin{proof}
  Exercise.
\end{proof}

The discussion above shows that the ramification behavior of a finite extension of number fields $L|K$ can be checked ``locally'' in the completion.
Explicitly, suppose that $\pfrak$ is a prime of $K$ with associated completion $K_{\pfrak}$.
Let $\Pfrak_{1},\ldots,\Pfrak_{g}$ be the prolongations to $L$, and write $L_{\Pfrak_{i}}$ for the associated completions.
For the extension of local fields $L_{\Pfrak_{i}}|K_{\pfrak}$, we have well-defined invariants $e$ and $f$, and similarly in the global case for the fixed $\Pfrak_{i}|\pfrak$.
The two notions coincide.

Even better, consider the map $K \to \hat K$, and choose an algebraic closure $\Omega$ of $\hat K$, and let $\bar K$ denote the algebraic closure of $K$ inside of $\Omega$.
The proposition above then shows that the map
\[ \Gal(\Omega|\hat K) \to \Gal(\bar K|K) \]
obtained by restriction, is \emph{injective}.
Also, recall that there is a unique extension of $|-|$ to $\Omega$, and thus $\bar K$ inherits an absolute value extending that of $K$ via the inclusion.
This corresponds to a prolongation $\bar \pfrak$ of $\pfrak$, and the iamge of the inclusion above is precisely $Z_{\bar\pfrak|\pfrak}$ (compatibly also with inertia subgroups).

Let me now introduce one of the main players in global class field theory, the \emph{ring of Adeles}:
\[ \Abb_{K} := \prod_{\pfrak}{}^{'} K_{\pfrak}. \]
Here $\prod'$ is a \emph{restricted product} (explained during lecture) and $K_{\pfrak}$ is the completion of $K$ with respect to the place $\pfrak$, which varies over all primes including the $\pfrak|\infty$.
This is a \emph{topological ring}.
We also consider the \emph{ideles} $\Ibb_{K} := \Abb_{K}^{\times}$, topologized by identifying $\Ibb_{K}$ with the set of elements $(u,v)$ in $\Abb_{K}^{2}$ where $u \cdot v = 1$.

As a group, $\Ibb_{K}$ has a similar description as a restricted product
\[ \prod_{\pfrak} {}^{'} K_{\pfrak}^{\times}. \]
There is a canonical map
\[ K^{\times} \to \Ibb_{K} \]
defined as the diagonal inclusion associated with $K^{\times} \to K_{\pfrak}^{\times}$.
The group $\Ibb_{K}$ is locally compact (but not compact) and $K^{\times}$ embeds discretely in $\Ibb_{K}$.
The cokernel of this map is called the \emph{idele class group}, $\Cfrak_{K} := \Ibb_{K}/K^{\times}$.

If $L|K$ is a finite extension, and $\Pfrak|\pfrak$ is an extension of primes in $L|K$, then the norm induces a morphism $L_{\Pfrak}^{\times} \to K_{\pfrak}^{\times}$.
Collecting these, we obtain a \emph{norm map}
\[ N_{L|K} : \Ibb_{L} \to \Ibb_{K}, \]
whose $\pfrak$-component is $\prod_{\Pfrak|\pfrak}N_{L_{\Pfrak}|K_{\pfrak}}((-)_{\Pfrak})$.
The norm is clearly compatible with the field norm, and thus induces a norm on idele class groups as well.

Now let me discuss the global reciprocity law.
Let $L|K$ be an abelian Galois extension.
For $\Pfrak|\pfrak$ in $L|K$, we have a natural isomorphism
\[ \Gal(L_{\Pfrak}|K_{\pfrak}) \cong Z_{\Pfrak|\pfrak} \subset \Gal(K|L). \]
The local reciprocity law gives us a map
\[ (-/L|K)_{\pfrak} : K_{\pfrak}^{\times} \to \Gal(L_{\Pfrak}|K_{\pfrak}) \hookrightarrow \Gal(K|L). \]
\begin{lemma}
  The maps above induce a map
  \[ \Ibb_{K} \to \Gal(L|K) \]
\end{lemma}
\begin{proof}
  Since almost all primes are unramified and almost all components of $\Ibb_{K}$ are units, the map
  \[ (x_{\pfrak})_{\pfrak} \mapsto \prod_{\pfrak}(x_{\pfrak}/L|K)_{\pfrak} \]
  is well-defined as the product is actually \emph{finite}.
\end{proof}

\begin{theorem}
  The map
  \[ \Ibb_{K} \to \Gal(L|K) \]
  descends to a surjective map
  \[ \Cfrak_{K} \to \Gal(L|K) \]
  whose kernel is $N_{L|K}(\Cfrak_{L})$.
  Furthermore, the subgroups of $\Cfrak_{K}$ of the form $N_{L|K}(\Cfrak_{L})$ are precisely the closed subgroups of finite index.
\end{theorem}

The \emph{product formula} for a number field states that for $a \in K$, one has
\[ \prod_{\pfrak}|x|_{\pfrak} = 1 \]
where $|x|_{\pfrak}$ is normalized as $\#\kappa(\pfrak)^{-v_{\pfrak}(x)}$.
This has an analogue for \emph{Hilbert Symbols}, which we did not discuss in detail (although it essentially comes from the cup-product in Galois cohomology, details explained in lecture).
\begin{theorem}
  For $a,b \in K^{\times}$ and an integer $n$ such that $\mu_{n} \subset K$, write $(a,b)_{\pfrak} \in \mu_{n}$ for the Hilbert symbol at $K_{\pfrak}$.
  Then one has
  \[ \prod_{\pfrak}(a,b)_{\pfrak} = 1. \]
\end{theorem}
\begin{proof}
  The formula for the Hibert symbol shows that
  \[ (a,K_{\pfrak}(\sqrt[n]{b})|K_{\pfrak}))_{\pfrak} \cdot \sqrt[n]{b} = (a,b)_{\pfrak} \sqrt[n]{b}. \]
  Now let's calculate:
  \[ (\prod_{\pfrak}(a,b)_{\pfrak}) \cdot \sqrt[n]{b} = (\prod_{\pfrak}(a,K_{\pfrak}(\sqrt[n]{b})|K_{\pfrak})) \cdot \sqrt[n]{b} = (a,K(\sqrt[n]{b})|K)\sqrt[n]{b} = \sqrt[n]{b}. \]
  The key identity here follows from the fact that $a$ is a \emph{global element} and that the reciprocity map from global class field theory factors through the idele \emph{class} group.
\end{proof}

Define $(a/\pfrak) := (\pi,a)_{\pfrak}$ where $a \in U_{\pfrak}$ and $\pi$ is a uniformizer at $\pfrak$, whenever $\pfrak$ doesn't divide $n$.
It follows from local class field theory that $(a/\pfrak) = 1$ if and only if $a \equiv b^{n} \bmod p$ for some $b$.
More generally, $(a/\pfrak) = a^{(q-1)/n} \bmod \pfrak$.

For an ideal $\bfrak$ of $K$ coprime to $n$ whose $\pfrak$-component has exponent $v_{\pfrak}$, write $(a/\bfrak) = \prod_{\pfrak} (a/\pfrak)^{v_{\pfrak}}$ where the product runs over the primes coprime to $n$.

\begin{theorem}
  For $a,b \in K^{\times}$ coprime to eachother and to $n$, one has
  \[ (a/b)(b/a)^{-1} = \prod_{\pfrak|n\infty}(a,b)_{\pfrak}. \]
\end{theorem}
\begin{proof}
  A calculation using the above theorem.
\end{proof}
Gauss' quadratic reciprocity law follows from this by taking $n = 2$ and $K = \Qbb$.

%%% Local Variables:
%%% mode: latex
%%% TeX-master: "main"
%%% End:
