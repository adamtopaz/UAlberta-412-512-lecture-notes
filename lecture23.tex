\section{Lecture 23}

Last time we discussed the main statements of local class field theory.
One of these statements is the existence of the \emph{local reciprocity map}
\[ K^{\times} \to \Gal_{K}^{\ab} \]
which induces an isomorphism after profinite completion on the domain.
We also saw how $K^{\times} = \Zbb \times U_{K}$, where $U_{K}$ corresponds to the inertia group with respect to the reciprocity map.
In particular, if $\pi$ is a uniformizer, and $K^{\ab}$ denotes the maximal abelian extension of $K$, then we have two subfields of $K^{\ab}$, defined by taking invariants with respect to the action of $\pi$, or with respect to the action of $U_{K}$, via the reciprocity map.

The second, $K^{U_{K}}$ is the maximal unramified extension of $K$ (which is abelian, since the absolute Galois group of the residue field is abelian).
It is denoted $K^{ur}$.
The second, must be \emph{totally ramified} since it is disjoint from $K^{ur}$ (in the field-theoretic sense).
This extension depends on $\pi$, and we denote it by $K_{\pi}$.
By Galois theory, we have
\[ K^{\ab} = K^{ur} \cdot K_{\pi}. \]

Let's first discuss the structure of $K^{ur}$.
Any unramified extension of $K$ arises from an extension of the residue field.
We can thus obtain such extensions by adjoining roots of unity of the appropriate (prime-to-$p$) order.
See the end of the last lecture for details.

Understanding $K_{\pi}$ is much more complicated (and interesting!).
To do this, we need to discuss Lubin-Tate formal groups.
A \emph{formal group law} (the one-dimensional, commutative case) over a ring $A$ is a power series $F(X,Y) \in A[[X,Y]]$ in two variables, which satisfies the following conditions:
\begin{enumerate}
  \item $F$ has the form $F(X,Y) = X + Y + \ldots$ where the $\ldots$ involve only higher order terms.
  \item $F(F(X,Y),Z) = F(X,F(Y,Z))$.
  \item $F(X,Y) = F(Y,X)$.
\end{enumerate}
Formal group laws form a category where morphisms $F \to G$ are power series $f \in TA[[T]]$ such that $f(F(X,Y)) = G(f(X),f(Y))$.
Composition of morphisms is given by composition of \emph{power series} with no constant term (which makes sense).

Now let's focus on $K$ with uniformizer $\pi$ and ring of integers $\Ocal_{K}$.
Let $\Fcal_{\pi}$ denote the set of all elements $f$ of $\Ocal_{K}[[T]]$ such that
\begin{enumerate}
  \item $f = \pi X + \ldots$ where $\ldots$ only involves higher order terms.
  \item $f \equiv X^{q} \bmod \pi$.
\end{enumerate}

\begin{theorem}
  Let $f \in \Fcal_{\pi}$ be given.
  There exists a unique formal group law $F_{f}$ with coefficients in $\Ocal_{K}$ for which $f$ is an endomorphism.
\end{theorem}
\begin{proof}[Proof Idea]
  Construct $F_{f} \bmod (X,Y)^{n}$ by induction on $n$.
\end{proof}

\begin{proposition}
  For $f,g \in \Fcal_{\pi}$ and $a \in \Ocal_{K}$, there is a unique element of $\Ocal_{K}[[T]]$, denoted $[a]_{g,f}$, such that
  \begin{enumerate}
    \item $[a]_{g,f}(T) = aT + \ldots$.
    \item $g \circ [a]_{g,f} = [a]_{g,f} \circ f$.
  \end{enumerate}
  This $[a]_{g,f}$ is a morphism $F_{f} \to F_{g}$.
  Also, one has $[a+b]_{g,f} = [a]_{g,f} + [b]_{g,f}$ and $[ab]_{h,f} = [a]_{h,g}[b]_{g,f}$.
  In particular, $[1]_{g,f}$ is the unique isomorphism $F_{f} \cong F_{g}$ where $[1]_{g,f} = X + \ldots$, while $[u]_{g,f}$ is another isomorphism for any $u \in U_{K}$.
\end{proposition}

Now let's discuss how to construct $K_{\pi}$.
First, choose some $f \in \Fcal_{\pi}$, for example $f = \pi X + X^{q}$, and put $F := F_{f}$.
Let $\bar\mfrak$ denote the maximal ideal of the integral closure of $\Ocal_{K}$ in $\bar K$.
Endow $\bar\mfrak$ with the structure of an abelian group, written $+_{F}$, where
\[ x +_{F} y = F(x,y). \]
This is well-defined since $x,y$ are in the open unit ball, so that $F(x,y)$ converges, while the axioms of a formal group show that this is indeed an abelian group structure.

We have a natural action of $\Ocal_{K}$ on $(\bar\mfrak,+_{F})$, defined by
\[ \Ocal_{K} \to End(F), \ \ a \mapsto [a]_{f,f} =: [a]_{f}. \]
In particular, we may consider the \emph{$\pi$-torsion} of $\bar\mfrak$ with respect to this action.
Note that we have $[\pi]_{f} = f$, and that $\pi \cdot a = f(a)$ for $a \in \bar \mfrak$.
The $\pi^{n}$-torsion of this action is thus just the roots of $f^{(n)} = f \circ \cdots \circ f$, whose absolute value is $< 1$.
Write $\Lambda_{n}$ for this set, and put $K_{\pi,n} := K(\Lambda_{n})$.

\begin{theorem}
  As an $\Ocal_{K}$-module, $\Lambda_{n}$ is isomorphic to $\Ocal_{K}/\mfrak^{n}$.
  Thus one has $End_{\Ocal_{K}}(\Lambda_{n}) = \Ocal/\mfrak^{n}$ and $Aut_{\Ocal_{K}}(\Lambda_{n}) = U_{K}/U_{K}^{(n)}$.
  This action of $U_{K}/U_{K}^{(n)}$ on $\Lambda_{n}$ induces an isomorphism
  \[ U_{K}/U_{K}^{(n)} \cong \Gal(K(\Lambda_{n})|K). \]
  Also, one has $K_{\pi} = \bigcup_{n} K_{\pi,n}$.
\end{theorem}

I would now like to explain the relationship between global and local fields, particularly with respect to ramification behavior.
Let $K$ be a number field, and let $|-|$ be an absolute value on $K$.
Ostrowski's theorem tells us that $|-|$ is either Archemedean arising from some complex embedding, or it arises from some nonzero prime ideal of the ring of integers of $K$.
In any case, we know that for any finite extension $L$ of $K$, there are finitely many extensions of $|-|$ to $L$.

\begin{proposition}
  Let $|-|_{i}$, $i = 1,\ldots,g$ denote the extensions of $|-|$ to $L$.
  Let $\hat K$ denote the completion of $K$ and $\hat L_{i}$ the completion of $L$ with respect to $|-|_{i}$.
  Then $L \otimes_{K} \hat K = \prod_{i} L_{i}$.
\end{proposition}

Now let's focus on the nonarchemedean case.
Let $L|K$ be a finite extension of number fields, and $|-|$ compatible nonarchemedean absolute values on $L$ and $K$.
Since these arise from primes, say $\Pfrak|\pfrak$, we can talk about the decompositoin group $Z_{\Pfrak|\pfrak}$.
This is also the stabilizer of the action of $\Gal(L|K)$ on the absolute value chosen for $L$ (why?).
It follows that $Z_{\Pfrak|\pfrak}$ acts on $L$ continuously with respect to the chosen absolute value, and thus we have a natural map
\[ Z_{\Pfrak|\pfrak} \to \Gal(\hat L|\hat K). \]
\begin{proposition}
  This map is an isomorphism.
\end{proposition}
\begin{proof}[Proof sketch]
  This map is clearly injective since $L$ is dense in $\hat L$.
  For $\sigma \in \Gal(L|K)$, the absolute value $|\sigma(-)|$ has decomposition group $\sigma Z_{\Pfrak|\pfrak} \sigma^{-1}$, which has the same size as $Z_{\Pfrak|\pfrak}$, so that its size is bounded above by the degree of $\hat L_{\sigma}|\hat K$, for all such $\sigma$.
  thus we have
  \[ \# \Gal(L|K) = [G:Z] \cdot \# Z \leq \sum_{\sigma \in \Gal(L|K)/Z} [\hat L_{\sigma}:\hat K] \leq [L:K] \]
  So equality must hold everywhere, and the claim follows.
\end{proof}

\begin{proposition}
  This isomorphism is compatible with inertia groups.
\end{proposition}
\begin{proof}
  Exercise.
\end{proof}

The discussion above shows that the ramification behavior of a finite extension of number fields $L|K$ can be checked ``locally'' in the completion.
Explicitly, suppose that $\pfrak$ is a prime of $K$ with associated completion $K_{\pfrak}$.
Let $\Pfrak_{1},\ldots,\Pfrak_{g}$ be the prolongations to $L$, and write $L_{\Pfrak_{i}}$ for the associated completions.
For the extension of local fields $L_{\Pfrak_{i}}|K_{\pfrak}$, we have well-defined invariants $e$ and $f$, and similarly in the global case for the fixed $\Pfrak_{i}|\pfrak$.
The two notions coincide.

Even better, consider the map $K \to \hat K$, and choose an algebraic closure $\Omega$ of $\hat K$, and let $\bar K$ denote the algebraic closure of $K$ inside of $\Omega$.
The proposition above then shows that the map
\[ \Gal(\Omega|\hat K) \to \Gal(\bar K|K) \]
obtained by restriction, is \emph{injective}.
Also, recall that there is a unique extension of $|-|$ to $\Omega$, and thus $\bar K$ inherits an absolute value extending that of $K$ via the inclusion.
This corresponds to a prolongation $\bar \pfrak$ of $\pfrak$, and the iamge of the inclusion above is precisely $Z_{\bar\pfrak|\pfrak}$ (compatibly also with inertia subgroups).

%%% Local Variables:
%%% mode: latex
%%% TeX-master: "main"
%%% End:
