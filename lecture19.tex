\section{Lecture 19}

In today's lecture we will discuss Hensel's lemma and some applications.
First, let me start with another result which should have been discussed last time.
Let's consider a complete discretely valued field $K$, with associated absolute value $|-|$, valuation $v$ and valuation ring $\Ocal$ whose maximal ideal is $\mfrak$.
Recall that $\Ocal$ is a DVR, and is thus a PID.
Let $\pi$ be a uniformizer of $\Ocal$, which just means that $\mfrak = (\pi)$.
Let $R \subset \Ocal$ be a system of representatives for $\kappa := \Ocal/\mfrak$, which contains $0$.

\begin{theorem}
  Every element of $K$ admits a unique representation as a convergent series
  \[ x = \pi^{m}(a_{0} + a_{1}\pi + \cdots) \]
  where $a_{i} \in R$, $a_{0} \neq 0$ and $m \in \Zbb$ is the valuation of $x$.
\end{theorem}
\begin{proof}
  Recall that $x$ can be written uniquely as $\pi^{m} u$ for $u$ a unit of $\Ocal$, so we may as well assume that $x = u$ and $m = 0$.
  The image of $u$ in $\kappa$ has a unique representative $a_{0} \in R$ which is nonzero, hence $u = a_{0} + \pi b_{1}$ for some $b_{1} \in \Ocal$.
  Assume that $a_{0},\ldots,a_{n-1}$ have been found, satisfying
  \[ u = a_{0} + a_{1}\pi + \cdots + a_{n-1}\pi^{n-1} + \pi^{n} b_{n} \]
  for some $b_{n} \in \Ocal$, and that the $a_{i}$ are unique with this property.
  Then the representative $a_{n} \in R$ of $b_{n}$ modulo $\pi$ is again uniquely determined by $u$, and we have $b_{n} = a_{n} + \pi b_{n+1}$.
  We can thus continue in the recursion.
  The series converges because the terms tend to zero.
\end{proof}

Note that if $\Ocal$  and $K$ are the completion of $\Ocal_{0}$ and $K_{0}$ which are possibly noncomplete discrete valuation ring and its fraction field, then the $R$ can be chosen to be contained in $\Ocal_{0}$, since $\Ocal$ and $\Ocal_{0}$ have isomorphic residue fields as discussed in the last lecture.

For example, we can take the system of representatives
\[ R := \{0,1,\ldots,p-1\} \]
for $\Fbb_{p}$, which is the residue field of $\Zbb_{p}$.
Recall here that $\Qbb_{p}$ is the completion of $\Qbb$ with respect to the $p$-adic valuation, and that $\Zbb_{p}$ is its associated valuation ring.
This shows that any element of $\Qbb_{p}$ has a unique representation as a \emph{``Laurent''} series in $p$:
\[ \sum_{n \gg -\infty} a_{n} p^{n} \]
where $a_{n} \in \{0,1,\ldots,p-1\}$ are $p$-adic digits.

Let me now discuss Hensel's lemma, as promised.
\begin{theorem}[Hensel's Lemma]
  Suppose that $K$ is a complete nonarchemedean valued field with valuation ring $\Ocal$ and residue field $\kappa$.
  Let $f \in \Ocal[X]$ be a polynomial.
  Assume that $f$ is \emph{primitive}, which means that the image $\bar f$ in $\kappa[X]$ is nonzero (equivalently, at least one of the coefficients of $f$ is a unit).
  Suppose that
  \[ \bar f = \bar g \cdot \bar h \]
  for some \emph{coprime} $\bar g, \bar h \in \kappa[X]$.
  Then $f$ admits a factorization
  \[ f = g \cdot h \]
  where $g,h \in \Ocal[X]$ such that $\deg(g) = \deg(\bar g)$, $\deg(h) = \deg(\bar h)$ and $\bar g$ resp.~$\bar h$ is the image of $g$ resp.~$h$ in $\kappa[X]$
\end{theorem}
\begin{proof}
  Write $d = \deg(f)$, $m = \deg(\bar g)$, so that $d - m \geq \deg(\bar h)$.
  Choose some polynomials $g_{0}$ and $h_{0}$ mapping to $g$ and $h$ with the same degrees.
  We will modify these $g_{0}$ and $h_{0}$ iteratively and eventually reach $g$ and $h$ of the form
  \[ g = g_{0} + p_{1} \pi + p_{2} \pi^{2} + \cdots \]
  \[ h = h_{0} + q_{1} \pi + q_{2} \pi^{2} + \cdots \]
  where $\pi$ is some element of the maximal ideal.
  Setting $g_{n}$ and $h_{n}$ the truncated version, we would like to ensure that $f = g_{n} h_{n}$ modulo $\pi^{n+1}$ and that the degrees match up.

  Since $\bar g$ and $\bar h$ are coprimes, we can find $a,b$ such that $ag_{0} + b h_{0} = 1$ modulo $\mfrak$.
  Note that the coefficients of $f - g_{0} h_{0}$ and of $ag_{0} + b h_{0} - 1$ all have absoltue value $< 1$; choose one of maximal absolute value call it $\pi$.
  Note that any other coefficient of these polynomials is \todo{contained}{Why?} in $\pi \cdot \Ocal$.

  So far, we found $g_{0}$ and $h_{0}$.
  These satsify:
  \begin{enumerate}
    \item $f = g_{0} h_{0} \bmod \pi^{1}$.
    \item $\deg(g_{0}) = \deg(\bar g)$.
    \item $\deg(h_{0}) = \deg(\bar h)$.
    \item $\bar g_{0} = \bar g$.
    \item $\bar h_{0} = \bar h$.
  \end{enumerate}
  Suppose that $p_{1},\ldots,p_{n}$ and $q_{1},\ldots,q_{n}$ have been found.
  Now we are looking for $g_{n+1}$ and $h_{n+1}$ of the form
  \[ g_{n+1} = g_{n} + p_{n+1}\pi^{n+1}, \ h_{n+1} = h_{n} + q_{n+1} \pi^{n+1}. \]
  Consider the condition on $g_{n+1}h_{n+1}$.
  We have
  \[ g_{n+1} h_{n+1} = g_{n}h_{n} + g_{n} q_{n+1} \pi^{n+1} + h_{n} p_{n+1}\pi^{n+1} + p_{n+1}q_{n+1}\pi^{2n+2}. \]
  We want this to be congruent to $f$ modulo $\pi^{n+2}$, so we can ignore the last term.
  The condition thus becomes
  \[ f - g_{n} h_{n} = \pi^{n+1}(g_{n}q_{n+1}+h_{n}p_{n+1}) \bmod \pi^{n+2}. \]
  Dividing by $\pi^{n+1}$, this reduces to
  \[ f_{n+1} := \frac{f-g_{n}h_{n}}{\pi^{n}} = g_{n} q_{n+1} + h_{n} p_{n+1} \bmod \pi. \]
  Also, note that $g_{n}$ is congruent to $g_{0}$ mod $\pi$ and similarly with the $h$'s, so that the LHS is congruent to $g_{0} q_{n+1} + h_{0} p_{n+1}$ modulo $\pi$.
  Using that $g_{0}a + h_{0} b = 1 \bmod \pi$, a good initial guess is to take $p_{n+1} = b f_{n+1}$ and $q_{n+1} = a f_{n}$.

  The issue with this is that the degrees might become too large.
  What we do instead is use the division algorithm the reduce the degrees, as follows.
  Write
  \[ b \cdot f_{n+1} := q \cdot g_{0} + p_{n+1} \]
  with $\deg(p_{n+1}) < \deg(g_{0}) = \deg(\bar g) = m$.
  Since the degrees of $g$ and $g_{0}$ agree, the leading coefficient of $g_{0}$ is a unit, and thus $q$ is an element of $\Ocal[X]$.
  We thus have
  \[ g_{0}(a f_{n+1} + h_{0}q) + h_{0} p_{n+1} = f_{n+1} \bmod \pi. \]
  Now ignore all terms in the expansion of $a f_{n+1} + h_{0} q$ whose coefficients are divisible by $\pi$ to obtain $q_{n+1}$ such that
  \[ g_{0} q_{n+1} + h_{0} p_{n+1} = f_{n+1} \bmod \pi. \]
  Since $\deg(f_{n+1}) \leq d$, $\deg(g_{0}) = m$ and $\deg(h_{0} p_{n+1}) < d$, we see that $q_{n+1}$ has degree $\leq d-m$, as required.
\end{proof}

%%% Local Variables:
%%% mode: latex
%%% TeX-master: "main"
%%% End:
