\section{Lecture 12}

In today's lecture we will start to discuss our next topic: ramification theory.
Recall the following setup: $A$ is a Dedekind domain with fraction field $K$, $L$ is a seaparble extension of $K$ and $B$ is the normalization of $A$ in $L$.
Given a prime $\Pfrak$ of $B$ lying above $\pfrak$ in $A$, we define
\[ e(\Pfrak|\pfrak), \ f(\Pfrak|\pfrak), \]
both positive integers, as follows:
\begin{enumerate}
  \item First, $f(\Pfrak|\pfrak) = [B/\Pfrak : A/\pfrak]$, the degree of the residue field extension.
  \item Second, $e(\Pfrak|\pfrak)$ is the degree of $\Pfrak$ in the factorization of $\pfrak \cdot B$.
\end{enumerate}
Finally, recall that we have proven the \emph{fundamental identity}: Given a fixed $\pfrak$ as above, one has
\[ [L:K] = \sum_{\Pfrak|\pfrak} e(\Pfrak|\pfrak) \cdot f(\Pfrak|\pfrak). \]

In homework 2, you visited the Dedekind-Kummer theorem.
Let's discuss a variant of this formulated slightly differently.
Let $\theta$ be a primitive element for the extension $L|K$, and assume it is integral over $A$ with minimal polynomial $p$.
Consider the subring $A[\theta]$ of $B$.
The \emph{conductor} of $A[\theta]$ (or just of $\theta$) is the largest ideal of $B$ which is contained in $A[\theta]$.
Explicitly, the conductor, as a set, is given by the following formula:
\[ \{ x \in B \ | \ x \cdot B \subset A[\theta] \} \]
(note that this depends on the algebra generated by $\theta$, and not on $\theta$ itself).
Since $B$ is finite over $A$, we see that this is a nonzero ideal (why?).

\begin{proposition}
  Suppose that $\pfrak$ is a prime of $A$ which is coprime to the conductor of $A[\theta]$.
  Let $\bar p = \bar p_{1}^{e_{1}} \cdots \bar p_{r}^{e_{r}}$ be the factorization of the image of $p$ in $A/\pfrak$ into irreducibles where $p_{i} \in A[X]$ is monic with image $\bar p_{i}$.
  Then the factorization of $\pfrak \cdot B$ is
  \[ \Pfrak_{1}^{e_{1}} \cdots \Pfrak_{r}^{e_{r}} \]
  where $\Pfrak_{i} = \pfrak \cdot B + p_{i}(\theta) \cdot B$.
  The degree of $\bar p_{i}$ is the inertia degree of $\Pfrak_{i}|\pfrak$ and $e_{i}$ above is (obviously) the ramification degree.
\end{proposition}
\begin{proof}
  Argue as in the homework, noting that
  \[ B/\pfrak \cdot B = A[\theta]/\pfrak \cdot A[\theta], \]
  due to the coprimality of $\pfrak$ and the conductor.
\end{proof}

\begin{definition}
  Let $\pfrak$ be a prime ideal of $A$.
  Consider the decomposition of $\pfrak$ in $B$:
  \[ \pfrak \cdot B = \Pfrak_{1}^{e_{1}} \cdots \Pfrak_{k}^{e_{k}}. \]
  We say that $\pfrak$ is totally split in $B$ (or $L$) if $k = [L:K]$, which is equivalent to $e_{i} = f_{i} = 1$ for all $i$.
  We say that $\pfrak$ is nonsplit $k = 1$ or equivalently if there is a unique prime $\Pfrak|\pfrak$.
\end{definition}

\begin{definition}
  We say that $\pfrak$ is unramified in $L$ if $e_{i} = 1$ for all $i$ and the residue field extensiosn are all separable.
  Otherwise we say that $\pfrak$ ramifies in $L$.
  If $\pfrak$ is ramified then we say it is totally ramified if $f_{i} = 1$.
  The extension $L|K$ (or $B|A$) is called (un)ramified if this holds for all primes of $A$.
\end{definition}

\begin{proposition}
  There are only finitely many prime ideals which ramify in $L$.
\end{proposition}
\begin{proof}
  Let $\theta$ be as in the disucssion above.
  Write $d(\theta)$ for the discriminant of the basis $1,\theta,\ldots,\theta^{n-1}$.
  \todo{By the above proposition}{Add label and ref.} it follows that every prime ideal which is coprime to $d$ and to the conductor of $\theta$ is unramified.
  Indeed, the discriminant in question being nonzero in $A/\pfrak$ implies that all the roots the image of $p$ in $A/\pfrak$ are distinct, which is equivalent $\pfrak$ being unramified in in the case where it is coprime to the discriminant.
  The image of $\theta$ also generates the residue extensions in such cases, and \todo{these extensions are therefore separable.}{Why?}
\end{proof}

The proof of the above proposition at least suggests that there is a relationship between the discriminant and ramification.
This is indeed the case, but note that the ramification is a relative notion whereas we really only discussed an \emph{absolute} notion of the discriminant, so we need some slight addendum to the previous discussions.
The \emph{relative discriminant} of $B|A$ is the ideal $\dfrak$ of $A$ which is generated by all the discriminants $d(x_{1},\ldots,x_{n})$ of all bases $x_{1},\ldots,x_{n}$ of $L|K$ which are contained in $B$.

Let's study an important example (quadratic fields).
Consider the number field $\Qbb(\sqrt{D})$, where $D$ is a square-free integer.
Recall that the Legendre symbol $(D/p)$, for $p$ coprime to $D$, is defined to be $1$ if $D$ is a square mod $p$ and $-1$ otherwise.
This is multiplicative: $(AB/p) = (A/p)(B/p)$, and one has $(D/p) = D^{(p-1)/2}$ mod $p$.

Recall from your calculations in the first homework that the conductor of $\Zbb[\sqrt{D}]$ in $\Ocal_{K}$, $K = \Qbb(\sqrt{D})$, is some power of $2$.
From these observations, we obtain:
\begin{proposition}
  Suppose $D$ is square-free and $p$ is a prime number which is coprime to $2 \cdot D$.
  Then $(D/p) = 1$ if and only if $p$ is totally split in $\Qbb(\sqrt{D})$.
\end{proposition}

The quadratic reciprocity law (covered in Math 312, for example), shows that for two distinct odd primes $p$ and $q$, one has
\[ (p/q)(q/p) = (-1)^{T}, \ \ T = (p-1)(q-1)/4. \]
Also, $(-1/p) = (-1)^{(p-1)/2}$ and $(2/p) = (-1)^{(p^{2}-1)/8}$.
Combining all these facts gives a fairly straightforward method for calculating Legendre symbols, and therefore understanding some spitting behavior of primes in quadratic extensions (after ignoring the finitely many primes dividing the discriminant).

%%% Local Variables:
%%% mode: latex
%%% TeX-master: "main"
%%% End:
