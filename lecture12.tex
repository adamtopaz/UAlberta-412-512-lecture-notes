\section{Lecture 12}

In today's lecture we will start to discuss our next topic: ramification theory.
Recall the following setup: $A$ is a Dedekind domain with fraction field $K$, $L$ is a seaparble extension of $K$ and $B$ is the normalization of $A$ in $L$.
Given a prime $\Pfrak$ of $B$ lying above $\pfrak$ in $A$, we define
\[ e(\Pfrak|\pfrak), \ f(\Pfrak|\pfrak), \]
both positive integers, as follows:
\begin{enumerate}
  \item First, $f(\Pfrak|\pfrak) = [B/\Pfrak : A/\pfrak]$, the degree of the residue field extension.
  \item Second, $e(\Pfrak|\pfrak)$ is the degree of $\Pfrak$ in the factorization of $\pfrak \cdot B$.
\end{enumerate}
Finally, recall that we have proven the \emph{fundamental identity}: Given a fixed $\pfrak$ as above, one has
\[ [L:K] = \sum_{\Pfrak|\pfrak} e(\Pfrak|\pfrak) \cdot f(\Pfrak|\pfrak). \]

In homework 2, you visited the Dedekind-Kummer theorem.
Let's discuss a variant of this formulated slightly differently.
Let $\theta$ be a primitive element for the extension $L|K$, and assume it is integral over $A$ with minimal polynomial $p$.
Consider the subring $A[\theta]$ of $B$.
The \emph{conductor} of $A[\theta]$ (or just of $\theta$) is the largest ideal of $B$ which is contained in $A[\theta]$.
Explicitly, the conductor, as a set, is given by the following formula:
\[ \{ x \in B \ | \ x \cdot B \subset A[\theta] \} \]
(note that this depends on the algebra generated by $\theta$, and not on $\theta$ itself).
Since $B$ is finite over $A$, we see that this is a nonzero ideal (why?).

\begin{proposition}
  Suppose that $\pfrak$ is a prime of $A$ which is coprime to the conductor of $A[\theta]$.
  Let $\bar p = \bar p_{1}^{e_{1}} \cdots \bar p_{r}^{e_{r}}$ be the factorization of the image of $p$ in $A/\pfrak$ into irreducibles where $p_{i} \in A[X]$ is monic with image $\bar p_{i}$.
  Then the factorization of $\pfrak \cdot B$ is
  \[ \Pfrak_{1}^{e_{1}} \cdots \Pfrak_{r}^{e_{r}} \]
  where $\Pfrak_{i} = \pfrak \cdot B + p_{i}(\theta) \cdot B$.
  The degree of $\bar p_{i}$ is the inertia degree of $\Pfrak_{i}|\pfrak$ and $e_{i}$ above is (obviously) the ramification degree.
\end{proposition}
\begin{proof}
  Argue as in the homework, noting that
  \[ B/\pfrak \cdot B = A[\theta]/\pfrak \cdot A[\theta], \]
  due to the coprimality of $\pfrak$ and the conductor.
\end{proof}

\begin{definition}
  Let $\pfrak$ be a prime ideal of $A$.
  Consider the decomposition of $\pfrak$ in $B$:
  \[ \pfrak \cdot B = \Pfrak_{1}^{e_{1}} \cdots \Pfrak_{k}^{e_{k}}. \]
  We say that $\pfrak$ is totally split in $B$ (or $L$) if $k = [L:K]$, which is equivalent to $e_{i} = f_{i} = 1$ for all $i$.
  We say that $\pfrak$ is nonsplit $k = 1$ or equivalently if there is a unique prime $\Pfrak|\pfrak$.
\end{definition}

\begin{definition}
  We say that $\pfrak$ is unramified in $L$ if $e_{i} = 1$ for all $i$ and the residue field extensiosn are all separable.
  Otherwise we say that $\pfrak$ ramifies in $L$.
  If $\pfrak$ is ramified then we say it is totally ramified if $f_{i} = 1$.
  The extension $L|K$ (or $B|A$) is called (un)ramified if this holds for all primes of $A$.
\end{definition}

\begin{proposition}
  There are only finitely many prime ideals which ramify in $L$.
\end{proposition}
\begin{proof}
  Let $\theta$ be as in the disucssion above.
  Write $d(\theta)$ for the discriminant of the basis $1,\theta,\ldots,\theta^{n-1}$.
  \todo{By the above proposition}{Add label and ref.} it follows that every prime ideal which is coprime to $d$ and to the conductor of $\theta$ is unramified.
  Indeed, the discriminant in question being nonzero in $A/\pfrak$ implies that all the roots the image of $p$ in $A/\pfrak$ are distinct, which is equivalent $\pfrak$ being unramified in in the case where it is coprime to the discriminant.
  The image of $\theta$ also generates the residue extensions in such cases, and \todo{these extensions are therefore separable.}{Why?}
\end{proof}

The proof of the above proposition at least suggests that there is a relationship between the discriminant and ramification.
This is indeed the case, but note that the ramification is a relative notion whereas we really only discussed an \emph{absolute} notion of the discriminant, so we need some slight addendum to the previous discussions.
The \emph{relative discriminant} of $B|A$ is the ideal $\dfrak$ of $A$ which is generated by all the discriminants $d(x_{1},\ldots,x_{n})$ of all bases $x_{1},\ldots,x_{n}$ of $L|K$ which are contained in $B$.

Let's study an important example (quadratic fields).
Consider the number field $\Qbb(\sqrt{D})$, where $D$ is a square-free integer.
Recall that the Legendre symbol $(D/p)$, for $p$ coprime to $D$, is defined to be $1$ if $D$ is a square mod $p$ and $-1$ otherwise.
This is multiplicative: $(AB/p) = (A/p)(B/p)$, and one has $(D/p) = D^{(p-1)/2}$ mod $p$.

Recall from your calculations in the first homework that the conductor of $\Zbb[\sqrt{D}]$ in $\Ocal_{K}$, $K = \Qbb(\sqrt{D})$, is some power of $2$.
From these observations, we obtain:
\begin{proposition}
  Suppose $D$ is square-free and $p$ is a prime number which is coprime to $2 \cdot D$.
  Then $(D/p) = 1$ if and only if $p$ is totally split in $\Qbb(\sqrt{D})$.
\end{proposition}

The quadratic reciprocity law (covered in Math 312, for example), shows that for two distinct odd primes $p$ and $q$, one has
\[ (p/q)(q/p) = (-1)^{T}, \ \ T = (p-1)(q-1)/4. \]
Also, $(-1/p) = (-1)^{(p-1)/2}$ and $(2/p) = (-1)^{(p^{2}-1)/8}$.
Combining all these facts gives a fairly straightforward method for calculating Legendre symbols, and therefore understanding some spitting behavior of primes in quadratic extensions (after ignoring the finitely many primes dividing the discriminant).

Let us now consider the case where the field extension $L|K$ is Galois with Galois group $G$.
We saw before that $G$ acts in a natural way on the set of primes lying above a fixed prime $\pfrak$ of $A$, and that this action is \emph{transitive}.
Let $\Pfrak|\pfrak$ be an extension of primes in $B|A$.
We define
\[ Z_{\Pfrak|\pfrak} \subset G, \]
the decomposition group of $\Pfrak|\pfrak$ (or just $\Pfrak$ when $A$, or equivalently $\pfrak$, is understood from context), to be the \emph{stabilizer} of $\Pfrak$ with respect to this action.
The fixed field of $Z_{\Pfrak|\pfrak}$ in $L|K$ will be called the \emph{decomposition field} of $\Pfrak$ over $K$.

By the orbit-stabilizer theorem, we see that $G/Z_{\Pfrak|\pfrak}$ is in a natural bijection with the primes lying above $\pfrak$ via the bijection $\sigma \mapsto \sigma\Pfrak$.
It follows that $Z_{\Pfrak|\pfrak}$ is trivial if and only if $L$ is the decomposition field of $\Pfrak|\pfrak$, which is equivalent to saying that $\pfrak$ is totally split in $L$.
Similarly, if $Z_{\Pfrak|\pfrak} = G$ if and only if $\pfrak$ is nonsplit.

The decomposition groups are conjugate in $G$, as follows:
\[ Z_{\sigma\Pfrak|\pfrak} = \sigma Z_{\Pfrak|\pfrak} \sigma^{-1}. \]
\todo{This is a simple (purely group-theoretic exercise).}{Add argument.}

Recall that in the Galois case which we are in, the inertia and ramification degrees do not depend on the choice of prime $\Pfrak$ above $\pfrak$.
If $\Pfrak$ is a fixed prime above $\pfrak$, we have
\[ \pfrak = (\prod_{\sigma \in G/Z_{\Pfrak|\pfrak}} \sigma\Pfrak)^{e} \]
where $e = e(\Pfrak|\pfrak)$.

\begin{proposition}
  Let $M$ denote the decomposition field of $\Pfrak$ and let $\Pfrak_{Z} := \Pfrak \cap M$.
  The following hold:
  \begin{enumerate}
    \item $\Pfrak_{Z}$ is nonsplit in $L$, meaning that $\Pfrak$ is the only prime of $L$ lying above $\Pfrak_{Z}$.
    \item $\Pfrak|\Pfrak_{Z}$ has the same ramification and inertia degrees as $\Pfrak|\pfrak$.
    \item The ramification and inertia degree of $\Pfrak_{Z}|\pfrak$ are both $1$.
  \end{enumerate}
\end{proposition}
\begin{proof}
  These assertions all follow essentially from the fundamental identity and the fact that $\Gal(L|M) = Z_{\Pfrak|\pfrak}$, along with the observations above.
  \todo{Details left as an exercise.}{Add argument.}
\end{proof}

The decomposition group as we see above ``understands'' how many primes are above $\pfrak$.
In order to get information about the ramification degree, we will need another invariant, called the \emph{inertia group}.
First, let's make some observations.
Let $\sigma \in Z_{\Pfrak|\pfrak}$ be given.
This implies that $\sigma \Pfrak = \Pfrak$, while $\sigma B = B$ as well.
Thus $\sigma$ acts (by a field automorphism) on the residue field $B/\Pfrak$.
This gives us a map
\[ Z_{\Pfrak|\pfrak} \to \Aut(B/\Pfrak|A/\pfrak) \]
(note that the residue extension might be unramified, so I write $\Aut$ instead of $\Gal$; we also don't know yet whether it's normal).
We will write $\kappa(-)$ for the residue field of a prime.

\begin{proposition}
  The extension $\kappa\Pfrak|\kappa\pfrak$ is normal and the map
  \[ Z_{\Pfrak|\pfrak} \to \Aut(\kappa\Pfrak|\kappa\pfrak) \]
  is surjective.
\end{proposition}
\begin{proof}
  Let again $\Pfrak_{Z}$ denote the prime associated to $\Pfrak$ in the decomposition field of $\Pfrak|\pfrak$.
  The residue field $\kappa\Pfrak_{Z}$ is the same as that of $\pfrak$.
  We may thus assume that $\pfrak$ is nonsplit in $L$ by replacing $K$ with the decomposition field.

  Let $\theta$ be an integral element with image $\bar \theta$.
  Let $f$ be the minimal polynomial over $K$ and $\bar g$ the minimal polynomial of the image in $\kappa\Pfrak$ over $\kappa\pfrak$.
  Note that $\bar g$ divides $\bar f$.
  As $L|K$ is normal, $f$ splits completely so $\bar f$ splits as well.
  It follows that the residue extension is normal indeed.

  Now suppose $\bar\theta$ is a primitive element for the maximal separable subextension of the residue extension.
  Note that $\Aut(\kappa\Pfrak|\kappa\pfrak) = \Gal(\kappa\pfrak(\bar\theta)|\kappa\pfrak)$.
  Let $\bar\sigma$ be in this group so $\bar\sigma\bar\theta$ is a root of $\bar g$ hence also of $\bar f$.
  Thus there is a zero $\theta'$ of $f$ such that $\theta' = \bar\sigma\bar\theta$ modulo $\Pfrak$.
  But $\theta'$ is conjugate to $\theta$ so $\theta' = \sigma\theta$ for some $\sigma \in G$, and this $\sigma$ maps to $\bar \sigma$.
\end{proof}

\begin{definition}
  The kernel of the cnaonical map
  \[ Z_{\Pfrak|\pfrak} \to \Aut(\kappa\Pfrak|\kappa\pfrak) \]
  is called the \emph{inertia group} of $\Pfrak|\pfrak$, and will be denoted by $T_{\Pfrak|\pfrak}$.
\end{definition}

We have the groups $T_{\Pfrak|\pfrak} \subset Z_{\Pfrak|\pfrak} \subset \Gal(L|K)$.
We consider the fixed fields
\[ K \subset L^{Z} \subset L^{T} \subset L. \]
The $L^{Z}$ is the decomposition field considered above while the $L^{T}$ is called the \emph{inertia field}.
We also have an exact sequence
\[ 1 \to T \to Z \to \Aut \to 1. \]

\begin{proposition}
  The extension $L^{T}|L^{Z}$ is Galois with Galois group $\Aut(\kappa\Pfrak|\kappa\pfrak)$.
  If the residue extension is separable (\todo{this always holds if $K$ is a number field}{Why?}) then
  \[ \# T_{\Pfrak|\pfrak} = e(\Pfrak|\pfrak), \ [Z_{\Pfrak|\pfrak}|T_{\Pfrak|\pfrak}] = f(\Pfrak|\pfrak). \]
  Letting $\Pfrak_{Z}$ and $\Pfrak_{T}$ denote the primes below $\Pfrak$ assocaited to the inertia and decomposition fields, we have
  \begin{enumerate}
    \item $e(\Pfrak|\Pfrak_{T}) = e(\Pfrak|\pfrak)$ and $f(\Pfrak|\Pfrak_{T}) = 1$.
    \item $\Pfrak_{T}|\Pfrak_{Z}$ is unramified and $f(\Pfrak_{T}|\Pfrak_{Z}) = f$.
  \end{enumerate}
\end{proposition}

%%% Local Variables:
%%% mode: latex
%%% TeX-master: "main"
%%% End:
