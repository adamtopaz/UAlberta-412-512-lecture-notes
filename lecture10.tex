\section{Lecture 10}

Today we'll start by computing some more examples of class groups.
Here is an interesting one: $K = \Qbb(\sqrt{-65})$.
The discriminant is $-260$, while $n = 2$, $r = 0$ and $s = 1$.
Thus the Minkowski bound is approximately $10.27$, so we need to consider the primes lying above $2$, $3$, $5$ and $7$.

Note that $-65 \equiv 3 \bmod 4$, so the ring of integers is $\Zbb[\sqrt{-65}]$.
We must thus factor $X^{2} + 65$ over $\Fbb_{p}$ for $p = 2,3,5,7$.
\begin{enumerate}
  \item[p = 2:] We get $X^{2}+1 = (X+1)^{2}$, so $(2) = \pfrak_{2}^{2}$.
  \item[p = 3:] We get $X^{2}+2 = (X+1)(X-1)$, so $(3) = \pfrak_{3} \cdot \pfrak_{3}'$.
  \item[p = 5:] We get $X^{2}$ so $(5) = \pfrak_{5}^{2}$.
  \item[p = 7:] We get $X^{2} + 2$ which is irreducible over $\Fbb_{7}$, so $(7)$ is prime.
\end{enumerate}
Since $(7)$ is principal while $\pfrak_{3}' = \pfrak_{3}^{-1}$ in the class group, we see that $Cl_{K}$ is generated by $\pfrak_{2},\pfrak_{3},\pfrak_{5}$.

Let's compute the norms of these elements.
Note also that $f(\mathfrak p_p|p) = 1$ for $p=2, 3, 5, 7$, so the norm of $\pfrak_{p}$ is precisely $p$.
Since $p = 2,3,5$ cannot be written as $x^{2} + 65 y^{2}$ for any integer $x,y$, it follows that $\pfrak_{p}$ are not principal, so define nontrivial elements in the class group.
Note also that $\pfrak_{2}$ and $\pfrak_{5}$ have order $2$ in the class group.

Next, we look for integral elements whose norms involve primes $2,3,5$, since such elements will give us additional relations in the class group.
We can do this systematically by looking at $x + y \cdot \sqrt{-65}$ for small (in absolute value) integers $x,y$.

For example, $4 + \sqrt{-65}$ has norm $3^{4}$ so that $(4 + \sqrt{-65})$ involves only $\pfrak_{3}$ and $\pfrak_{3}'$.
If it involved both $\pfrak_{3}$ and $\pfrak_{3}'$, then $3$ would have to divide $4 + \sqrt{-65}$, which it doesn't.
It follows therefore that $(4 + \sqrt{-65}) = \pfrak_{3}^{4}$ or $\pfrak_{3}'^{4}$, and by the symmetry of the situation, we may as well assume that it's $\pfrak_{3}^{4}$ to deduce that $\pfrak_{3}$ has order dividing $4$, and since it's order is not $1$ or $2$ because $3$ and $9$ cannot be written as $x^2+65y^2$ for integer $x,y$, it has to have order $4$.

Let's consider $5 + \sqrt{-65}$ as well.
Its norm is $2 \cdot 3^{2} \cdot 5$.
Note that $\pfrak_{3}$ cannot divide $(5 + \sqrt{-65})$, for otherwise it would divide $(5 + \sqrt{-65}) + (4 + \sqrt{-65}) = (1)$.
It follows that
\[ (5 + \sqrt{-65}) = \pfrak_{2} \pfrak_{3}'^{2} \pfrak_{5}. \]
We deduce that
\[ \pfrak_{5} = \pfrak_{2} \cdot \pfrak_{3}^{2} \]
in the class group.
Thus the class group is generated by $\pfrak_{2}$, of order $2$ and by $\pfrak_{3}$, of order $4$.

In other words, we have a surjective map
\[ \Zbb/2 \times \Zbb/4 \to Cl_{K}. \]
The only way for this to \emph{not be an isomorphism} is if $\pfrak_{3}^{2} = \pfrak_{2}$ in the class group, since we already know that $\pfrak_{3}$ and $\pfrak_{2}$ are nonprincipal.

If this is the case, then $\pfrak_{2} \pfrak_{3}^{2}$ would be principal, say generated by $x + y \cdot \sqrt{-65}$, and its norm would be $2 \cdot 3^{2} = x^{2} + 65 \cdot y^{2}$.
But we can easily check that this equation has no integral solutions.
It follows that $\pfrak_{2} \pfrak_{3}^{2}$ is \emph{nontrivial} in the class group, and thus the class group of $K$ is isomorphic to $\Zbb/2 \times \Zbb/4$, with corresponding generators $\pfrak_{2}$ and $\pfrak_{3}$.

Let's think about what might happen if we took a nonimaginary quadratic extension, such as $K = \Qbb(\sqrt{82})$.
One can check that the Minkowski bound is approximately $9.06$ in this case, while $\Ocal_{K} = \Zbb[\sqrt{82}]$.
Factoring $x^{2}-82$ modulo $p$ for $p = 2,3,5,7$ gives
\[ (2) = \pfrak_{2}^{2}, \ (3) = \pfrak_{3} \pfrak_{3}' \]
while $(5)$ and $(7)$ are primes.
It follows that the class group is generated by $\pfrak_{2}$ and $\pfrak_{3}$, with $\pfrak_{2}$ having order $2$.

Arguing similar to the above, we can consider the norm of $10 + \sqrt{82}$ to see that $\pfrak_{2} \pfrak_{3}^{2}$ is principal, possibly after switching $\pfrak_{3}$ and $pfrak_{3}'$.
It follows that the class group is generated by $\pfrak_{3}$, which has order dividing $4$, and its square is $\pfrak_{2}$, all in the class group.
We also know that $\pfrak_{3}$ is nontrivial.
Thus the class group is $\Zbb/4$ or $\Zbb/2$, generated by $\pfrak_{3}$.

To check whether it is $\Zbb/4$ and not $\Zbb/2$, we need to check whether $\pfrak_{2}$ is principal, which means looking for $x + y \sqrt{82}$ where $|x^{2} - 82 \cdot y^{2}| = 2$.
This is a much more difficult task, because of the negative sign.
One can \emph{eventually} show that this equation has no integral solutions, but the computation involves an understanding of the unit group $\Ocal_{K}^{\times}$, which \emph{is} infinite in this case (this is the next topic we will discuss).
Indeed, if $(\alpha) = \pfrak_{2}$, then $(2) = (\alpha^{2})$, which means that $2 \cdot u = \alpha^{2}$ for some $u \in \Ocal_{K}^{\times}$.
One can eventually use this observation, together with an understanding o the structure of $\Ocal_{K}^{\times}$ (see later) to see that a solution to the above would imply that $\pm 2$ is a square in $K$, which cannot happen.

Let's consider a cubic example next.
Consider the polynomial $f = T^{3} - T - 9$.
It's irreducible modulo $2$ (where it's an Artin-Schreier polynoimal), so it must be irreducible over $\Zbb$ hence also over $\Qbb$.
Letting $\alpha$ be a root, $K = \Qbb(\alpha)$ is a cubic extension of $\Qbb$.

The discriminant of $f$ is $-37 \cdot 59$, which is square-free, hence it follows that $\Ocal_{K} = \Zbb[\alpha]$ due to the formula
\[ d_{K} = [\Ocal_{K}:\Zbb[\alpha]]^{2} \cdot d(1,\alpha,\alpha^{2}). \]
This polynomial has exactly one real root so $r = s = 1$, and thus the Minkowski bound is about $13.21$.
We must thus consider primes above $p = 2,3,5,7,11,13$.
Factoring $f$ over such primes yields:
\begin{enumerate}
  \item $(2)$, $(7)$ and $(13)$ are all prime.
  \item $(3) = \pfrak_{3} \pfrak_{3}' \pfrak_{3}''$.
  \item $(5) = \pfrak_{5} \pfrak_{5}'$ with $f(\pfrak_{5}'|5) = 2$.
  \item $(11) = \pfrak_{11} \pfrak_{11}'$ with $f(\pfrak_{11}'|5) = 2$.
\end{enumerate}
The class group is thus generated by $\pfrak_{3},\pfrak_{3}',\pfrak_{5},\pfrak_{11}$.

We can get additional relations among these primes by factoring $\alpha + n$ for integers $n$.
This has good reason to work.
Indeed, if $f(\pfrak|p) = 1$ then $\alpha + n$ will be divisible by $p$ for some integer $n$, and thus the same will be true for its (absolute) norm.
If this doesn't happen, then it must be the case that $f(\pfrak|p) > 1$, and it it happens for both integers $n$ and $m$, then $m$ and $n$ must be congruent modulo the rational prime $p$.

To compute the norm of $\alpha + n$, simply note that its minimal polynomial is $f(X - n)$ which has constant term $-n^{3} + n - 9$ so the norm of $\alpha + n$ has absolute value $|n^{3} - n + 9|$.

From this we can calculate the norms of $(\alpha + n)$ for small values of $n$ to obtain, after rearranging the primes above $3$,
\[ (\alpha) = \pfrak_{3}^{2}, \ (\alpha + 1) = \pfrak_{3}'^{2}, \ (\alpha - 1) = \pfrak_{3}''^{2}. \]

Next note that $\alpha - 2$ has norm $3$.
Also, since $\alpha + 1 \in (\alpha - 2)$, we find that $(\alpha - 2) = \pfrak_{3}'$ hence $\pfrak_{3}'$ is trivial in the class group.

Next note that $\alpha + 2$ has norm $3 \cdot 5$ and argue similarlly (using the fact that $\alpha + 2 - 3 = \alpha - 1$) to deduce that
\[ (\alpha + 2) = \pfrak_{3}'' \pfrak_{5}.\]
Finally, a similar argument would show that $(\alpha + 3) = \pfrak_{3} \cdot \pfrak_{11}$.
We conclude that the class group is generated by $\pfrak_{3}$ alone, and the above shows that it has order dividing $2$.
We will again have to use facts about units of $\Ocal_{K}$ to deduce that $\pfrak_{3}$ is nontrivial in the class group, so we leave the remainder of this example to a later date.

%%% Local Variables:
%%% mode: latex
%%% TeX-master: "main"
%%% End:
