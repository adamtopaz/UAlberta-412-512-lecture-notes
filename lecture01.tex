\section{Lecture 1}

This lecture is mostly an introduction and overview.
The we will revisit most of the concepts mentioned in this lecture at some point later on in the term.

As with all of mathematics, number theory starts with the \emph{natural numbers}
\[ \Nbb = \{0,1,2,\ldots\}. \]
This is a commutative \emph{semiring}, which is similar to a ring, except for the fact that addition gives us a (commutative) monoid as opposed to a group (all this means is that we cannot negate elements).

There is a general process by which we can turn a semiring into a ring in a ``free way.''
In more pedantic terms, the forgetful functor from the category of rings to the category of semirings has a left adjoint.
If we apply this construction to $\Nbb$, we obtain the \emph{integers}
\[ \Zbb = \{\ldots,-2,-1,0,1,2,\ldots\}, \]
which forms a commutative ring in the usual way.

In fact, $\Zbb$ is an \emph{integeral domain}, which means that it has no zero divisors.
The collection of all nonzero elements is then a multiplicative submonoid of $\Zbb$.
If we localize $\Zbb$ with respect to this submonoid, we obtain the \emph{rational numbers}
\[ \Qbb = \left\{\frac{a}{b} \ \Big| \ a,b \in \Zbb, \ b \neq 0 \right\}, \]
which is the \emph{fraction field} of $\Zbb$.

Now that we have the field of rational numbers, we would like to go further.
One way to do this is to take completions with respect to absolute values.
\begin{definition}
  Let $K$ be a field.
  An \emph{absolute value} on $K$ is a function
  \[ | - | : K \to \Rbb_{\geq 0} \]
  satisfying the following conditions:
  \begin{enumerate}
    \item One has $|0| = 0$ and if $x \in K$ satisfies $|x| = 0$ then $x = 0$.
    \item One has $|x \cdot y| = |x| \cdot |y|$ for all $x,y \in K$.
    \item One has $|x + y| \le |x| + |y|$ for all $x,y \in K$.
  \end{enumerate}
\end{definition}

\begin{definition}
  An absolute value $|-|$ on a field $K$ is called \emph{nonarchemedean} provided that the image of the map
  \[ \Nbb \xrightarrow{n \mapsto n} K \xrightarrow{|-|} \Rbb_{\geq 0} \]
  is a bounded subsset of $\Rbb_{\geq 0}$.
  An absolute value which is not nonarchemedean is, of course, called archemedean.
\end{definition}

\begin{example}
  If $K$ is any field, then the \emph{trivial absolute value} on $K$ is given by the following formula:
  \[ |x| = \begin{cases}
             0 & x = 0 \\
             1 & x \neq 0.
           \end{cases} \]
\end{example}

\begin{example}
  The \emph{standard absolute value} on $\Cbb$ is given by
  \[ |a + i \cdot b| = \sqrt{a^{2} + b^{2}}, \ a,b \in \Rbb. \]
\end{example}

\begin{example}
  If $\sigma : K \to L$ is an embedding of fields and $|-|_{L}$ is an absolute value on $L$, then its restriction along $\sigma$ is an absolute value on $K$.
  The restriction of the standard absolute value on $\Cbb$ to $\Qbb$ is denoted by $|-|_{\infty}$, and is the unique \emph{archemedean} absolute value on $\Qbb$.
\end{example}

\begin{example}\label{example:pi-adic_absolute_value}
  Let $A$ be a UFD and $\pi$ a prime element of $A$.
  Let $K$ denote the fraction field of $A$.
  Recall that any nonzero element $x$ of $K$ can be written (in an essentially unique way) in the form
  \[ x = u \cdot \pi^{e} \cdot \pi_{1}^{e_{1}} \cdots \pi_{k}^{e_{k}}, \]
  where $u \in A^{\times}$, $e, e_{i} \in \Zbb$, $\pi,\pi_{1},\ldots,\pi_{k}$ are distinct prime elements of $A$, $k \in \Nbb$, and $e_{i} \neq 0$.
  The map sending such an $x$ to $e \in \Zbb$ is a well-defined function
  \[ v_{\pi} : K^{\times} \to \Zbb. \]
  If $c$ is some positive real number, then the formula
  \[ |x| = \begin{cases}
             c^{-v_{\pi}(x)} & x \neq 0 \\
             0 & x = 0
           \end{cases} \]
  yields an absolute value on $K$ which is nonarchemedean.

  If $A = \Zbb$, $\pi = p$ and $c = p$, then the absolute value constructed above will be denoted by $|-|_{p}$, and will be called the \emph{$p$-adic absolute value} on $\Qbb$.
\end{example}

Note that an absolute value on a field $K$ induces a metric topology on $K$ via the formula $d(x,y) = |x-y|$.
Two absolute values are said to be \emph{equivalent} provided that they induce the same topology.
\begin{proposition}
  Two absolute values $|-|_{1}$ and $|-|_{2}$ on a field $K$ are equivalent if and only if there exists some $r \in \Rbb_{> 0}$ such that $|-|_{1} = |-|_{2}^{r}$.
\end{proposition}
In particular, the \emph{equivalence class} of the absolute value described in Example~\ref{example:pi-adic_absolute_value} only depends on $(A,\pi \cdot A^{\times})$, and not on the choice of $c$.

Now that we are equipped with the notion of an absolute value, we can try to classify all absolute values on $\Qbb$.
This is the content of Ostrowski's theorem.
\begin{theorem}[Ostrowski]
  Let $|-|$ be a nontrivial absolute value on $\Qbb$.
  Then $|-|$ is equivalent to $|-|_{p}$ for some prime number $p$, or to $|-|_{\infty}$.
\end{theorem}

If we complete $\Qbb$ with respect to $|-|_{\infty}$, then we obtain $\Rbb$, as usual.
However, if we complete with respect to $|-|_{p}$, we obtain the field of \emph{$p$-adic numbers}, denoted $\Qbb_{p}$.

There is another way to go beyond $\Qbb$ as one does in Galois theory, by adjoining roots of polynomials.
If we adjoin all roots of all polynomials, and iterate the process, we obtain an algebraically closed field containing $\Qbb$, say $\bar \Qbb$.
We can then try to understand how absolute values on $\Qbb$ \emph{extend} to absolute values on $\bar\Qbb$. We will later see that they always do extend, in various (interesting) ways, and study the structure of such extensions via Galois theory.
For now, let's just fix some choice of extension of $|-|_{p}$ and $|-|_{\infty}$, and use the same notation for the extension.
If we complete $\bar\Qbb$ with respect to $|-|_{\infty}$, we will obtain $\Cbb$, the algebraic closure of $\Rbb$.
On the other hand, if we complete $\bar\Qbb$ with respect to $|-|_{p}$, we obtain $\Cbb_{p}$, which is \emph{larger} than the algebraic closure of $\Qbb_{p}$, but is nevertheless still algebraically closed (we will see this later as a consequence of Krasner's Lemma).


%%% Local Variables:
%%% mode: latex
%%% TeX-master: "main"
%%% End:
