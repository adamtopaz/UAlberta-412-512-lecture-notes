\section{Extra Lecture 2}

In this lecture we will discuss the \emph{global theory}, continuing from the previous lecture.
I will discuss an argument is again different from Neukirch's original approach, which is due essentially to \emph{Hoshi}.

Recall that we are in the following context.
We have two number fields $K_{i}$, $i = 1,2$, and an isomorphism
\[ \alpha : \Gal_{K_{1}} \cong \Gal_{K_{2}} \]
between their absolute Galois groups.
In the last lecture we saw that $\alpha$ is compatible with decomposition groups, and that primes are uniquely determined by their decomposition groups.

Let's make this more explicit.
First, consider the set $\Xfrak_{i}$ of pairs $\bar v|v$ where $v = v_{\pfrak}$ is a valuation associated to a prime $\pfrak$ of $K_{i}$, and $\bar v$ is a a prolongation of $v$ to $\bar K_{i}$.
Then there is a natural \emph{bijection} $\beta : \Xfrak_{1} \cong \Xfrak_{2}$, which satisfies (and is uniquely determined by) the following property: For all $\bar v|v \in \Xfrak_{1}$, one has
\[ \alpha(Z_{\bar v|v}) = Z_{\beta(\bar v|v)}. \]
Furthermore, there is a natural Galois action of $\Gal_{K_{i}}$ on $\Xfrak_{i}$, and this bijection is compatible with these actions.
Namely, for $\sigma \in \Gal_{K_{1}}$ and $\bar v|v \in \Xfrak_{i}$, one has
\[ \beta(\sigma(\bar v|v)) = \alpha(\sigma)\beta(\bar v|v). \]
This follows from the fact that
\[ \sigma Z_{\bar v|v} \sigma^{-1} = Z_{\sigma(\bar v|v)}, \]
and the fact that $\beta$ is determined by $\alpha$ as described above.
By modding out by this action, we obtain a bijection
\[ \bar\beta : P_{1} \cong P_{2} \]
where $P_{i}$ is the collection of primes of $K_{i}$, and $\bar \beta$ is induced by $\beta$.

Let's see what else we can do with this information.
First, let's fix some $\bar v|v \in \Xfrak_{i}$.
The decomposition group $Z_{\bar v|v}$ is isomorphic to the absolute Galois group of $K_{v}$, as discussed in class, where $K_{v}$ is the completion of $K$ with respect to the absolute value assocaited with $v$.
This absolute Galois group therefore determines the following invariants:
\begin{enumerate}
  \item The residue characteristic, and the size of the residue field of $v$.
  \item The absolute ramification degree of $v$ and the degree of $K_{v}$ over $\Qbb_{p}$ where $p$ is the residue characteristic of $\kappa(v)$.
\end{enumerate}
This also determines the inertia subgroup of $Z_{\bar v|v}$ purely group-theoretically, by taking the intersection of all open subgroups whose assocaited absolute ramification degree is the same as that of $K_{v}$.

The prime-to-$p$ part of the inertia group of $\Gal_{K_{v}}$ is isomorphic to the prime to $p$-part of $\hat\Zbb$, and the conjugation action of the absolute Galois group of the residue field factors through the cyclotomic character.
This allows us to determine the Frobenius element in the quotient of $Z_{\bar v|v}$ modulo the inertia group, simply as the unique element which acts on this prime-to-$p$ part by raising to the $q$-th powers.
This therefore determines the Weil group in $\Gal_{K_{v}}$ in a group-theoretic way, and by local class field theory its abelianization is isomorphic to $K_{v}^{\times}$.
Recall that the inertia group corresponds to the unit group and that the the Sylow $p$-subgroup of the unit group is the group of principal units.
Additionally, we obtain the natural map
\[ v : K_{v}^{\times} \to \Zbb \]
since the frobenius element maps to $1$ with respect to $v$ (and the local reciprocity map).

By working in the abelianization of $\Gal_{K_{i}}$, and looking at the images of $Z_{\bar v|v}$ in there, we obtain the decomposition groups in the abelianization of the Galois group of the number field.
By the discussion above, we thus obtain the \emph{global reciprocity map}
\[ \Ibb^{f}_{K} = \prod_{v}{}^{'}K_{v} \to \Gal_{K}^{\ab}, \]
restricted to the \emph{finite ideles} (the restricted product over the nonarchemedean places).
By GCFT, the kernel of this map is \emph{contained} in $K^{\times}$ (it's not the whole thing since we're missing the archemedean part).
One can check using global class field theory that if $K$ is sufficiently large (for example $\sqrt{-1} \in K$ suffices), then the torsion of this kernel agrees with the torsion of $K$.
Passing to the colimit over $L|K$ finite (equivalently, over open subgroups of $\Gal_{K}$), we thus obtain $\mu_{\infty}$ as a subgroup of $\bar K^{\times}$, compatibly with the Galois action.
Let $\hat\Zbb(1)$ denote the $\Gal_{K}$-module isomorphic to $\hat\Zbb$ as an abelian group, where the action is cyclotomic.
Then compute $\HH^{1}(K,\hat\Zbb(1))$, and use Kummer theory to identify this with $\hat{K^{\times}}$, the \emph{adic} completion of $K^{\times}$.
We can restrict to decomposition groups to obtain $\hat{K_{v}^{\times}}$ in a similar way.

The map $K^{\times} \to \hat{K^{\times}}$ is injective.
Now use the fact that $x \in \hat{K^{\times}}$ is contained in $K^{\times}$ if and only if its image in $\hat{K_{v}^{\times}}$ is contained in $K_{v}^{\times}$ for all $v$ to characterize $K^{\times}$.
(There is a subtlety here that we characterized $K_{v}^{\times}$ using local class field theory, whereas the above characterizes it using Kummer theory, but this isn't an issue due to the definition of the local reciprocity map using cup-products and Kummer theory, as discussed in class.)

In any case, we deduce that $K^{\times}$ can be characterized using $\Gal_{K}$.
At the same time, we also obtain the natural maps
\[ K^{\times} \to K_{v}^{\times} \]
for every $v$.

To summarize, we see that $\alpha$ induces an isomorphism of multiplicative groups
\[ K_{1}^{\times} \cong K_{2}^{\times} \]
as well as isomorphisms
\[ K_{1,v}^{\times} \cong K_{2,\beta v}^{\times} \]
for every $v$, compatible with the isomorphism $K_{1}^{\times} \cong K_{2}^{\times}$ via the natural inclusions.
By formally adding a ``zero'' which is a multiplicative absorbing element, we get an isomorphism $K_{1} \cong K_{2}$ of multiplicative monoids compatible with isomorphisms of multiplicative monoids $K_{1,v} \cong K_{2,\beta v}$ for all $v$.
By restricting the valuation $v$ from $K_{v}$ we see that the isomorphism $K_{1} \cong K_{2}$ of monoids is compatible with $v$ in the sense that $v : K_{1} \to \Zbb$ agrees with $K_{1} \cong K_{2} \xrightarrow{\beta v} \Zbb$.
This implies that the unit groups $U_{v}$ are mapped isomorphically onto $U_{\beta v}$ for all $v$.
Additionally, the principal units $U_{v}^{1}$ are mapped isomorphically to $U_{\beta v}^{1}$ for all $v$, as a consequence of the local picture.

Let $\delta : K_{1} \cong K_{2}$ be the isomorphism above.
Note that $\delta$ actually restricts to an automorphism of multiplicative monoids on $\Qbb$.
Indeed, we can characterize $\Qbb$ as the set of elements $x$ of $K_{i}$ such that $x^{p-1} \in U_{v}^{1}$ for all $v$ where $x \in U_{v}$, where $p$ is the residue characteriztic of $v$ (this is an exercise using Chebotarev!).
Since these invariants are compatible with $\delta$, the claim follows.
This automorphism of $\Qbb$ (as a multiplicative monoid) is again compatible with all the (analogous) data mentioned above.

The multiplicative group $\Qbb^{\times}$ has the form
\[ \{\pm 1\} \times \bigoplus_{p} p^{\Zbb}, \]
and $\delta$ induces an automorphism of this object compatible with the various valuations $v_{p}$, taking value in $\Zbb$.
This implies that $\delta(-1) = -1$, and $\delta(p) = \pm p$ for all $p$.

If there are two primes $p$ and $\ell$ such that $\delta(p) = -p$ and $\delta(\ell) = -\ell$, then $\delta(p\ell) = p\ell$, and otherwise $\delta(p) = p$ for some prime $p$.
In any case, we see that there exists a nontorsion element $t \in \Qbb^{\times}$ such that $\delta(t) = t$.

Let $s$ be an arbitrary element of $\Qbb^{\times}$, and let $p$ be a prime where both $s$ and $t$ are $p$-adic units.
Recall that $\delta$ induces an automorphism $\Fbb_{p}^{\times} \cong \Fbb_{p}^{\times}$.
The collection $T$ of such $p$ where $s$ is some power of $t$ in the residue field $\Fbb_{p}$ is \emph{infinite} (this is not completely obvious, but it's true, and the argument is elementary -- I may add an explanation later on; see if you can come up with your own argument).
These observations imply that $\delta(s) \in s \cdot U_{p}^{1}$ for all such $p \in T$, while the intersection of $U_{p}^{1}$ as $p$ varies is the trivial group.
This shows that $\delta(s) = s$ for all $s \in \Qbb^{\times}$.

Now let's conclude by showing that $\alpha$ is a field isomorphism, i.e.~that it's compatible with addition.
Let $x,y \in K_{1}$ be given such that $x \neq -y$ (this is something we can test, since $\alpha(-1) = -1$).

Consider the collection $R$ of all primes $\pfrak$ of $K_{1}$ satisfying the following:
\begin{enumerate}
  \item $\pfrak$ lies over a rational prime which is totally split in $K_{1}$.
  \item $x$, $y$ and $x+y$ are all units at $\pfrak$.
\end{enumerate}
By Chebotarev, the set $R$ is infinite, and the discussion above shows that $\beta R$ is precisely the set satisfying the analogous conditions for $K_{2}$.

Since $\pfrak \in R$ lies over a totally split prime, there exist $x'$ and $y' \in \Qbb$ such that $x',y',x'+y'$ are units at $\pfrak$ and the images of $x$ and $x'$ agree in the residue field of $\pfrak$ and similarly for $y$ and $y'$.
But since $\delta(x'+y') = x'+y'$, we deduce that $\alpha(x+y)$ agrees with $\alpha(x) + \alpha(y)$ modulo $U_{\beta\pfrak}^{1}$ for all such $\pfrak$.
As the set $R$ (hence $\beta R$) is infinite, this shows that $\alpha(x+y) = \alpha(x) + \alpha(y)$.
This concludes the proof that $\alpha$ is a field isomorphism $K_{1} \cong K_{2}$, as contended.

%%% Local Variables:
%%% mode: latex
%%% TeX-master: "main"
%%% End:
