\section{Lecture 6}

The goal of this lecture is to prove Theorems~\ref{theorem:ideal_factorization} and~\ref{theorem:fractional_ideal_factorization}.
Throughout, $\Ocal$ will denote a Dedekind domain with fraction field $K$.
We start with some preliminary lemmas.

\begin{lemma}
  Let $\afrak$ be a nonzero ideal of $\Ocal$.
  Then there exist nonzero prime ideals $\pfrak_{1},\ldots,\pfrak_{k}$ of $\Ocal$ such that
  \[ \pfrak_{1} \cdots \pfrak_{k} \subset \afrak. \]
\end{lemma}
\begin{proof}
  Suppose not.
  Since $\Ocal$ is Noetherian, there exists a maximal witness $\afrak$ contradicting the assertion.
  Note that $\afrak$ is not prime, and thus we may find $x,y \in \Ocal$ such that $x \cdot y \in \afrak$ but $x,y \notin \afrak$.
  The ideals $\afrak + (x)$ and $\afrak + (y)$ are strictly larger than $\afrak$, so the maximality of $\afrak$ ensures that we can find a product of primes $\pfrak_{1} \cdots \pfrak_{k}$ contained in $\afrak + (x)$ and a product of primes $\qfrak_{1} \cdots \qfrak_{s}$ contained in $\afrak + (y)$.
  The product $\pfrak_{1} \cdots \pfrak_{k} \cdot \qfrak_{1} \cdots \qfrak_{s}$ is then contained in $\afrak$.
  This product of ideals is generated by elements of the form
  \[\prod_{i=1^k}(p_i+r_ix)\cdot\prod_{i=1}^s(q_i+t_iy)\]
  With $p_i, q_i \in \afrak$, $r_i, t_i \in \Ocal$. 
  Expanding, this product is observed to be of the form $a+cxy$ with $a \in \afrak$, $c \in \Ocal$.
  This element is thus in $\afrak$, as $xy \in \afrak$.
\end{proof}

\begin{definition}
  For a fractional ideal $\afrak$ of $\Ocal$, we write
  \[ \afrak^{-1} := \{x \in K \ | \ x \cdot \afrak \subset \Ocal\}. \]
  This is again a fractional ideal: it's clear that this is an $\Ocal$ module, so we must show it is finitely generated.
  Let $x \in \afrak^{-1}$, and fix some element $\frac pq$ in $\afrak$.
  Observe that $x\frac pq=r$ for some $r\in \Ocal$, thus $x=r\frac qp$, so $x \in \frac 1p \Ocal$, which implies $\afrak^{-1}\subseteq \frac 1q \Ocal$.
  Thus $\afrak^{-1}$ is a submodule of a finitely generated module over a Noetherian domain, so is finitely generated.
  Also note that it is nonzero as it contains say, the product of the denominators of generators of $\afrak$. 
\end{definition}

\begin{lemma}
  Let $\pfrak$ be a prime ideal of $\Ocal$.
  then $\afrak \cdot \pfrak^{-1} \neq \afrak$ for every nonzero ideal $\afrak$ of $\Ocal$.
\end{lemma}
\begin{proof}
  We may assume that $\pfrak$ is nonzero, as if not then $\afrak \pfrak =0 \neq \afrak$ for any nonzero $\afrak$.
  Let $a \in \pfrak$ be a nonzero element.
  Find a product of nonzero primes $\pfrak_{1} \cdots \pfrak_{k}$ contained in $(a)$, with $k$ minimal.
  One of the $\pfrak_{i}$ is contained in $\pfrak$, since $\pfrak$ is prime.
  If not, for each $i$ find $x_i \in \pfrak_{i}\setminus \pfrak$.
  Then $\prod x_i \in \pfrak$, so by primality one of the $x_i$ is as well.
  Rearranging, we can assume $i = 1$, and thus $\pfrak = \pfrak_{1}$ since $\pfrak_{1}$ is maximal.
  Since $k$ is minimal, we can find $b \in \pfrak_{2} \cdots \pfrak_{k}$ with $b \notin (a)$, hence $a^{-1}b \notin \Ocal$.
  But we do have $b \cdot \pfrak \subset (a)$ hence $a^{-1} b \pfrak \subset \Ocal$ hence $a^{-1}b \in \pfrak^{-1}$.
  It follows that $\pfrak^{-1} \neq \Ocal$.

  Let $\afrak$ be an ideal, and $x_{1},\ldots,x_{n}$ generators of $\afrak$.
  Assume that $\afrak \pfrak^{-1} = \afrak$.
  Then for every $x \in \pfrak^{-1}$, we have
  \[ x x_{i} = \sum_{j} a_{ij} x_{j}, \ \ a_{ij} \in \Ocal. \]
  Consider the matrix $M := xI - (a_{ij})$ and observe that
  \[ M (x_{1},\ldots,x_{n})^{t} = 0.  \]
  Thus $\det(M)$ annihilates $\afrak$, and since $\afrak$ is nonzero, we have $\det(M) = 0$.
  Thus $x$ is integral over $\Ocal$, hence $x \in \Ocal$.
  It follows that $\pfrak^{-1} = \Ocal$, contradicting what we did above.
\end{proof}

\begin{proof}[Proof of Theorem~\ref{theorem:ideal_factorization}]
  First we show the existence of a factorization.
  Suppose that $\afrak$ is a maximal counterexample to this existence assertion.
  This $\afrak$ is not the unit ideal, since the unit ideal is the empty product of primes.
  Thus $\afrak$ is contained in a maximal ideal, say $\pfrak$.
  The inclusion $\Ocal \subset \pfrak^{-1}$ gives
  \[ \afrak \subset \afrak \pfrak^{-1} \subset \pfrak \pfrak^{-1} \subset \Ocal. \]
  By the above, we have $\afrak \neq \afrak \pfrak^{-1}$, and $\pfrak \neq \pfrak \pfrak^{-1} \subset \Ocal$.
  Since $\pfrak$ is maximal, we find $\pfrak \pfrak^{-1} = \Ocal$.

  The maximality of $\afrak$ shows that the ideal $\afrak \pfrak^{-1}$ admits a factorization as $\pfrak_{1} \cdots \pfrak_{k}$.
  We know $\afrak \pfrak^{-1}$ is an ideal as $\afrak \subset \pfrak$, and so $\afrak \pfrak^{-1}\subset \pfrak \pfrak^{-1} \subset \Ocal.$
  Multiplying by $\pfrak$ gives
  \[ \afrak = \pfrak \pfrak_{1} \cdots \pfrak_{k}, \]
  which contradicts the assumption on $\afrak$.

  Now for uniqueness.
  Suppose that
  \[ \afrak = \pfrak_{1} \cdots \pfrak_{k} = \qfrak_{1} \cdots \qfrak_{s}. \]
  Then $\pfrak_{1}$ divides one of the $\qfrak_{i}$, say $\qfrak_{1}$.
  Since both are maximal, they are equal.
  Multiply by the inverse of $\pfrak_{1} = \qfrak_{1}$ to decrease $k$ and $s$ by $1$, and repeat.
\end{proof}

Next, we start our discussion for Theorem~\ref{theorem:fractional_ideal_factorization}.

\begin{proposition}
  The set $J_{K} = J_{K}(\Ocal)$ of fractional $\Ocal$ ideals is an abelian group with respect to the product of fractional ideals and inverses as defined above.
\end{proposition}
\begin{proof}
  The operation is easily seen to be associative, commutative, and $(1)$ acts as a unit.

  If $\afrak$ is an integral ideal, we have
  \[ \afrak = \pfrak_{1} \cdots \pfrak_{k} \]
  where $\pfrak_{i}$ are primes, and we saw before that $\pfrak_{i} \cdot \pfrak_{i}^{-1} = \Ocal = (1)$.
  It follows that
  \[ \bfrak = \pfrak_{1}^{-1} \cdots \pfrak_{k}^{-1} \]
  is an inverse of $\afrak$.
  Noting that $\bfrak \afrak = \Ocal$ implies that $\bfrak \subset \afrak^{-1}$.
  Also, if $x \in \afrak^{-1}$, then $x \afrak \bfrak \subset \bfrak$ so $x \in \bfrak$, hence $\bfrak = \afrak^{-1}$.

  Finally, if $\afrak$ is a fractional ideal, we can find some $c \in \Ocal$ such that $c \cdot \afrak \subset \Ocal$ is integral, hence $(c \afrak)^{-1} = c^{-1} \afrak^{-1}$ is the inverse of $c \afrak$, hence $\afrak \afrak^{-1} = \Ocal$, as requried.
\end{proof}

\begin{proof}[Proof of Theorem~\ref{theorem:fractional_ideal_factorization}]
  We can write any fractional ideal as a quotient $\afrak / \bfrak$ of two integral ideals (in fact, we can arrange for $\bfrak$ to be principal!).
  The claim thus reduces to that of Theorem~\ref{theorem:ideal_factorization}.
\end{proof}

Let us now discuss the behaviour of these groups with respect to localization.

\begin{proposition}
  Let $S$ be a multiplicative subset of $\Ocal$, not containing zero.
  The map $J_{K}(\Ocal) \to J_{K}(S^{-1}\Ocal)$ given by
  \[ \afrak \mapsto S^{-1} \afrak \]
  is a surjective morphism of groups whose kernel generated by prime ideals $\pfrak$ which meet $S$.
\end{proposition}
\begin{proof}
  The product of fractional ideals is compatible with localization:
  \[ S^{-1}(\afrak \bfrak) = S^{-1}\afrak \cdot S^{-1} \bfrak. \]
  For surjectivity, simply write a fractional ideal as a quotient of two integral ideals to reduce to the integral case, and use the fact that the inflation of the contraction of an ideal is the ideal itself.

  Now concerning the kernel, recall that the primes of $S^{-1}\Ocal$ correspond, via the map in question, to primes of $\Ocal$ which do not meet $S$.
  Also, if $\pfrak$ is a prime which does meet $S$, then its inflation in $S^{-1}\Ocal$ is the unit ideal, and is thus contained in the kernel of the map above.
  The assertion concerning the kernel follows from this, along with Theorem~\ref{theorem:fractional_ideal_factorization}, as follows.
  Suppose that $\afrak$ is in the kernel.
  Write
  \[ \afrak = \prod_{\pfrak} \pfrak^{v_{\pfrak}}. \]
  Then $S^{-1}\afrak$ has a factorization
  \[ S^{-1}\afrak = \prod_{\pfrak, \ \pfrak \cap S = \varnothing} S^{-1}\pfrak^{v_{\pfrak}}. \]
  This is the unit ideal trivial by assumption, and uniqueness of factorizations shows that $v_{\pfrak} = 0$ for all $\pfrak$ such that $\pfrak \cap S = \varnothing$.
  This shows that the support of $\afrak$ (i.e.~the primes $\pfrak$ for which $v_{\pfrak} \neq 0$), is concentraded in the set of primes which meet $S$.
  The claim follows.
\end{proof}

\begin{proposition}[Killing the class group]
  Suppose that $Cl_{K}(\Ocal)$ is finite, and let $\afrak_{1},\ldots,\afrak_{k}$ be integral ideals which represent all the ideal classes.
  Suppose that $S$ is a multiplicative set not containing zero.
  Assume that there exists some $b \in S$ such that $b \in \cap_{i} \afrak_{i}$.
  Then the class group of $S^{-1}\Ocal$ is trivial, hence $S^{-1}\Ocal$ is a PID.
\end{proposition}
\begin{proof}
  \todo{The vanishing of the class group follows from the previous proposition.}{Add proof.}
  Also, if the class group of $\Ocal$ is trivial, then every (integral) ideal is principal, and since $\Ocal$ is a domain, it follows that $\Ocal$ is a PID.
\end{proof}

%%% Local Variables:
%%% mode: latex
%%% TeX-master: "main"
%%% End:
