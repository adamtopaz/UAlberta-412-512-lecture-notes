\section{Extra Lecture 1}

In these extra lectures, I will discuss a proof of the \emph{Neukirch-Uchida theorem}.
First, let's set up the general context.
Suppose that $K_{i}$, $i = 1,2$ are fields with fixed algebraic closures $\bar K_{i}$.
We write
\[ \Isom(\bar K_{1}|K_{1},\bar K_{2}|K_{2}) \]
for the set of isomorphisms $\sigma : \bar K_{1} \cong \bar K_{2}$ which satisfy $\sigma K_{1} = K_{2}$.
Any such isomorphism $\sigma$ induces an isomorphism of Galois groups $\Gal(\bar K_{2}|K_{2}) \cong \Gal(\bar K_{1}|K_{1})$ by conjugation.
Namely, if $\tau \in \Gal(\bar K_{2}|K_{2})$ is some element, then $\sigma^{*} \tau := \sigma^{-1} \circ \tau \circ \sigma \in \Gal(\bar K_{1}|K_{1})$.
This gives us a natural map
\[ \Isom(\bar K_{1}|K_{1},\bar K_{2}|K_{2}) \to \Isom(\Gal(\bar K_{2}|K_{2}),\Gal(\bar K_{1}|K_{1})). \]

\begin{theorem}[Neukirch-Uchida]\label{theorem:NU}
  In the above context, suppose that $K_{1}$ and $K_{2}$ are two number fields.
  Then the map
  \[ \Isom(\bar K_{1}|K_{1},\bar K_{2}|K_{2}) \to \Isom(\Gal(\bar K_{2}|K_{2}),\Gal(\bar K_{1}|K_{1})). \]
  is a bijection.
\end{theorem}

In particular, this shows that $K_{1} \cong K_{2}$ if and only if $\Gal_{K_{1}} \cong \Gal_{K_{2}}$.
One can formulate a functorial statement using outer morphisms, as follows:
\begin{theorem}[Neukirch-Uchida, variant]
  In the above context, suppose that $K_{1}$ and $K_{2}$ are two number fields.
  Let $\Gal_{K_{i}}$ denote the absolute Galois group of $K_{i}$, computed with respect to some choice of algebraic closure.
  Then the natural map
  \[ \Isom(K_{1},K_{2}) \to \Isom(\Gal_{K_{2}},\Gal_{K_{1}})/\Inn(\Gal_{K_{1}}) \]
  is a bijection.
  Here the inner automorphism group of $\Gal_{K_{1}}$ acts on the set $\Isom(\Gal_{K_{2}},\Gal_{K_{1}})$ by composition.
\end{theorem}
\begin{proof}
  This follows easily from Theorem~\ref{theorem:NU} and basic Galois theory.
  Details left as an exercise.
\end{proof}

\begin{remark}
  This theorem was the start of the subject of \emph{anabelian geometry}.
  The term ``anabelian'' means ``beyond abelian'' and was coined by Grothendieck in his famous letter to Faltings (the NU theorem came before this letter!).
  In this letter, he outlines the ``anabelian conjectures'', one of which is a generalization of the NU theorem replacing number fields by infinite finitely generated fields.
  This generalization was eventually proved by Pop and separately my Mochizuki (who also proved several more of Grothendieck's anabelian conjectures).
\end{remark}

The proof involves two main steps: A \emph{local theory} and a \emph{global theory}.
I'll split up the discussion of this proof into two lectures, one on the local theory and one on the global theory.
In this lecture, we'll discuss the local theory.

Let's recal some notation from our course.
Let $K$ be a number field and $\pfrak$ a prime of $K$.
Choose some prolongation $\bar \pfrak$ of $\pfrak$ to the algebraic closure (equivalently, choose a prolongation of the absolute value $|-|_{\pfrak}$ to $\bar K$).
The \emph{decomposition group} $Z_{\bar\pfrak|\pfrak}$ of $\bar\pfrak|\pfrak$ is the stabilizer of $\bar \pfrak$ with respect to the action of $\Gal_{K} = \Gal(\bar K|K)$ on the (profinite) set of prolongations of $\pfrak$.
We also saw that $Z_{\bar\pfrak|\pfrak}$ is isomorphic to $\Gal_{K_{\pfrak}}$, the absolute Galois group of the completion of $K$ with respect to $|-|_{\pfrak}$.
The conjugacy class of $Z_{\bar\pfrak|\pfrak}$ only depends on $\pfrak$, and we write $Z_{\pfrak}$ for this conjugacy class.

\begin{lemma}
  Suppose that $\pfrak_{i}$, $i = 1,2$ are two primes of $K$.
  Then $Z_{\pfrak_{1}} = Z_{\pfrak_{2}}$ if and only if $\pfrak_{1} = \pfrak_{2}$.
\end{lemma}
\begin{proof}
  Exercise.
  This can be done in several ways (for example, using the approximation theorem for independent valuations, or using the main statements of CFT, etc.).
\end{proof}

By the above, we see that in order to characterize the set of primes of $K$, it suffices to characterize the collection of all decomposition groups in $\Gal_{K}$.
Furthermore, if $Z_{\pfrak}$ is a conjugacy class of such a decomposition group as above, we can obtain several invariants about $\pfrak$ using the fact that $Z_{\pfrak} \cong \Gal_{K_{\pfrak}}$, essentially as a consequence of local class field theory.
These invariants are:
\begin{enumerate}
  \item The residue characteristic $p$.
  \item The absolute ramification degree $e$ of $\pfrak|p$.
  \item The absolute inertia degree $f$ of $\pfrak|p$.
\end{enumerate}
Using (2), we can also determine the inertia group $T_{\bar\pfrak|\pfrak}$ in $Z_{\bar\pfrak|\pfrak}$ and thus also the conjugacy class of inertia groups $T_{\pfrak}$ as well.

If we have an isomorphism $\Gal_{K_{2}} \cong \Gal_{K_{1}}$, the first goal is thus to show that this isomorphism respect decomposition groups, and this is precisely what the local theory accomplishes.


%%% Local Variables:
%%% mode: latex
%%% TeX-master: "main"
%%% End:
