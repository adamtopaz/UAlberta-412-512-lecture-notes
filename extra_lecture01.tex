\section{Extra Lecture 1}

In these extra lectures, I will discuss a proof of the \emph{Neukirch-Uchida theorem}.
First, let's set up the general context.
Suppose that $K_{i}$, $i = 1,2$ are fields with fixed algebraic closures $\bar K_{i}$.
We write
\[ \Isom(\bar K_{1}|K_{1},\bar K_{2}|K_{2}) \]
for the set of isomorphisms $\sigma : \bar K_{1} \cong \bar K_{2}$ which satisfy $\sigma K_{1} = K_{2}$.
Any such isomorphism $\sigma$ induces an isomorphism of Galois groups $\Gal(\bar K_{2}|K_{2}) \cong \Gal(\bar K_{1}|K_{1})$ by conjugation.
Namely, if $\tau \in \Gal(\bar K_{2}|K_{2})$ is some element, then $\sigma^{*} \tau := \sigma^{-1} \circ \tau \circ \sigma \in \Gal(\bar K_{1}|K_{1})$.
This gives us a natural map
\[ \Isom(\bar K_{1}|K_{1},\bar K_{2}|K_{2}) \to \Isom(\Gal(\bar K_{2}|K_{2}),\Gal(\bar K_{1}|K_{1})). \]

\begin{theorem}[Neukirch-Uchida]\label{theorem:NU}
  In the above context, suppose that $K_{1}$ and $K_{2}$ are two number fields.
  Then the map
  \[ \Isom(\bar K_{1}|K_{1},\bar K_{2}|K_{2}) \to \Isom(\Gal(\bar K_{2}|K_{2}),\Gal(\bar K_{1}|K_{1})). \]
  is a bijection.
\end{theorem}

In particular, this shows that $K_{1} \cong K_{2}$ if and only if $\Gal_{K_{1}} \cong \Gal_{K_{2}}$.
One can formulate a functorial statement using outer morphisms, as follows:
\begin{theorem}[Neukirch-Uchida, variant]
  In the above context, suppose that $K_{1}$ and $K_{2}$ are two number fields.
  Let $\Gal_{K_{i}}$ denote the absolute Galois group of $K_{i}$, computed with respect to some choice of algebraic closure.
  Then the natural map
  \[ \Isom(K_{1},K_{2}) \to \Isom(\Gal_{K_{2}},\Gal_{K_{1}})/\Inn(\Gal_{K_{1}}) \]
  is a bijection.
  Here the inner automorphism group of $\Gal_{K_{1}}$ acts on the set $\Isom(\Gal_{K_{2}},\Gal_{K_{1}})$ by composition.
\end{theorem}
\begin{proof}
  This follows easily from Theorem~\ref{theorem:NU} and basic Galois theory.
  Details left as an exercise.
\end{proof}

\begin{remark}
  This theorem was the start of the subject of \emph{anabelian geometry}.
  The term ``anabelian'' means ``beyond abelian'' and was coined by Grothendieck in his famous letter to Faltings (the NU theorem came before this letter!).
  In this letter, he outlines the ``anabelian conjectures'', one of which is a generalization of the NU theorem replacing number fields by infinite finitely generated fields.
  This generalization was eventually proved by Pop and separately my Mochizuki (who also proved several more of Grothendieck's anabelian conjectures).
\end{remark}

The proof involves two main steps: A \emph{local theory} and a \emph{global theory}.
I'll split up the discussion of this proof into two lectures, one on the local theory and one on the global theory.
In this lecture, we'll discuss the local theory.

Let's recal some notation from our course.
Let $K$ be a number field and $\pfrak$ a prime of $K$.
Choose some prolongation $\bar \pfrak$ of $\pfrak$ to the algebraic closure (equivalently, choose a prolongation of the absolute value $|-|_{\pfrak}$ to $\bar K$).
The \emph{decomposition group} $Z_{\bar\pfrak|\pfrak}$ of $\bar\pfrak|\pfrak$ is the stabilizer of $\bar \pfrak$ with respect to the action of $\Gal_{K} = \Gal(\bar K|K)$ on the (profinite) set of prolongations of $\pfrak$.
We also saw that $Z_{\bar\pfrak|\pfrak}$ is isomorphic to $\Gal_{K_{\pfrak}}$, the absolute Galois group of the completion of $K$ with respect to $|-|_{\pfrak}$.
The conjugacy class of $Z_{\bar\pfrak|\pfrak}$ only depends on $\pfrak$, and we write $Z_{\pfrak}$ for this conjugacy class.

\begin{lemma}
  Suppose that $\pfrak_{i}$, $i = 1,2$ are two primes of $K$.
  Then $Z_{\pfrak_{1}} = Z_{\pfrak_{2}}$ if and only if $\pfrak_{1} = \pfrak_{2}$.
\end{lemma}
\begin{proof}
  Exercise.
  This can be done in several ways (for example, using the approximation theorem for independent valuations, or using the main statements of CFT, etc.).
\end{proof}

By the above, we see that in order to characterize the set of primes of $K$, it suffices to characterize the collection of all decomposition groups in $\Gal_{K}$.
Furthermore, if $Z_{\pfrak}$ is a conjugacy class of such a decomposition group as above, we can obtain several invariants about $\pfrak$ using the fact that $Z_{\pfrak} \cong \Gal_{K_{\pfrak}}$, essentially as a consequence of local class field theory.
These invariants are:
\begin{enumerate}
  \item The residue characteristic $p$.
  \item The absolute ramification degree $e$ of $\pfrak|p$.
  \item The absolute inertia degree $f$ of $\pfrak|p$.
\end{enumerate}
Using (2), we can also determine the inertia group $T_{\bar\pfrak|\pfrak}$ in $Z_{\bar\pfrak|\pfrak}$ and thus also the conjugacy class of inertia groups $T_{\pfrak}$ as well.

If we have an isomorphism $\Gal_{K_{2}} \cong \Gal_{K_{1}}$, the first goal is thus to show that this isomorphism respect decomposition groups, and this is precisely what the local theory accomplishes.
This is what we will show in this lecture.

The argument I'll explain is different from Neukirch's original proof (which relies on local-to-global principles from global class field theory).
Instead, I'll explain a proof which is more elementary and works in greater generality.
The key theorem is the following.
\begin{theorem}\label{theorem:localthy}
  Let $\ell$ be an odd prime and $K$ any field.
  Let $f,g : K^{\times} \to \Zbb/\ell$ be two homomorphisms.
  The following are equivalent:
  \begin{enumerate}
    \item For all $x,y \in K^{\times}$ such that $x + y = 1$, one has $f(x)g(y) = f(y)g(x)$.
    \item There exists a valuation $v$ of $K$ such that $f,g$ are both trivial on $1 + \mfrak_{v}$ and some nontrivial $\Zbb/\ell$-linear combination of $f,g$ is trivial on $U_{v}$.
  \end{enumerate}
\end{theorem}
\begin{proof}
  This uses the theory of \emph{rigid elements}, as developed by Ware, Arason-Elman-Jacob and others.
  I'll sketch the argument in the recording for this lecture.
  In fact, this proof has now been formalized in Lean3, see \href{https://github.com/adamtopaz/lean-acl-pairs}{this github repository}.
\end{proof}

Now suppose that $K$ is a number field which contains $\mu_{\ell}$, and let $v$ be a valuation of $K$.
By Ostrowski's theorem, this valuation is $v = v_{\pfrak}$ for some prime $\pfrak$ of $K$.
Assume that $\pfrak$ does \emph{not} lie over $\ell$ (so the residue characteristic of $v$ is different from $\ell$).
Note that $\Hom(K^{\times}/(1+\mfrak_{v}),\mu_{\ell}) \cong \Hom(K^{\times}/(1+\mfrak_{v}),\Zbb/\ell)$ is $2$-dimensional over $\Zbb/\ell$.
Indeed, to see this use the (split) exact sequence
\[ 1 \to \kappa(v)^{\times} \to K^{\times}/(1+\mfrak_{v}) \xrightarrow{v} \Zbb \to 1 \]
and take $\Zbb/\ell$-duals.
Then use the fact that $\kappa(v)^{\times}$ is cyclic along with the fact that $\mu_{\ell} \subset \kappa(v)$ (why?) and that $\kappa(v)$ has characteristic different from $\ell$.

On the other hand, if $\kappa(v)$ has residue characteristic $\ell$ then $\ell$ is coprime to the size of $\kappa(v)^{\times}$ and thus $\Hom(K^{\times}/(1+\mfrak_{v}),\Zbb/\ell) \cong \Zbb/\ell$ using a similar argument.

Let's also observe that if $v$ and $w$ are two nonequivalent valuations of $K$, then
\[ \Hom(K^{\times}/(1 + \mfrak_{v}),\Zbb/\ell) \cap \Hom(K^{\times}/(1+\mfrak_{w}),\Zbb/\ell)) = 0 \]
where the intersection is taken inside of $\Hom(K^{\times},\Zbb/\ell)$.
Indeed, this is equivalent to the assertion that $K^{\times\ell} \cdot (1+\mfrak_{v}) \cdot (1+\mfrak_{v}) = K^{\times}$, which follows from the approximation theorem for independent valuations (which $v$ and $w$ are).

Now let's see what we can do with this observations.
For any $K$ and $\ell$ as above, consider the set $S_{K,\ell}$ consisting of subgroups of $\Gal_{K}^{\ab}/\ell$ of the form $\langle \sigma,\tau \rangle$ which satisfy the following conditions:
\begin{enumerate}
  \item First, $\langle \sigma,\tau \rangle \cong (\Zbb/\ell)^{2}$.
  \item For all $x,y \in \HH^{1}(K,\Zbb/\ell)$ such that $x \cup y = 0$ in $\HH^{2}(K,\Zbb/\ell)$, one has $\sigma(x)\tau(y) = \sigma(y) \tau(x)$.
\end{enumerate}
In item (2), we are identifying
\[ \HH^{1}(K,\Zbb/\ell) = \Hom(\Gal_{K},\Zbb/\ell) = \Hom(\Gal_{K}^{\ab}/\ell,\Zbb/\ell). \]
Condition (2) turns out to be \emph{equivalent} to the \emph{condition (1)} from Theorem~\ref{theorem:localthy}, as a consequence of the \emph{Merkurjev-Suslin Theorem} (I will not discuss the details here).
The key observation to make here is that condition (2) can be computed using $\Gal_{K}$ as a profinite group while Theorem~\ref{theorem:localthy} ensures the existence of a valuation $v$ such that $\sigma,\tau$ are trivial on $1 + \mfrak_{v}$.
By the discussion above, we know that the elements of $S_{K,\ell}$ are precisely the subgroups of the form $\Hom(K^{\times}/(1+\mfrak_{v}),\Zbb/\ell)$ where $v$ is a valuation whose residue characteristic is prime to $\ell$.

Now suppose that $L|K$ is a Galois extension of number fields with $\mu_{\ell} \subset L$ (but maybe not contained in $L$).
The Galois group $\Gal(L|K)$ acts naturally on $S_{L,\ell}$ by conjugation.
If $v$ is a valuation of $L$ whose residue characteristic is  different from $\ell$, then $\Hom(K^{\times}/(1+\mfrak_{v},\Zbb/\ell)$ is an element of $S_{L,\ell}$, and an element $\sigma \in \Gal(L|K)$ sends it to $\Hom(K^{\times}/(1+\mfrak_{\sigma^{-1}v},\Zbb/\ell))$.
This can only agree with $\Hom(K^{\times}/(1+\mfrak_{v},\Zbb/\ell)$ if $\sigma$ fixes $v$ itself, by the discussion above using the approximation theorem.
Thus, the stabilizer of $\Hom(K^{\times}/(1+\mfrak_{v},\Zbb/\ell)$ in $\Gal(L|K)$ is the decomposition group of $v$ over its restriction of $K$.
Let's write $T_{L|K,\ell}$ for the collection of stabilizers in $\Gal(L|K)$ of the elements of $S_{L,\ell}$.

We deduce the following theorem.
\begin{theorem}
  Let $K$ be any number field and $\ell_{i}$, $i = 1,2$ two distinct primes.
  Then for all sufficiently large finite Galois extensions $L|K$, the following hold.
  Let $H$ be a subgroup of $\Gal(L|K)$.
  The following are equivalent:
  \begin{enumerate}
    \item There exists valuations $w|v$ of $L|K$ such that $H = Z_{w|v}$.
    \item $H \in T_{L|K,\ell_{i}}$ for at least one $i$.
  \end{enumerate}
\end{theorem}

This theorem characterizes the decomposition groups of $\Gal(L|K)$ whenever $L$ is sufficiently large (more precisely, we just need $\mu_{\ell_{1}\ell_{2}} \subset L$).
Now the collection of such $L$ is cofinal in the full system of finite extensions of $K$ (contained in a fixed $\bar K$).
Since decomposition groups are compatible in projections of Galois groups of towers, we can take limits along such $L$ to obtain a characterization of the \emph{absolute} decomposition groups in $\Gal(\bar K|K)$.
To summarize, we obtain the following result.
\begin{theorem}\label{theorem:mainlocalthy}
  Let $K_{i}$, $i = 1,2$ be two number fields, and let $\phi : \Gal_{K_{1}} \cong \Gal_{K_{2}}$ be an isomorphism of absolute Galois groups.
  Let $\bar v|v$ be some prolongation of valuations in $\bar K_{1}|K_{1}$.
  Then there exist a unique extension of valuations $\bar w|w$ of $\bar K_{2}|K_{2}$ such that $\phi Z_{\bar v|v} = Z_{\bar w|w}$.
\end{theorem}

This is the main result from the so-called \emph{local theory}.
In the next lecture, I'll explain how this can be used in conjunction with Chebotarev's density theorem to show that $K_{1}\cong K_{2}$ under the assumptions of Theorem~\ref{theorem:mainlocalthy}.

%%% Local Variables:
%%% mode: latex
%%% TeX-master: "main"
%%% End:
