\section{Extra Lecture 1}

In these extra lectures, I will discuss a proof of the \emph{Neukirch-Uchida theorem}.
First, let's set up the general context.
Suppose that $K_{i}$, $i = 1,2$ are fields with fixed algebraic closures $\bar K_{i}$.
We write
\[ \Isom(\bar K_{1}|K_{1},\bar K_{2}|K_{2}) \]
for the set of isomorphisms $\sigma : \bar K_{1} \cong \bar K_{2}$ which satisfy $\sigma K_{1} = K_{2}$.
Any such isomorphism $\sigma$ induces an isomorphism of Galois groups $\Gal(\bar K_{2}|K_{2}) \cong \Gal(\bar K_{1}|K_{1})$ by conjugation.
Namely, if $\tau \in \Gal(\bar K_{2}|K_{2})$ is some element, then $\sigma^{*} \tau := \sigma^{-1} \circ \tau \circ \sigma \in \Gal(\bar K_{1}|K_{1})$.
This gives us a natural map
\[ \Isom(\bar K_{1}|K_{1},\bar K_{2}|K_{2}) \to \Isom(\Gal(\bar K_{2}|K_{2}),\Gal(\bar K_{1}|K_{1})). \]

\begin{theorem}[Neukirch-Uchida]\label{theorem:NU}
  In the above context, suppose that $K_{1}$ and $K_{2}$ are two number fields.
  Then the map
  \[ \Isom(\bar K_{1}|K_{1},\bar K_{2}|K_{2}) \to \Isom(\Gal(\bar K_{2}|K_{2}),\Gal(\bar K_{1}|K_{1})). \]
  is a bijection.
\end{theorem}

In particular, this shows that $K_{1} \cong K_{2}$ if and only if $\Gal_{K_{1}} \cong \Gal_{K_{2}}$.
One can formulate a functorial statement using outer morphisms, as follows:
\begin{theorem}[Neukirch-Uchida, variant]
  In the above context, suppose that $K_{1}$ and $K_{2}$ are two number fields.
  Let $\Gal_{K_{i}}$ denote the absolute Galois group of $K_{i}$, computed with respect to some choice of algebraic closure.
  Then the natural map
  \[ \Isom(K_{1},K_{2}) \to \Isom(\Gal_{K_{2}},\Gal_{K_{1}})/\Inn(\Gal_{K_{1}}) \]
  is a bijection.
  Here the inner automorphism group of $\Gal_{K_{1}}$ acts on the set $\Isom(\Gal_{K_{2}},\Gal_{K_{1}})$ by composition.
\end{theorem}
\begin{proof}
  This follows easily from Theorem~\ref{theorem:NU} and basic Galois theory.
  Details left as an exercise.
\end{proof}

\begin{remark}
  This theorem was the start of the subject of \emph{anabelian geometry}.
  The term ``anabelian'' means ``beyond abelian'' and was coined by Grothendieck in his famous letter to Faltings (the NU theorem came before this letter!).
  In this letter, he outlines the ``anabelian conjectures'', one of which is a generalization of the NU theorem replacing number fields by infinite finitely generated fields.
  This generalization was eventually proved by Pop and separately my Mochizuki (who also proved several more of Grothendieck's anabelian conjectures).
\end{remark}

The proof involves two main steps: A \emph{local theory} and a \emph{global theory}.
I'll split up the discussion of this proof into two lectures, one on the local theory and one on the global theory.
In this lecture, we'll discuss the local theory.

%%% Local Variables:
%%% mode: latex
%%% TeX-master: "main"
%%% End:
