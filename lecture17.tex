\section{Lecture 17}

We will now turn our attention from the \emph{global} setting to a \emph{local} setting.
Intuitively, what this means is to choose a compatible system of primes in a number field $K$ and all its extensions, with respect to divisibility.
This allows us to study things like the invariants $f(-|-)$ and $e(-|-)$ one prime at a time, at the expense of replacing Galois groups over number fields with decomposition groups.

There are two ways to accomplish this, each with its own benefits.
The first is to take a \emph{Henselization} (we won't say much about this in our course).
The other approach, which is what we will focus on in this course, is \emph{completion}.

Let me first discuss some of the general context.
Recall that a \emph{valuation ring} is a domain $A$ in which the divisibility order is total.
What this means explicitly is that for every $a,b \in A$, there exists a $c$ such that $a \cdot c = b$ or $b \cdot c = a$.
If we let $K$ denote the fraction field of $A$, then this is equivalent to the condition that for all $x \in K$, either $x \in A$ or $x^{-1} \in A$.

\begin{lemma}
  The spectrum of a valuation ring is totally ordered.
  In particular, any valuation ring is local.
\end{lemma}
\begin{proof}
  \todo{Start by reducing to the case of principal ideals, which follows from the definition.}{Add details.}
  In fact it holds that the ideals of a valuation ring are totally ordered.
  Given $I,J$ ideals such that $I \not\subseteq J$, then we have $a \in I$ such that $a \notin J$.
  But the divisibility order is total in a valuation ring. Thus we see that $a$ divdes every element of $J$. 
  So we have that $J \subset I$.

  Thus the inclusion order on ideals is total, and in particular the spectrum is totally ordered.
\end{proof}

Also recall that a valuation on a field $K$ is a surjective homomorphism
\[ v : K^{\times} \to \Gamma \]
to a totally ordered abelian group which satisfies the following \emph{ultrametric} condition:
\[ \forall x,y \in K, \ v(x+y) \geq \min(v(x),v(y)). \]
Here we abuse the notation and extend $v$ to all of $K$ by setting $v(0) = \infty$.
Two valuations $v_{1},v_{2}$ with value groups $\Gamma_{1},\Gamma_{2}$ are called \emph{equivalent} provided that there exists some isomorphism $e : \Gamma_{1} \cong \Gamma_{2}$ making \todo{the obvious diagram commute.}{Use tikz to typeset this diagram.}

To any valuation ring $A$ we can associate a \emph{valuation} $v$ on the fraction field $K$, as follows.
The value group $\Gamma$ is $K^{\times}/A^{\times}$, and the map $v$ is just the quotient map.
The condition on $A$ ensures that this is indeed a valuation.
Conversely, to any valuation $v$ we can associate a valuation subring $A$ of $K$ (meaning that $A$ is a valuation ring with fraction field $K$) by setting $A := v^{-1}[0,\infty]$.
Note that the unit group of $A$ is precisely the kernel of $v$.

\begin{lemma}
  The two operations above provide a one-to-one correspondence between valuation rings with fraction field $K$ and valuations of $K$ considered up-to equivalence.
\end{lemma}
\begin{proof}
  \todo{Exercise.}{Exercise.}
\end{proof}

The \emph{residue field} of a valuation $v$ (or equivalently, its associated valuation ring) is the residue field of the associated valuation ring $A$ (recall that this is a local ring).

One can then develop much of the ramification theory that we have discussed so far in this more general context of valued fields.
We will not do this.
Instead, we will soon restrict our attention to the valuations arising from number theory.

First, let's say that a valuation $v$ has \emph{rank $1$} provided that the value group $\Gamma$ embeds in $\Rbb$ (as totally ordered abelian groups).
When we have a rank $1$ valuation $v$, we can obtain an associated absolute value on $K$ by setting
\[ |x|_{v} := e^{-v(x)}. \]
Note that the $e$ is really irrelevant here, as we are choosing some embedding $\iota : \Gamma \to \Rbb$, which we can always scale by some positive real number.
The resulting absolute values are equivalent (in the sense of absolute values).

Let $K$ be a number field, and $\pfrak$ a prime ideal of $\Ocal_{K}$.
Recall that the localization $\Ocal := \Ocal_{K,\pfrak}$ is a local PID, hence a DVR.
What this means explicitly is that $\Ocal$ is a local ring whose maximal ideal $\mfrak := \pfrak_{\pfrak}$ is principal, say generated by $\pi$ (we call such a $\pi$ a \emph{uniformizer}), and every nonzero element of $\Ocal$ has a unique representation of the form $u \cdot \pi^{n}$ for some $u \in \Ocal^{\times}$ and some $n \in \Nbb$.
Any element of $K$ can thus be represented as $u \cdot \pi^{n}$ for some $u \in \Ocal^{\times}$ and some $n \in \Zbb$, so it follows that $\Ocal$ is a valuation ring in the above sense.
The value group in this case is isomorphic to $\Zbb$ as a totally ordered abelian group, and the associated valuation $v_{\pfrak}$ on the number field $K$ is the one which satisfies
\[ v_{\pfrak}(u \cdot \pi^{n}) = n, \]
where $u \in \Ocal^{\times}$ and $n \in \Zbb$.
In this particular case, we \emph{normalize} the corresponding absolute value as follows.
First, embed $\Zbb$ into $\Rbb$ in the usual way ($1 \mapsto 1$).
We then take
\[ \|x\|_{\pfrak} := N(\pfrak)^{-v_{\pfrak}(x)}, \]
where $N$ is the absolute norm, so that $N(\pfrak)$ is the size of the residue field $\kappa(\pfrak)$.
What this means explicitly is that
\[ \|u \cdot \pi^{n}\|_{\pfrak} = N(\pfrak)^{-n}. \]
We also make the following choice of normalization (following Lang):
\[ |p|_{\pfrak} = 1/p, \]
where $p$ is the rational prime which is divisible by $\pfrak$.
This amounts to $|\pi|_{\pfrak} = 1/(p^{1/e})$ where $e$ is the \emph{absolute} ramification degree ($e = e(\pfrak|p)$).
We will see why we choose these particular normalizations later on (the product formula).

Whenever we have an absolute value on a field $K$, say $|-|$, we can \emph{complete} $K$ with respect to the absolute value, in the usual way (Cauchy sequences modulo nullsequences).
We obtain a resulting field $\hat K$.
In the number field case where $|-| = |-|_{\pfrak}$ for a prime, we shall write $K_{\pfrak}$ for this completion.
The absolute value $|-|_{\pfrak}$ extends to $K_{\pfrak}$ in the \todo{natural way by taking the limit of the absolute values of some Cauchy sequence which converges to the given element.}{Explain}
In this case, the absolute value $|-|_{\pfrak}$ on $K_{\pfrak}$ also comes from a discrete valuation of rank $1$ which we will also denote by $v_{\pfrak}$.
The valuation ring $\Ocal_{\pfrak}$ of $K_{\pfrak}$ with respect to this valuation will be called the \emph{ring of integers} of $K_{\pfrak}$.
One may check that $\Ocal_{\pfrak}$ is the completion (either adic, or in terms of Cauchy sequences) of $\Ocal_{K,\pfrak}$, and that $K_{\pfrak}$ is its fraction field.

In general, a number field $K$ has many absolute values.
We have the ones arising from (nonzero) prime ideals of $\Ocal_{K}$, as discussed above, but we also have absolute values of $K$ arising from any real or complex embedding.
If $\sigma : K \to \Rbb$ is an embedding, then $|-|_{\sigma} := |\sigma(-)|$ is an absolute value, and similarly if we replace $\Rbb$ with $\Cbb$.
We will write $M_{K}$ for the collection of absolute values of $K$ of the form $|-|_{\pfrak}$ or $|-|_{\sigma}$.

If $L|K$ is a finite extension and $v \in M_{K}$ then any absolute value of $L$ extending $v$ lies in $M_{L}$ (this was the reason for our normalization of $|-|_{\pfrak}$ above).
Note that for nonarchmedean primes $v$ of $M_{K}$, we have $v|p$ for some rational prime $p \in M_{\Qbb}$.
Also,
\[ M_{\Qbb} = MaxSpec \Zbb \cup \{\infty\} \]
and the archemedean primes $v$ of $K$ satisfy $v|\infty$.

One may check that any two elements of $M_{K}$ are \emph{independent} as absolute values, meaning that they induce distinct topologies on $K$.

%%% Local Variables:
%%% mode: latex
%%% TeX-master: "main"
%%% End:
