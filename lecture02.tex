\section{Lecture 2}

In this course, a \emph{ring} will always be \emph{commutative} with $1$.
If $A$ is a ring, then an $A$-algebra will be a ring endowed with a \emph{structure morphism} $A \to B$, which will almost always be left implicit.
A morphism $B_{1} \to B_{2}$ of $A$-algebras is a morphism of rings which is compatible with the structure morphisms in the obvious way.

Recall that $A[X_{1},\ldots,X_{n}]$, the polynomial ring in $n$ variables, is the free $A$-algebra on the set $\{X_{1},\ldots,X_{n}\}$.
Thus, if $B$ is an $A$-algebra and $b_{1},\ldots,b_{n} \in B$ are given, then there is a unique morphism of $A$-algebras
\[ A[X_{1},\ldots,X_{n}] \to B \]
satisfying $X_{i} \mapsto b_{i}$.
This map is called \emph{evaluation} at $b_{1},\ldots,b_{n}$, and will be denoted in the usual way as
\[ f(X_{1},\ldots,X_{n}) \mapsto f(b_{1},\ldots,b_{n}). \]


\begin{definition}
  Let $A$ be a ring and $B$ an $A$-algebra.
  An element $b \in B$ is called \emph{integral} over $A$ provided that there exists some \emph{monic} nonzero polynomial $f \in A[X]$ such that $f(b) = 0$.
  We say that $B$ is \emph{integral over $A$} if all the elements of $B$ are integral over $A$.
\end{definition}

If $b_{1},\ldots,b_{n} \in B$ are given, then the image of the morphism $A[X_{1},\ldots,X_{n}] \to B$ will be denoted by $A[b_{1},\ldots,b_{n}]$.
This is precisely the $A$-subalgebra of $B$ which is generated by $b_{1},\ldots,b_{n}$.
Recall that the algebra $B$ is said to be \emph{of finite type} provided that $B$ is finitely generated, meaning that $B = A[b_{1},\ldots,b_{n}]$ for some choice of $b_{1},\ldots,b_{n} \in B$.
A stronger notion of finiteness is being a \emph{finite} algebra, which means that $B$ is finitely generates as an $A$-module.

\begin{proposition}
  Let $B$ be an $A$-algebra.
  Let $b_{1},\ldots,b_{n} \in B$ be given.
  Then $b_{1},\ldots,b_{n}$ are all integral over $A$ if and only if $A[b_{1},\ldots,b_{n}]$ is finite as an $A$-algebra.
\end{proposition}
\begin{proof}
  First we focus on the forward direction.
  Argue by induction on $n$.
  The case $n = 0$ is trivial.
  In general, note that $A[b_{1},\ldots,b_{n+1}] = A[b_{1},\ldots,b_{n}][b_{n+1}]$.
  By induction $A[b_{1},\ldots,b_{n}]$ is finite over $A$, so it sufficies (by transitivity of finiteness) to assume $n = 1$, and put $b = b_{1}$.
  In this case, let $g$ be a nonzero monic polynomial over $A$ such that $g(b) = 0$.
  Note that $A[b]$ is a quotient of $A[X]/(g)$, which is easily seen to be finitely-generated as a module over $A$, as it is generated by the images of $1,X,\ldots,X^{d-1}$ where $d = \deg(g)$.
  The same therefore holds for $A[b]$.

  Conversely, suppose that $A[b_{1},\ldots,b_{n}]$ is generated as an $A$-module by $c_{1},\ldots,c_{k}$, and let $b \in A[b_{1},\ldots,b_{n}]$ be arbitrary.
  We have for every $i$,
  \[ b \cdot c_{i} = \sum_{j} a_{i,j} c_{j}, \ \ a_{ij} \in A. \]
  In particular, letting $I$ denote the $k \times k$ identity matrix and $M = (a_{ij})$, we have
  \[ (b \cdot I - M) \cdot (c_{1},\ldots,c_{k})^{t} = 0. \]
  Let $L$ denote the adjugate of $N := b \cdot I - M$ and recal that $L \cdot N = \det(N) \cdot I$ to deduce that
  \[ 0 = L \cdot N \cdot (c_{1},\ldots,c_{k})^{t} = \det(N) \cdot (c_{1},\ldots,c_{k})^{t}, \]
  and thus $\det(N) \cdot c_{i} = 0$ for all $i$.
  As $c_{1},\ldots,c_{k}$ generate $A[b_{1},\ldots,b_{n}]$, it follows that $\det(N)$ annihilates all of $A[b_{1},\ldots,b_{n}]$, and thus $\det(N) = 0$ since $\det(N)$ annihilates $1$.
  But $\det(N) = \det(b \cdot I - M)$ is a nonzero monic polynomial in $b$ with coefficients in $A$, so that $b$ is integral over $A$, as contended.
\end{proof}

In particular, this proposition shows that if $b_{1},b_{2} \in B$ are integral over $A$, then so is every other element $b$ of $A[b_{1},b_{2}]$, since $A[b_{1},b_{2}] = A[b,b_{1},b_{2}]$.
It follows that the sums and products of any integral elements are again integral, and thus the collection of all elements of $B$ which are integral over $A$ forms a subalgebra of $B$.

\begin{definition}
  Let $B$ be an $A$-algebra.
  The set
  \[ \bar A = \{b \in B \ | \ \text{$b$ is integral over $A$}\}, \]
  which is an $A$-subalgebra of $B$ as noted above, is the \emph{integral closure} of $A$ in $B$.
  Note that the structure map $f : A \to B$ factors as
  \[ A \to \image(f) \subset \bar A \subset B. \]
\end{definition}

%%% Local Variables:
%%% mode: latex
%%% TeX-master: "main"
%%% End:
