\section{Lecture 2}

In this course, a \emph{ring} will always be \emph{commutative} with $1$.
If $A$ is a ring, then an $A$-algebra will be a ring endowed with a \emph{structure morphism} $A \to B$, which will almost always be left implicit.
A morphism $B_{1} \to B_{2}$ of $A$-algebras is a morphism of rings which is compatible with the structure morphisms in the obvious way.

Recall that $A[X_{1},\ldots,X_{n}]$, the polynomial ring in $n$ variables, is the free $A$-algebra on the set $\{X_{1},\ldots,X_{n}\}$.
Thus, if $B$ is an $A$-algebra and $b_{1},\ldots,b_{n} \in B$ are given, then there is a unique morphism of $A$-algebras
\[ A[X_{1},\ldots,X_{n}] \to B \]
satisfying $X_{i} \mapsto b_{i}$.
This map is called \emph{evaluation} at $b_{1},\ldots,b_{n}$, and will be denoted in the usual way as
\[ f(X_{1},\ldots,X_{n}) \mapsto f(b_{1},\ldots,b_{n}). \]

If $b_{1},\ldots,b_{n} \in B$ are given, then the image of the morphism $A[X_{1},\ldots,X_{n}] \to B$ will be denoted by $A[b_{1},\ldots,b_{n}]$.
This is precisely the $A$-subalgebra of $B$ which is generated by $b_{1},\ldots,b_{n}$.

\begin{definition}
  Let $A$ be a ring and $B$ an $A$-algebra.
  An element $b \in B$ is called \emph{integral} over $A$ provided that there exists some \emph{monic} nonzero polynomial $f \in A[X]$ such that $f(b) = 0$.
  We say that $B$ is \emph{integral over $A$} if all the elements of $B$ are integral over $A$.
\end{definition}

%%% Local Variables:
%%% mode: latex
%%% TeX-master: "main"
%%% End:
