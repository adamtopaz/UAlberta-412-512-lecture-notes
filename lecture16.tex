\section{Lecture 16}

In this section we discuss the ramification and splitting behavior of primes in cyclotomic extensions.

\begin{theorem}
  Let $n = \prod_{p} p^{v_{p}}$ be the prime factorization of $n$.
  For a prime $p$, let $f_{p}$ denote the order of $p$ in the unit group of $\Zbb/m$ where $m = n/p^{v_{p}}$.
  Then in $\Qbb(\mu_{n})$, the prime $p$ splits as follows:
  \[ p \cdot \Ocal_{K} = (\pfrak_{1} \cdots \pfrak_{k})^{\phi(p^{v_{p}})} \]
  where $\pfrak_{i}$ are distinct primes lying above $p$, all of degree $f_{p}$.
\end{theorem}
\begin{proof}
  Let $\xi$ be a primitive $n$-th roof of unity.
  We need to understand how the minimal polynomial of $\xi$ splits modulo $p$.
  Call this polynomial $\Phi_{n}$.

  Write $n = p^{v_{p}} \cdot m$.
  One has
  \[ \mu_{n} = \mu_{p^{v_{p}}} \times \mu_{m} \]
  and thus
  \[ \Phi_{n} = \prod_{a \in \mu_{m}, b \in \mu_{p^{v_{p}}}}(X - ab). \]
  Also, $X^{p^{v_{p}}} - 1 = (X-1)^{p^{v_{p}}}$ mod $p$, so that the $b$'s above are alll congruent to $1$ modulo primes lying above $p$.
  Thus:
  \[ \Phi_{n} = \Phi_{m}^{\phi(p^{v_{p}})} \bmod \pfrak \]
  where $\pfrak$ is any prime lying above $p$, and this implies a similar congruence modulo $p$.
  This reduces to the case where $p$ does not divide $n$.

  As $\kappa(\pfrak)$ has characteristic $p$ which does not divide $n$ by our reduction, the polynomial $X^{n}-1$ is separable over $\Fbb_{p}$.
  Thus $\mu_{n}$ maps isomorphically under the quotient map
  \[ \Ocal \to \kappa(\pfrak). \]
  The extension of $\Fbb_{p}$ generated by $\mu_{n}$ is $\Fbb_{p^{f_{p}}}$  (note that $n$ divides $p^{f_{p}}-1$ which is the size of the unit group of this field extension).
  The reduction of $\Phi_{n}$ mod $p$ therefore splits in $\Fbb_{p^{f_{p}}}$.
  If $\bar\Phi_{n}$ factorizes as $q_{1} \cdots q_{r}$, then each of the $q_{i}$ is the minimal polynomial of a primitive $n$-th root of unity in $\Fbb_{p^{f_{p}}}$, and thus the degree must be $f_{p}$.
\end{proof}

Here are some special cases of the above theorem.
A prime number $p$ is raimfied in $K := \Qbb(\mu_{n})$ if and only if $p$ divides $n$, in almost all cases.
The exception is when $2$ divides $n$ exactly once, and in that case $2$ is unramified (why?).
An odd prime is totally split in $K$ if and only if $p$ is congruent to $1$ modulo $n$.

Let's also see some examples of inertia and decomposition groups.
Consider $K := \Qbb(\mu_{n})$.
Let's first focus on \emph{odd} primes which are unramified in $K$, which is equivalent to saying that $p$ does not divide $n$ as noted above.
Recall that the Galois group of $K$ over $\Qbb$ is $(\Zbb/n)^{\times}$.
Since $p$ is unramified, its factorization in $K$ is
\[ \pfrak_{1} \dots \pfrak_{r} \]
where $\pfrak_{i}$ is a prime of degree $f_{p}$ as above.
The fundamental equality tells us that $\phi(n) = r \cdot f_{p}$.
The inertia groups are trivial (due to our unramifiedness condition), so the decomposition group must be the same as the Galois group of the residue field extension.
And in this case, the residue Galois group can be identified with the subgroup of $(\Zbb/n)^{\times}$ generated by $p$, which has order $f_{p}$ by assumption.
We will see later that $p$ here agrees with the Frobenius element at $p$ (with some natural choice of normalization).

What about the ramified case?
Suppose that $p$ divides $n$, and write $n = m p^{v_{p}}$ as before.
Then the inertia groups have size $\phi(p^{v_{p}})$ and the decomposition group modulo the inertia group is $\Zbb/f_{p}$, which we identify with the subgroup of $(\Zbb/m)^{\times}$ generated by $p$ as above.
The Galois group in this case is
\[ (\Zbb/n)^{\times} = (\Zbb/p^{v_{p}})^{\times} \times (\Zbb/m)^{\times},\]
The inertia group is $(\Zbb/p^{v_{p}})^{\times}$ and the decomposition group is the subgroup generated by this inertia group together with the image of $p$ in the second component.

\begin{exercise}
  Verify all of the assertions made above.
\end{exercise}

%%% Local Variables:
%%% mode: latex
%%% TeX-master: "main"
%%% End:
