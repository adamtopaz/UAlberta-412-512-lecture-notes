\section{Lecture 20}

There is another variant of Hensel's lemma whose proof I will only sketch.
\begin{theorem}
  Let $K$ be a complete nonarchemedean valued field and $f \in \Ocal[X]$.
  Let $a \in \Ocal$ be an element such that
  \[ |f(a)| < |f'(a)|^{2}. \]
  Then there exists a unique $b \in \Ocal$ such that $f(b) = 0$ and $|b-a| < |f'(a)|$.
  Moreover,
  \[ |b-a| = |f(a)/f'(a)|, \ |f'(b)| = |f'(a)|. \]
\end{theorem}
\begin{proof}[Proof sketch]
  This is essentialy Newtons method, adapted to this context.
  Start with $a = a_{0}$ and define iteratively
  \[ a_{n+1} = a_{n} - f(a_{n})/f'(a_{n}). \]
  Show by induction on $n$ that:
  \begin{enumerate}
    \item $|a_{n}| \leq 1$
    \item $|f'(a_{n})| = |f'(a_{0})|$.
    \item $|f(a_{n})| \leq |f'(a_{0})| t^{2^{n-1}}$.
  \end{enumerate}
  Here $t = |f(a)/f'(a)| < 1$.
  Conclude by letting $b$ be the limit of the $a_{n}$.
\end{proof}

To finish off today, let's discuss some applications.

\begin{example}
  The polynomial $X^{p-1}-1$ splits completely in $\Zbb_{p}$.
  To see this, apply Hensel's lemma to $X^{p-1}-1$, using that $\Fbb_{p}$ has $p-1$ distinct solutions to this equation.
  This shows that $\mu_{p-1}$ is contained in $\Zbb_{p}$.
  In particular, going back to the beginning of today's lecture, the map
  \[ \Zbb_{p} \to \Fbb_{p} \]
  has a \emph{multiplicative} section with image $\mu_{p-1} \cup \{0\}$.
  In fact, there is in some sense a ``canonical'' choice for such a section, where the image of $t \in \Fbb_{p}$ is called the \emph{Teichmuller lift} of $t$.
  In any case, if we put $R = \mu_{p-1} \cup \{0\}$, this gives us a system of representatives $R$ to which we can apply the discussion in the beginning of today's lecture.
  More on this later!
\end{example}

\begin{corollary}
  Suppose $K$ is a complete nonarchemedean field.
  Every irreducible polynomial $f$ with constant coefficient $a_{0}$ and leading coefficient $a_{n}$ such that $a_{n} a_{0} \neq 0$ has
  \[ |f| = \max_{i}\{|a_{i}|\} = \max(|a_{0}|,|a_{n}|). \]
  In particular, if $f$ is monic and $a_{0}$ is integral, then $f$ is integral.
\end{corollary}
\begin{proof}
  WLOG assume that $|f| = 1$.
  Let $a_{r}$ be such that $|a_{r}| = 1$ with $r$ minimal.
  Thus
  \[ f = x^{r}(a_{r} + a_{r+1} x + \cdots + a_{n} x^{n-r}) \bmod \mfrak. \]
  If $\max(|a_{0}|,|a_{n}|) < 1$ then $0 < r < n$ and the congruence above would contradict Hensel's lemma.
\end{proof}

\begin{theorem}
  Suppose that $K$ is complete with respect to a nonarchemedean absolute value and let $L$ be a finite extension of $K$.
  Then $|-|$ has a unique extension to $L$ given by
  \[ |x| = \sqrt[n]{|N_{L|K}(x)|}. \]
  Also, $L$ is again complete with respect to this absolute value.
\end{theorem}
\begin{proof}
  Completeness is a general fact about finite dimensional vector spaces over complete fields, and we omit it (see Proposition 4.9 in Neukirch Ch II).

  Let $\Ocal_{K}$ be the valuation ring of $K$ and write $\Ocal_{L}$ for the integral closure of $\Ocal_{K}$ in $L$.
  We claim that $\Ocal_{L}$ consists precisely of those elements of $L$ whose norm is integral.
  This is clearly a necessary condition (why?), but let's see that it's sufficient.
  Suppose that $x \in L$ is a nonzero element whose norm is integral, and consider the minimal polynomial $f$ of $x$ over $K$.
  The norm is (upto $\pm 1$) a power of the constant coefficient of $f$, and is thus integral hence of absolute value $\leq 1$.
  But $\Ocal_{K}$ is normal (why?) and this shows that the constant coefficient of $f$ is contained in $\Ocal_{K}$.
  It follows from the previous corollary that all the coefficients of $f$ are in $\Ocal_{K}$, hence $x$ is integral.

  Now consider the function $|x| = \sqrt[n]{N_{L|K}(x)}$ in question.
  It is clear that $|x| = 0$ if and only if $x = 0$ and that this function is multiplicative.
  We must thus check the ultrametric inequality, and for this it suffices to show that $|x| \leq 1$ implies $|x+1| \leq 1$ (why?).
  But this is clear from the above, and it is furthermore clear from the argument that $\Ocal_{L}$ is the valuation ring associated to this absolute value.

  Now for the uniqueness, we can proceed in a few ways.
  First, one can use general results from valuation theory to see that the valuations extending that of $K$ are characterized as localizations at maximal ideals of $\Ocal_{L}$ and since we proved above that this is a valuation ring, hence local, it follows that this is unique (once it's normalized properly).
  But let's do this in a more hands-on way, as follows.

  Suppose that $|-|_{i}$, $i = 1,2$ are two absolute values extending that of $K$ and write $\Ocal_{i}$ for their valuation rings.
  Suppose that $x \in \Ocal_{1} \smin \Ocal_{2}$.
  Write $f$ for the minimal polynomial of $x$.
  As proved above, the coefficients of $f$ are all in $\Ocal_{K}$, while $x^{-1}$ is in the maximal ideal of $\Ocal_{2}$.
  This eventually lets us see that $1$ is in the maximal ideal of $\Ocal_{2}$, which is nonsense.
  Thus $\Ocal_{1} \subset \Ocal_{2}$ and by symmetry the two are equal, so the absolute values are equivalent as their valuation rings are equal.
  Since the two absolute values agree on $K$, they must be actually \emph{equal}.
\end{proof}
\begin{remark}
  Using Ostrowski's theorem discussed in the last lecture, the same holds in the Archemedean case as well.
\end{remark}

Let us now turn our attention to a more restrictive class of complete nonarchemedean valued fields, namely, the \emph{local fields}.
For our purposes, a \emph{local field} is a field which is complete with respect to a nonarchemedean absolute value whose assocaited valuation is discrete, and whose residue field is finite.

\begin{remark}
  Some authors define a local field to be a field with a nontrivial absolute whose assocaited topology is locally compact.
  One can prove that all such fields are isomorphic to one of the following:
  \begin{enumerate}
    \item $\Rbb$ or $\Cbb$.
    \item Finite extensions of $\Qbb_{p}$.
    \item Finite extensions of $\Fbb_{p}((t))$.
  \end{enumerate}
  In our definition of \emph{local field}, we are omitting the archemedean case.
\end{remark}

%%% Local Variables:
%%% mode: latex
%%% TeX-master: "main"
%%% End:
