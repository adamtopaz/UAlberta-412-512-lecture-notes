\section{Lecture 20}

There is another variant of Hensel's lemma whose proof I will only sketch.
\begin{theorem}
  Let $K$ be a complete nonarchemedean valued field and $f \in \Ocal[X]$.
  Let $a \in \Ocal$ be an element such that
  \[ |f(a)| < |f'(a)|^{2}. \]
  Then there exists a unique $b \in \Ocal$ such that $f(b) = 0$ and $|b-a| < |f'(a)|$.
  Moreover,
  \[ |b-a| = |f(a)/f'(a)|, \ |f'(b)| = |f'(a)|. \]
\end{theorem}
\begin{proof}[Proof sketch]
  This is essentialy Newtons method, adapted to this context.
  Start with $a = a_{0}$ and define iteratively
  \[ a_{n+1} = a_{n} - f(a_{n})/f'(a_{n}). \]
  Show by induction on $n$ that:
  \begin{enumerate}
    \item $|a_{n}| \leq 1$
    \item $|f'(a_{n})| = |f'(a_{0})|$.
    \item $|f(a_{n})| \leq |f'(a_{0})| t^{2^{n-1}}$.
  \end{enumerate}
  Here $t = |f(a)/f'(a)| < 1$.
  Conclude by letting $b$ be the limit of the $a_{n}$.
\end{proof}

To finish off today, let's discuss some applications.

\begin{example}
  The polynomial $X^{p-1}-1$ splits completely in $\Zbb_{p}$.
  To see this, apply Hensel's lemma to $X^{p-1}-1$, using that $\Fbb_{p}$ has $p-1$ distinct solutions to this equation.
  This shows that $\mu_{p-1}$ is contained in $\Zbb_{p}$.
  In particular, going back to the beginning of today's lecture, the map
  \[ \Zbb_{p} \to \Fbb_{p} \]
  has a \emph{multiplicative} section with image $\mu_{p-1} \cup \{0\}$.
  In fact, there is in some sense a ``canonical'' choice for such a section, where the image of $t \in \Fbb_{p}$ is called the \emph{Teichmuller lift} of $t$.
  In any case, if we put $R = \mu_{p-1} \cup \{0\}$, this gives us a system of representatives $R$ to which we can apply the discussion in the beginning of today's lecture.
  More on this later!
\end{example}

\begin{corollary}
  Suppose $K$ is a complete nonarchemedean field.
  Every irreducible polynomial $f$ with constant coefficient $a_{0}$ and leading coefficient $a_{n}$ such that $a_{n} a_{0} \neq 0$ has
  \[ |f| = \max_{i}\{|a_{i}|\} = \max(|a_{0}|,|a_{n}|). \]
  In particular, if $f$ is monic and $a_{0}$ is integral, then $f$ is integral.
\end{corollary}
\begin{proof}
  WLOG assume that $|f| = 1$.
  Let $a_{r}$ be such that $|a_{r}| = 1$ with $r$ minimal.
  Thus
  \[ f = x^{r}(a_{r} + a_{r+1} x + \cdots + a_{n} x^{n-r}) \bmod \mfrak. \]
  If $\max(|a_{0}|,|a_{n}|) < 1$ then $0 < r < n$ and the congruence above would contradict Hensel's lemma.
\end{proof}

\begin{theorem}
  Suppose that $K$ is complete with respect to a nonarchemedean absolute value and let $L$ be a finite extension of $K$.
  Then $|-|$ has a unique extension to $L$ given by
  \[ |x| = \sqrt[n]{|N_{L|K}(x)|}. \]
  Also, $L$ is again complete with respect to this absolute value.
\end{theorem}
\begin{proof}
  Completeness is a general fact about finite dimensional vector spaces over complete fields, and we omit it (see Proposition 4.9 in Neukirch Ch II).

  Let $\Ocal_{K}$ be the valuation ring of $K$ and write $\Ocal_{L}$ for the integral closure of $\Ocal_{K}$ in $L$.
  We claim that $\Ocal_{L}$ consists precisely of those elements of $L$ whose norm is integral.
  This is clearly a necessary condition (why?), but let's see that it's sufficient.
  Suppose that $x \in L$ is a nonzero element whose norm is integral, and consider the minimal polynomial $f$ of $x$ over $K$.
  The norm is (upto $\pm 1$) a power of the constant coefficient of $f$, and is thus integral hence of absolute value $\leq 1$.
  But $\Ocal_{K}$ is normal (why?) and this shows that the constant coefficient of $f$ is contained in $\Ocal_{K}$.
  It follows from the previous corollary that all the coefficients of $f$ are in $\Ocal_{K}$, hence $x$ is integral.

  Now consider the function $|x| = \sqrt[n]{N_{L|K}(x)}$ in question.
  It is clear that $|x| = 0$ if and only if $x = 0$ and that this function is multiplicative.
  We must thus check the ultrametric inequality, and for this it suffices to show that $|x| \leq 1$ implies $|x+1| \leq 1$ (why?).
  But this is clear from the above, and it is furthermore clear from the argument that $\Ocal_{L}$ is the valuation ring associated to this absolute value.

  Now for the uniqueness, we can proceed in a few ways.
  First, one can use general results from valuation theory to see that the valuations extending that of $K$ are characterized as localizations at maximal ideals of $\Ocal_{L}$ and since we proved above that this is a valuation ring, hence local, it follows that this is unique (once it's normalized properly).
  But let's do this in a more hands-on way, as follows.

  Suppose that $|-|_{i}$, $i = 1,2$ are two absolute values extending that of $K$ and write $\Ocal_{i}$ for their valuation rings.
  Suppose that $x \in \Ocal_{1} \smin \Ocal_{2}$.
  Write $f$ for the minimal polynomial of $x$.
  As proved above, the coefficients of $f$ are all in $\Ocal_{K}$, while $x^{-1}$ is in the maximal ideal of $\Ocal_{2}$.
  This eventually lets us see that $1$ is in the maximal ideal of $\Ocal_{2}$, which is nonsense.
  Thus $\Ocal_{1} \subset \Ocal_{2}$ and by symmetry the two are equal, so the absolute values are equivalent as their valuation rings are equal.
  Since the two absolute values agree on $K$, they must be actually \emph{equal}.
\end{proof}
\begin{remark}
  Using Ostrowski's theorem discussed in the last lecture, the same holds in the Archemedean case as well.
\end{remark}

Let us now turn our attention to a more restrictive class of complete nonarchemedean valued fields, namely, the \emph{local fields}.
For our purposes, a \emph{local field} is a field which is complete with respect to a nonarchemedean absolute value whose assocaited valuation is discrete, and whose residue field is finite.

\begin{remark}
  Some authors define a local field to be a field with a nontrivial absolute whose assocaited topology is locally compact.
  One can prove that all such fields are isomorphic to one of the following:
  \begin{enumerate}
    \item $\Rbb$ or $\Cbb$.
    \item Finite extensions of $\Qbb_{p}$.
    \item Finite extensions of $\Fbb_{p}((t))$.
  \end{enumerate}
  In our definition of \emph{local field}, we are omitting the archemedean case.
\end{remark}

\begin{proposition}
  A local field is locally compact and its valuation ring is compact.
\end{proposition}
\begin{proof}
  It sufffices to show that the valuation ring is compact.
  We know that this valuation ring $\Ocal$ has the form
  \[ \Ocal \cong \varprojlim_{n} \Ocal/\mfrak^{n} \]
  while multiplication by $\pi^{n}$ ($\pi$ a uniformizer) induces an isomorphism between $\Ocal/\mfrak$ and $\mfrak^{n}/\mfrak^{n+1}$.
  This shows that $\Ocal/\mfrak^{n}$ is finite hence compact.
  By general topology, the same thus holds for the inverse limit and thus also for $\Ocal$.
\end{proof}

\begin{theorem}
  The local fields are the finite extensions of $\Qbb_{p}$ or of $\Fbb_{p}((t))$.
\end{theorem}
\begin{proof}[Proof Sketch]
  It is easy to see from what we have already done that such finite extensions are local, so we only focus on the forward direction.
  We will have to use Ostrowski's theorem, which we recall (without proof) during lecture.

  Let $K$ be a local field with absolute value $|-|$ and valuation $v$.
  Assume first that $K$ has characteristic $0$.
  The restriction of $v$ to $\Qbb$ is thus $v_{p}$ for some prime $p$ (Ostrowski), which must then be the residue characteristic of $K$.
  Taking the closure of $\Qbb$ in $K$ must therefore give the completion of $\Qbb$ with respect to the $p$-adic valuation, i.e.~$\Qbb_{p} \subset K$.
  To see that this extension is finite, one uses the local compactness of $K$.
  If $K$ has positive characteristic then we proceed similarly with $\Fbb_{p}(t)$ instead of $\Qbb$, where $t$ is a uniformizer of $K$.
\end{proof}

Let $K$ be a local field.
We consider the following invariants of $K$:
\begin{enumerate}
  \item The valuation ring $\Ocal = \Ocal_{K}$ with maximal ideal $\mfrak$.
  \item The absolute value $|-|$ on $K$.
  \item The residue field $\kappa := \Ocal/\mfrak$, which is \emph{finite}.
  \item The \emph{size} of the residue field $\kappa$, denoted by $q$.
  \item $U^{(n)}$ as before.
\end{enumerate}

Let us start by studying the multiplicative structure of a local field using these invariants.
\begin{theorem}
  Let $K$ be a local field.
  Suppose that $K$ has characteristic $0$.
  Then there is a (topological) isomorphism
  \[ K^{\times} \cong \Zbb \times \Zbb/(q-1) \times \Zbb/p^{a} \times \Zbb_{p}^{d} \]
  where $p^{a}$ is the (finite) size of $\mu_{p^{\infty}}(K)$ and $d = [K:\Qbb_{p}]$.
  If $K$ has positive characteristic $p$ then
  \[ K^{\times} \cong \Zbb \times \Zbb/(q-1) \times \Zbb_{p}^{\Nbb}. \]
\end{theorem}

We will need to prove some intermediate results and discuss some additional constructions before we can actually prove this.
Let's start.

\begin{proposition}
  Let $K$ be a local field and choose a uniformizer $\pi$ of $K$.
  The canonical map
  \[ \pi^{\Zbb} \times \mu_{q-1} \times U^{(1)} \to K^{\times}. \]
  is a (topological) isomorphism.
\end{proposition}
\begin{proof}
  \todo{Use Hensel's lemma plus general nonsense.}{Add proof.}
\end{proof}

At this point, our task is to describe the structure of $U^{(1)}$.
We will do this by considering analogues of the exponential and logarithm in the $p$-adic case.
In the equal characteristic case, we can use the discussion from previous lectures to see that $K \cong \Fbb_{q}((t))$, and do things ``explicitly'' (I won't prove this, see the argument in Neukirch, which follows an argument of Iwasawa).

\begin{proposition}
  Suppose $K = \Fbb_{q}((t))$.
  Then $U^{(1)} \cong \Zbb_{p}^{\Nbb}$.
\end{proposition}

From now on, let's focus on the $p$-adic case and assume $K$ is $p$-adic.
\begin{lemma}
  The series
  \[ \log(1+x) := x - x^{2}/2 + x^{3}/3 - \cdots \]
  converges in $K$ for $|x| < 1$.
\end{lemma}
\begin{proof}
  Let's compute the valuation of the terms of the series.
  \[ v(x^{n}/n) = n v(x) - v(n). \]
  But $v(x) > 0$.
  Say $\pi$ is a uniformizer and $v(\pi) = c$ so that $v(x) = m \cdot c$ for some positive integer $m$.
  Renormalize and choose $c$ so that $v(p) = 1$.
  Thus $p^{v(n)} \le n$, so that $v(n) \le \log(n)/\log(p)$.
  Put $r := p^{v(x)} > 1$.
  Thus we have
  \[ v(x^{n}/n) = n v(x) - v(n) \geq n \log_{p}(r) - \log_{p}(n) = \log_{p}(r^{n}/n). \]
  But $r^{n}/n$ goes to $\infty$ as $n \to \infty$ since $r > 1$ and so the same holds after taking logs.
  It follows that $x^{n}/n$ is a nullsequence so the series converges.
\end{proof}

This defines a function $\log : U^{(1)} \to K$ which is easily seen to satisfy $\log(x y) = \log(x) + \log(y)$ by the formal properties of the series.
\todo{We can extend this to all of $K^{\times}$ by setting $\log(p) = 0$.}{Explain how.}
In fact, $\log$ is uniquely determined by these properties and is a continuous homomorphism.

%%% Local Variables:
%%% mode: latex
%%% TeX-master: "main"
%%% End:
