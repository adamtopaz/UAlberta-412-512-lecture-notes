\section{Lecture 5}

Let's start with a quick addendum to the last lecture, specficially regarding ideals in localizations.
Let $A$ be a domain and $S$ a multiplicative submonoid of $A$ not containing zero, and view $S^{-1}A$ as an $A$-algebra via the inclusion $A \hookrightarrow S^{-1}A$.

Let $\afrak$ be an ideal of $A$.
Then $S^{-1}\afrak := \afrak \cdot S^{-1}A$ is an ideal of $S^{-1}A$.
If $\Afrak$ is an ideal of $S^{-1}A$ then $\Afrak \cap A$ is an ideal of $A$.
It turns out that $\Afrak = S^{-1}(\Afrak \cap A)$.
On the other hand, in general we have $\afrak \subset S^{-1}\afrak \cap A$, but equality does not always hold.
Instead, $S^{-1}\afrak \cap A$ is the \emph{saturation} of $\afrak$ with respect to $S$, i.e.~the ideal $\tilde \afrak$ consisting of all $a \in A$ such that $s \cdot a \in \afrak$ for some $s \in S$.

For prime ideals $\pfrak$ of $A$, the inflation $S^{-1}\pfrak$ is again prime, and if $\pfrak \cap S = \varnothing$, then $\pfrak$ is saturated.
Conversely, if $\Pfrak$ is a prime of $S^{-1}A$, then $\Pfrak \cap A$ is a saturated prime of $A$.

This discussion shows that inflation/intersection induce an order-preserving bijection between the set of prime ideals of $A$ which are disjoint from $S$, and the prime ideals of $S^{-1}A$.
In particular, for a prime $\pfrak$ of $A$, the prime ideals of the localization $A_{\pfrak}$ are in bijection with the prime ideals $\qfrak$ of $A$ such that $\qfrak \cap (A \smin \pfrak) = \varnothing$, which is equivalent to the condition that $\qfrak \subset \pfrak$.
The ring $A_{\pfrak}$ thus has a unique maximal ideal, which is the inflation of $\pfrak$ itself.

Let's now go back to the context of the last lecture, where $L|K$ is a finite separable extension, $A$ is a normal domain whose fraction field is $K$ and $B$ is the normalization of $A$ in $L$.
\begin{proposition}
  Let $\Pfrak$ be a prime of $B$ lying above a prime $\pfrak$ of $A$.
  Then $\Pfrak$ is maximal if and only if $\pfrak$ is maximal.
\end{proposition}
\begin{proof}
  The extension $A/\pfrak \to B/\Pfrak$ is integral.
  Assume $\pfrak$ is maximal.
  Then $A/\pfrak$ is a field while every element of $B/\Pfrak$ is algebraic over $A/\pfrak$ so that $B/\Pfrak$ is a field, hence $\Pfrak$ is maximal.
  Conversely, suppose $\Pfrak$ is maximal.
  If $A/\pfrak$ is not a field, then it has a nonzero maximal ideal $\mfrak$, and by Proposition~\ref{proposition:exists_prime_above},  there exists a prime ideal $\Mfrak$ of $B/\Pfrak$ above $\mfrak$, which must be zero, but this is impossible since $\mfrak$ was assumed to be nonzero.
\end{proof}

As a corollary, we now know that $\Ocal_{K}$ is Dedekind whenever $K$ is a number field.
A similar argument can be used to show the following theorem.
\begin{theorem}
  Let $k$ be a field, and $L$ a finite separable extension of $k(t)$.
  Then the integral closure of $k[t]$ in $L$ is a Dedekind domain.
\end{theorem}
\begin{proof}
  See homework 1.
\end{proof}

We wish to stress that the discussion above lends itself to computations.
In this case, suppose that $B = A[\alpha]$ where $\alpha$ is a primitive element of $L|K$, and let $f$ be the minimal polynomial of $\alpha$ over $K$.
Let $\mfrak$ be a maximal ideal of $A$, and write $k$ for the quotient $A/\mfrak$.
Write $\bar f$ for the image of $f$ in $k[X]$.

\begin{theorem}
  There is a bijection between the prime (hence maximal) ideals $\Mfrak$ of $B$ lying above $\mfrak$ and the irreducible factors of $\bar f$.
\end{theorem}
\begin{proof}
  Exercise.
\end{proof}

In general, we will not be able to find an $\alpha$ such that $B = A[\alpha]$, but this can always be done locally (i.e.~after a suitable localization).
Similarly, we can get a similar result for prime ideals by localizing to reduce to the maximal case.

Let's now turn to the issue of factorization of ideals in a Dedekind domain.
To hint at the main case of interest which is the ring of integers of a number field, we switch notations and use $\Ocal$ to denote some Dedekind domain with fraction field $K$.

Given two ideals $\afrak$ and $\bfrak$ in any ring, we write $\afrak | \bfrak$ provided that $\bfrak \subset \afrak$.
We also write $\afrak \cdot \bfrak$ for the smallest ideal containing $a \cdot b$ for any $a \in A$ and $b \in B$.
Explicitly, elements of $\afrak \cdot \bfrak$ are finite sums of the form
\[ \sum_{i} a_{i} b_{i}, \ \ a_{i} \in \afrak, \ \ b_{i} \in \bfrak. \]

Our goal is now to prove the following theorem.
\begin{theorem}\label{theorem:ideal_factorization}
  Every nonzero ideal $\afrak$ of $\Ocal$ admits an essentially unique factorization
  \[ \afrak = \pfrak_{1} \cdots \pfrak_{k}, \]
  where $\pfrak_{i}$ are nonzero prime ideals.
\end{theorem}

We will actually prove more.
\begin{definition}
  A \emph{fractional ideal} of $K$ is a finitely-generated nonzero $\Ocal$-submodule of $K$.
\end{definition}

\begin{theorem}\label{theorem:fractional_ideal_factorization}
  Every fractional ideal $\afrak$ admits a unique representation of the form
  \[ \afrak = \prod_{\pfrak} \pfrak^{v_{\pfrak}}, \]
  where $v_{\pfrak} \in \Zbb$, $\pfrak$ varies over the nonzero primes of $\Ocal$, and all but finitely many of the $v_{\pfrak}$ are zero.
\end{theorem}

Thus, the collection $J_{K}$ of all fractional ideals is the free abelian group on the nonzero prime ideals of $\Ocal$.
We will also see that a fractional ideal $\afrak$ is integral (i.e.~it is an ideal of $\Ocal$) if and only if all of the $v_{\pfrak}$ are nonnegative.
The unit of $J_{K}$ is the unit ideal $(1)$ and the inverse $\afrak^{-1}$ is given by
\[ \afrak^{-1} = \{x \in K \ | \ x \cdot \afrak \subset \Ocal \}. \]

Any element $x$ of $K^{\times}$ defines a \emph{principal} fractional ideal $(x)$, and the map
\[ K^{\times} \to J_{K}, \ \ x \mapsto (x) \]
is a homomorphism of abelian groups whose kernel is $\Ocal^{\times}$.
The group of principal fractional ideals is denoted by $P_{K}$.
\begin{definition}
  The cokernel of $K^{\times} \to J_{K}$ is called the \emph{class group of $\Ocal$}, and is denoted $Cl_{K}$.
\end{definition}
In particular, we have an exact sequence of the form
\[ 1 \to \Ocal^{\times} \to K^{\times} \to J_{K} \to Cl_{K} \to 1. \]


%%% Local Variables:
%%% mode: latex
%%% TeX-master: "main"
%%% End:
