\section{Lecture 22}

In this lecture and the next, we will discuss the main \emph{statements} from local class field theory.
In a nutshell, \emph{local class field theory} (henceforth abbreviated LCFT) is a theory which describes abelian Galois theory over local fields, say $K$, using the \emph{arithmetic} of $K$.

Let $K$ be a local field and choose a separable closure $K^{s}$ once and for all.
Consider the absolute Galois group $\Gal_{K}$ of $K$, defined as $\Gal_{K} := \Gal(K^{s}|K)$.
This is a topological group endowed with a \emph{profinite topology} arising from the fact that
\[ \Gal_{K} = \varprojlim_{L}\Gal(L|K) \]
where $L|K$ varies over the finite Galois subextensions of $K^{s}|K$.
Each $\Gal(L|K)$ above is endowed with the discrete topology, which is compact Hausdorff and totally disconnected (i.e.~a profinite group).
The equations defining the limit are closed conditions, and thus the limit is again profinite.

At the heart of LCFT is Artin's \emph{local reciprocity map}, which is a continuous homomorphism of topological groups
\[ \rec_{K} : K^{\times} \to \Gal_{K}^{\ab}. \]

Let me start by giving a brief indication for how such a map is constructed cohomologically.
There are other constructions as well.
First, we need the key fact about Brauer groups of local fields.
\begin{theorem}
  Let $K$ be a local field.
  There is a natural isomorphism
  \[ \Br(K) \cong \Qbb/\Zbb. \]
  If $L|K$ is a finite extension of degree $n$, then the map $\Br(K) \to \Br(L)$ corresponds to multiplication by $n$ on $\Qbb/\Zbb$.
  The kernel $\Br(L|K)$ is thus isomorphic to $(1/n)\Zbb/\Zbb$.
\end{theorem}

With this theorem at hand, we can construct the reciprocity map as follows.
Let $L|K$ be a Galois extension of $K$ of degree $n$ with Galois group $G$.
The \emph{fundamental class} of $L|K$ is defined as the element of $\Br(L|K) = \HH^{2}(G,K^{\times})$ whose image in $\Qbb/\Zbb$ is $1/n$.
For abelian $L|K$, we would like to define a map
\[ K^{\times} \to \Gal(L|K) \]
and finally take a limit in $L$.
As $L|K$ is finite of degree $n$, $\Br(L|K)$ is all $n$-torsion.
Consider
\[ \HH^{2}(L|K,\mu_{n}) = \HH^{2}(K,\mu_{n}) = \frac{1}{n}\Zbb/\Zbb, \]
The cup product gives us a pairing
\[ \HH^{1}(K,\mu_{n}) \times \HH^{1}(K,\frac{1}{n}\Zbb/\Zbb) \to \HH^{2}(K,\mu_{n}) = \frac{1}{n}\Zbb/\Zbb. \]
The Kummer map gives us $K^{\times} \to \HH^{1}(K,\mu_{n})$, and thus we obtain a natural map
\[ K^{\times} \to \Gal_{K}^{\ab} \otimes \Zbb/n. \]
As $\Gal(L|K)$ is killed by $n$, the map $\Gal_{K}^{\ab} \to \Gal(L|K)$ factors through $\Gal_{K}^{\ab} \otimes \Zbb/n$, and so we obtain a natural map $K^{\times} \to \Gal(L|K)$.
One then checks that this map is compatible in extensions in $L$, and takes a limit as $L$ varies over all finite abelian extensions of $K$ to obtain the map in question
\[ K^{\times} \to \Gal_{K}^{\ab}. \]

The ``proper'' way to define this map is actually via \emph{Tate} cohomology, which I will not discuss.
What I will explain is that a consequence of this approach is that the map
\[ K^{\times} \to \Gal(L|K) \]
is surjective for $L|K$ finite abelian and its kernel is $N_{L|K}(K^{\times})$.
Thus $K/N_{L|K}(K^{\times}) \cong \Gal(L|K)$, and $\Gal(L|K)$ is isomorphic to the \emph{norm completion} of $K^{\times}$.

One must then classify precisely which subgroups of $K^{\times}$ are norm subgroups of finite abelian extensions of $K$.
This is called the \emph{existence theorem} in LCFT, which asserts that any finite index subgroup is such a group.

\begin{theorem}[The existence theorem]
  Let $K$ be a $p$-adic field, and let $U$ be a finite index subgroup of $K^{\times}$.
  There exists a unique finite abelian extension $L$ of $K$ such that $U = \Norm_{L|K}(L^{\times})$.
\end{theorem}

\begin{corollary}
  The reciprocity map $K^{\times} \to \Gal_{K}^{\ab}$ induces an isomorphism after profinite completions.
  Thus $\Gal_{K}^{\ab} \cong \hat\Zbb \times U_{K}$.
  Furthermore, the reciprocity map is \emph{injective}.
\end{corollary}

What about the relationship with ramification?
Let $L|K$ be a finite Galois extension with group $G$.
Recall that $K$ is Henselian, and thus $K$ is its own decomposition field.
Let $T \subset G$ denote the inertia group, so the quotient $G/T$ is isomoprhic to the Galois group of the residue field extension.

Now if $L|K$ is abelian, we have a surjection
\[ K^{\times} \to \Gal(L|K) \]
arising from LCFT.
We also know that $K^{\times} = \Zbb \times U_{K}$.

We need a small calculation.
\begin{lemma}
  Suppose that $L|K$ is unramified so that $\Gal(L|K)$ is identified with the Galois group of $\kappa(L)|\kappa(K)$, and is thus generated by Frobenius $F$.
  Then $\rec_{L|K}(x) = F^{v(x)}$ where $v$ is the normalized valuation of $K$.
\end{lemma}

As a corollary to this, we find that the image of $U_{K}$ in $\Gal(L|K)$ is the inertia group.
In the profinite setting, we see that the image of the units $U_{K}$ is the inertia group of $\Gal_{K}^{\ab}$.
\todo{I will draw a diagram in class.}{Typeset this diagram here.}

At this point, taking the statements for granted, we deduce that the maximal abelian extension of a $p$-adic field $K$ has the following form
\[ K^{\ab} = K^{ur} \cdot K_{\pi} \]
where $K_{\pi}$ is the fixed field of $\rec(\pi)$ for some chosen uniformizer $\pi$.
We understand exactly how to describe $K^{ur}$, the maximal unramified extension of $K$.
All such extensions arise by extending the residue field $\kappa$, which has size $q$.
If $\Fbb_{q^{r}}$ is such an extension, then we can obtain the analogous unramified extension of $K$, for example, by applying Hensel's lemma to the polynomial $X^{q^{r}-1}-1$.
In any case, we see that $K^{ur}$ can be obtained by adjoining roots of unity of order $q^{r}-1$ as $r$ varies.

But what about the \emph{ramified} part $K_{\pi}$?
By general ramification theory together with LCFT, note that $K_{\pi}|K$ must be \emph{totally} ramified, meaning that the residue field of $K_{\pi}$ is the same as that of $K$.
Also, note that $\Gal(K_{\pi}|K) = U_{K}$ via LCFT.

One way to describe $K_{\pi}$ explicitly is to use \emph{Lubin-Tate theory}, which I will now briefly summarize.

%%% Local Variables:
%%% mode: latex
%%% TeX-master: "main"
%%% End:
