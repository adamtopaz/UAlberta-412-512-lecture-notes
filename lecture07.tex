\section{Lecture 7}

Our goal for the next few lectures is to discuss the finiteness of class groups of number fields, via Minkowski theory.

\begin{definition}
  Let $V$ be a finite-dimensional real vector space.
  A \emph{lattice} $L$ in $V$ is a subgroup of the form
  \[ \Zbb \cdot v_{1} + \cdots + \Zbb \cdot v_{n}, \]
  where $v_{1},\ldots,v_{n}$ are linearly independent.
\end{definition}

The span, say $W$, of $L$ is the span of $v_{1},\ldots,v_{n}$, and so a fundamental system of representatives for $W/L$ is given by
\[ \{ x_{1}v_{1} + \cdots + x_{n} v_{n} \ | \ x_{i} \in [0,1)\}. \]
We call this set \emph{the fundamental mesh} of $L$.
If $W = V$, then $L$ is called a \emph{complete lattice}.

\begin{proposition}
  A subgroup $L$ of $V$ is a lattice if and only if it is discrete.
\end{proposition}
\begin{proof}
  If $L$ is a lattice, then we leave it as an exercise to see that $L$ is discrete.

  Conversely, assume that $L$ is discrete.
  Let $W$ denote the span of $L$, and choose a basis $v_{1},\ldots,v_{n}$ of $W$ which is contained in $L$.
  Let $L_{0}$ be the $\Zbb$-span of $v_{1},\ldots,v_{n}$, so that $L_{0} \subset L$.

  It turns out that $L/L_{0}$ is finite.
  Indeed, let $S \subset L$ be a system of representatives of $L/L_{0}$.
  Consider the fundamental mesh $F_{0}$ associated to $v_{1},\ldots,v_{n}$.
  For each $t \in S$, we can thus write $t = x + y$, where $x \in F_{0}$ and $y \in L_{0}$.
  The elements $x \in F_{0}$ are contained in $F_{0}$ and in $L$.
  Since $L$ is discrete and $F_{0}$ is bounded, there are only finitely many such $x$.

  From this it follows that
  \[ L \subset \frac{1}{[L:L_{0}]} \cdot L_{0}, \]
  and since $L_{0}$ is of finite rank free as a $\Zbb$-module, the same holds for $L$.
\end{proof}

\begin{lemma}
  A lattice $L$ in $V$ is complete if and only if there exists a bounded subset $M$ such that $M + L = V$.
\end{lemma}
\begin{proof}
  The forward direction is clear.
  Conversely, suppose we have such an $M$.
  Write $W$ for the span of $L$, and let $v \in V$ be given.
  Consider the multiples $n \cdot v$ for $n \in \Nbb$.
  Each one can be expressed as
  \[ n \cdot v = l_{n} + m_{n}, \ \ l_{n} \in L, \ \ m_{n} \in M. \]
  \todo{Since $M$ is bounded, the $1/n \cdot m_{n}$ converge to zero.}{Why?}
  Now we can write $v = 1/n \cdot l_{n} + 1/n \cdot m_{n}$, and take the limit in $n$ (along with the fact that $W$ is closed in $V$) to see that indeed $v \in W$.
\end{proof}

We will be particularly interested in the \emph{covolume} of a complete lattice.
In this case, we are considering real vector spaces $V$ endowed with an inner product $(-,-)$.
This allows us to discuss \emph{volumes} in $V$.
Indeed, if $e_{1},\ldots,e_{n}$ is an orthonormal basis, then the cube
\[ [0,1] \cdot e_{1} + \cdots + [0,1] \cdot e_{n} \]
has volume $1$.
More generally, if $v_{1},\ldots,v_{n}$ are a basis, then the cube spanned by the $v$s has volume $|\det M|$ where $M$ is the change-of-basis matrix from $e_{1},\ldots,e_{n}$ to $v_{1},\ldots,v_{n}$.

\begin{lemma}
  Let $L$ be a complete lattice spanned by $v_{1},\ldots,v_{n}$ in an inner product space $V$.
  Then the volume of the fundamental mesh of $L$ does not depend on the choice of $v_{1},\ldots,v_{n}$.
  We call this volume the \emph{covolume of $L$}.
\end{lemma}
\begin{proof}
  \todo{If $w_{1},\ldots,w_{n}$ is another basis of $L$, then the change of basis matrix from $v$ to $w$ has determinant $\pm 1$.}{Explain further.}
\end{proof}

Let us now come back to number theory.
Let $K$ be a number field with ring of integers $\Ocal_{K}$.
Consider the base-change
\[ K_{\Cbb} := K \otimes_{\Qbb} \Cbb. \]
Explicitly, suppose that $n = [K:\Qbb]$, and $\tau_{1},\ldots,\tau_{n}$ are the \todo{$n$}{Explain why there are $n$ such embeddings.} embeddings of $K$ into $\Cbb$.
Then one has
\[ K_{\Cbb} = \prod_{i = 1}^{n} \Cbb \]
and the canonical map $K \to K_{\Cbb}$ is induced by the $n$ embeddings $\tau_{1},\ldots,\tau_{n}$, so that for $x \in K$, its image in $K_{\Cbb} = \prod_{i} \Cbb$ is $(\tau_{i}(x))_{i}$.
Endow $K_{\Cbb} = \Cbb^{n}$ with the standard (Hermitian) inner product.

The Galois group $\Gal(\Cbb|\Rbb)$ acts on $K_{\Cbb}$ via \todo{the second component.}{Explain how this action behaves when we identify $K_{\Cbb}$ with $\Cbb^{n}$ as above.}
Complex conjugation generates this Galois group, so the invariants,
\[ K_{\Rbb} := K_{\Cbb}^{\Gal(\Cbb|\Rbb)} \]
are exactly $K_{\Cbb}^{\sigma}$, where $\sigma$ is complex conjugation.
Clearly, we have $K_{\Rbb} = K \otimes_{\Qbb} \Rbb$, and the map $K \to K_{\Cbb}$ factors through $K_{\Rbb}$.

We can describe $K_{\Rbb}$ explicitly as follows.
Let $\rho_{1},\ldots,\rho_{r}$ denote the \emph{real} embeddings of $K$, and $\sigma_{1},\bar\sigma_{1},\ldots,\sigma_{s},\bar\sigma_{s}$ the complex-conjugate pairs of the strictly complex embeddings of $K$.
We thus have $n = r + 2s$.
We then have
\[ K_{\Rbb} = \prod_{i = 1}^{r} \Rbb \times \prod_{i = 1}^{s} \Cbb \]
and the map $K \to K_{\Rbb}$ acts via $(\rho_{1},\ldots,\rho_{r},\sigma_{1},\ldots,\sigma_{s})$.

Next, observe that the inner product on $K_{\Cbb}$ induces an inner product on $K_{\Rbb}$.
We describe it explicitly as follows.
\begin{lemma}
  Identify $K_{\Rbb}$ with $\Rbb^{r + 2s}$ as real vector spaces.
  The inner product on $K_{\Rbb}$ corresponds to
  \[ ((x_{1},\ldots,x_{r},x'_{1},\ldots,x'_{2s}),(y_{1},\ldots,y_{r},y'_{1},\ldots,y'_{2s})) = \sum_{i = 1}^{r} x_{i} y_{i} + 2 \cdot \sum_{j = 1}^{2s} x'_{j} y'_{i}. \]
\end{lemma}
\begin{proof}
  \todo{Omitted.}{Add proof from class.}
\end{proof}

The difference between the inner product on $K_{\Rbb}$ and the standard inner product on $\Rbb^{r + 2s}$ is thus a factor of $2^{s}$.
We will have to keep this factor in mind later on.

Let us recall some facts about discriminants.
First, recall that any nonzero (fractional) ideal of $\Ocal_{K}$ is a free $\Zbb$-module of rank $n = [K:\Qbb]$.
If $x_{1},\ldots,x_{n}$ is a basis, then the discriminant of this basis is defined as
\[ \det((\tau_{i}x_{j})_{i,j})^{2}. \]
If $x'_{1},\ldots,x'_{n}$ is another basis, then the change of basis matrix from $x$ to $x'$, say $M$, has determinant $\pm 1$, while
\[ d(x'_{1},\ldots,x'_{n}) = \det(M)^{2} d(x_{1},\ldots,x_{n}). \]
It follows that this discriminant does not depend on the choice of basis, and so we simply write $d(\afrak)$ when $\afrak$ is a fractional ideal.
We also write $d_{K} = d(\Ocal_{K})$.

The relationship between these discriminants is summarized as follows.
\begin{lemma}
  Suppose that $\afrak \subset \bfrak$ are fractional ideals.
  then $[\bfrak:\afrak]$ is finite and one has $d(\afrak) = [\bfrak : \afrak]^{2} \cdot d(\bfrak)$.
\end{lemma}
\begin{proof}
  Exercise.
\end{proof}

\begin{proposition}\label{proposition:covolume_ideal_computation}
  Let $\afrak$ be a nonzero ideal of $\Ocal_{K}$, and let $L$ denote its image in $K_{\Rbb}$.
  Then $L$ is a complete lattice whose covolume is $\sqrt{|d_{K}|} \cdot [\Ocal_{K}:\afrak]$.
\end{proposition}
\begin{proof}
  Let $x_{1},\ldots,x_{n}$ be a $\Zbb$-basis for $\afrak$, hence $L$ is spanned by the images of $x_{1},\ldots,x_{n}$ in $K_{\Rbb}$.
  We have
  \[ d(\afrak) = [\Ocal_{K}:\afrak]^{2} d_{K}. \]
  Let $\tau_{1},\ldots,\tau_{n}$ denote all the (complex) embeddings, and put $M = (\tau_{i} x_{j})_{i,j}$.
  The covolume of $L$ is thus
  \[ \sqrt{|d_{K}|} \cdot [\Ocal_{K}:\afrak] = |\det(A)| = |\det(A \bar A^{t})|^{1/2} = |\det((\tau_{i}x_{j},\tau_{i}x_{k})_{j,j}))|^{1/2}. \]
\end{proof}

%%% Local Variables:
%%% mode: latex
%%% TeX-master: "main"
%%% End:
